% arara: pdflatex: { shell: yes }
% arara: pythontex: {verbose: yes, rerun: modified }
% arara: pdflatex: { shell: yes }
% arara: move: {  files: ['chapter7.pdf'], target: ['..'] }


\documentclass[bh_problems_check]{subfiles}

\opt{check}{
\Opensolutionfile{hint}
\Opensolutionfile{ans}
}

\begin{document}



\setcounter{chapter}{6}
\chapter{Chapter 7}
\label{cha:questions-chapter-7}



\subfile{bh-7.1.tex}
\subfile{bh-7.9.tex}
\subfile{bh-7.10.tex}


\setcounter{theorem}{10}
\begin{exercise}
BH.7.11
\begin{hint}
a.
First find $f_{Y|X}$ and $f{Z|X}$. Then, given $X$, $Z$ and $Y$ are iid. Hence $f_{X,Y,Z} = f_{Y, Z |X} f_{X}$. Use independence to split $f_{Y,Z|X}$ into a product.


b.
Suppose that  a realization of $Y$ is really big. Since $Y$ is dependent on $X$, $X$ must be dependent on $Y$. But $Z$ is in turn dependent on $X$. What are the consequences?

\end{hint}
\begin{solution}
a. Use the hint. Next, $f_{Y|X}(y|x) \propto e^{-(y-x)^{2}}$, and a similar expression holds for $f_{Z|X}(z|x)$. Now follow the steps of the hint.

b. Read the hint. When $Y$ is really big, $X$ must be big (with large probability), so $Z$ must be big too.

 c.
Here is the answer. The ideas are important, you'll need them during nearly any course in statistics, given the importance of the normal distribution.
\begin{align*}
f_{Y,Z}(y, z) = \int \frac 1 {2\pi} e^{-(y-x)^2/2} e^{-(z-x)^2/2} \frac 1 {\sqrt{2\pi}} e^{-x^2/2}\d x.
\end{align*}
It remains to simplify $(y-x)^2 + (z-x)^2 + x^2$. With a bit of work, it follows that this can be written as
\begin{align*}
3(x-(y+z)/3)^{2} - (y+z)^2/3+y^2+z^2.
\end{align*}
When plugging this in the integral, the last two terms appear in front of the integral. The term $(y+z)/2$ is just a shift, hence can be neglected in the integration over $x$. The $3$ has to be absorbed in the standard deviation $\sigma=1/\sqrt{3}$. And therefore,
\begin{align*}
f_{Y,Z}(y, z) = \frac 1 {2\pi} \frac 1 {\sqrt{3}} e^{-y^2/2 - z^2/2 + (y+z)^2/6}.
\end{align*}
\end{solution}
\end{exercise}

\setcounter{theorem}{12}
\begin{exercise}
BH.7.13
\begin{hint}
\cref{sec:memoryl-exerc-conf} contains all the explanations.
\end{hint}
\begin{solution}
Read the material of \cref{sec:memoryl-exerc-conf} for many detailed explanations on the exponential. As we will not repeat that, here are just the results.
$\P{X<Y} = 1/2$. Hence, $\P{X\leq x|X<Y} = 2\P{X\leq x, X<Y}$.
\begin{align*}
2\P{X\leq x, X<Y}
  &= 2\lambda^2\int_0^{\infty}\int_0^{\infty}   \1{u\leq x} \1{u<v} e^{-\lambda u} e^{-\lambda v} \d v \d u \\
  &= 2\lambda^2\int_0^{\infty}   \1{u\leq x} e^{-\lambda u} \int_0^{\infty} \1{u<v}  e^{-\lambda v} \d v \d u\\
  &= 2\lambda\int_0^{\infty}   \1{u\leq x} e^{-\lambda u} e^{-\lambda u}  \d u \\
  1 - e^{-2\lambda x}.
\end{align*}

b. If $X<Y$, then we know that $X=\min\{X, Y\}$. But,
\begin{align*}
  \{\min\{X,Y\}\leq x\} =  \{X\leq x, X<Y\} \cup \{Y\leq x, Y\leq X\},
\end{align*}
and the two sets on the RHS are disjoint. Hence, $\P{\min\{X,Y\}\leq x}$ is the sum of the probabilities on the RHS. By symmetry, these are equal.
\end{solution}
\end{exercise}

\setcounter{theorem}{14}
\begin{exercise}
BH7.15.
\begin{hint}
Make a drawing.
\end{hint}
\begin{solution}
Use the hint, that is, really make the drawing of the rectangle mentioned in the exercise. (If you refuse to to this, then nothing can  help you to understand the rest of the answer.) Then, in the drawing, note that $F(x,y)$ is the area of an (infinite) square lying  south west of the point $(x,y)$. Add and subtract such (infinite) squares until the square $[a_1,a_2] \times [b_{1},b_2]$ is covered exactly once. Realize that in the process, the square $(-\infty, a_1] \times (-\infty, b_1]$ is subtracted twice.
\end{solution}
\end{exercise}


\setcounter{theorem}{23}
\begin{exercise}
BH.7.24.
In the assignments we'll develop a simulator.

\begin{hint}
Check BH.7.1.24 and BH.7.1.25
First draw the area over which we have to integrate. Then use an indicator function over which to integrate. What is the joint PDF  $f_{Y_1, Y-2}$?
\end{hint}
\begin{solution}
a. From the hint,
\begin{align*}
\P{Y_1<c Y_2}
  &= \int \int \1{x<c y}\lambda_1 e^{-\lambda_1 x} \lambda_2e^{-\lambda_2 y}\d x \d y
  = \lambda_1\lambda_2\int_0^{\infty} e^{-\lambda_{1}x}\int_{x/c}^{\infty} e^{-\lambda_{2} y}\d y \d x\\
  & = \lambda_1\int_0^{\infty} e^{-\lambda_{1}x} e^{-\lambda_{2} x/c} \d x
  = \frac{\lambda_1}{\lambda_1+\lambda_2/c}.
\end{align*}
Check the result for $c=0$ and $c=\infty$.

I prefer to use conditioning, like this:
\begin{align*}
\P{Y_1<c Y_2}
  &= \int \P{Y_1<cY_2| Y_1=x}\lambda_1 e^{-\lambda_1 x} \d x
  = \int \P{Y_2>x/c| Y_1=x}\lambda_1 e^{-\lambda_1 x} \d x\\
&= \int e^{-\lambda x/c} \lambda_1 e^{-\lambda_1 x} \d x,
\end{align*}
and the rest goes as before. Actually, I tend to use conditioning as it helps to make the reasoning easier. In this case, suppose that I know that $Y_1=x$, what can I say about $\P{Y_2 > c x}$?

BTW, conditioning does not always make things simpler. When rvs are dependent, then you have to watch out.

b. See the solutions of BH on the web.
\end{solution}
\end{exercise}


\setcounter{theorem}{28}
\begin{exercise}
BH.7.29
\begin{solution}
All is covered in~\cref{sec:memoryl-exerc-conf}.
\end{solution}
\end{exercise}

\setcounter{theorem}{37}
\begin{exercise}
BH.7.38. Besides the solution of BH, read our solution.
\begin{solution}
First check~\cref{ex:3a}.

In general, I am always very careful with such `shortcuts' such as $\max\{X,Y\} + \min\{X, Y\} = X +Y$.  As a matter of fact, I try to avoid such arguments because it is easy to go wrong. Seemingly plausible arguments are often wrong due to overlooked dependency or non-linearity (effects of higher moments).

It is useful to write $\max\{x,y\} = x\1{x\geq y}+y\1{y>x}$, and something similar for the minimum. In the present case, $\cov{X,Y} = \E{X Y}-\E X \E Y$, and, similarly, $\cov{M,L} = \E{ML}- \E M \E L$, where $M$ is max, and $L$ is min. With the above indicators, it is simple to show that $\E{ML} = \E{X Y}$:
\begin{align*}
 ML
  &= (X\1{X\geq Y} + Y\1{Y\geq X})((X\1{X<Y} + Y\1{Y<X}) \\
  &= XY\1{X\geq Y} + XY\1{Y<X} = XY
\end{align*}
since $\1{X\geq Y}\1{X<Y} =0$.

However, take $X,Y\sim \Exp{\lambda}$. Then, $\E M = 3/(2\lambda)$ and $\E L= 1/(2\lambda)$, but $\E X = \E Y = 1/\lambda$.
\end{solution}
\end{exercise}

\setcounter{theorem}{52}
\begin{exercise}
BH.7.53. We simulate this in one of the assignments.
The ideas of this exercise find much use in finance, physics, and actuarial sciences.
In particular, the expected time it takes the drunken person---It's not only guys that sometimes consume too much alcohol---to hit some boundary is interesting. The notation of the book is a bit clumsy. Here is better notation.
Let $X_i$ be the movement along the \(x\)-axis at step $i$, and $Y_i$ along the $y$-axis.
Then $S_n=\sum_{i=1}^n X_i$ and $T_n=\sum_{j=1}^n Y_{j}$, and $R_n^2= S_n^2+T_n^2$.
\begin{hint}
Use the hint of the book and independence to see that $\E{S_{n}^2 T_n^2} = \E{S_n^{2}} \E{T_n^{2}}$.
Then try to simplify.

b. It is immediate that $\E{S_n} = 0$.
Hence, focus on $\E{S_n T_n}$. Expand  the sums of $\E{S_n T_n}$, and consider the individual terms $\E{X_i Y_j}$. When $i\neq j$, are $X_i$ and $Y_{j}$  independent? What if  $i=j$?

c. It is clear that $R_n^2=S_n^2+T_{n}^2$. Now use linearity to split $\E{R^2_n}$. Finally, realize that $\E{S_n}=0$, hence $\E{S_n^2} = \V{S_n}$. But then we can use the formula of the variance of a sum to split it up into a sum of variances plus covariances.
\end{hint}

\begin{solution}
a. In my notation, $X_i=0 \implies Y_i\neq 0$ and $X_i\neq 0 \implies Y_i=0$. The reason is that in step $i$, the drunkard makes a step left or right OR up or down. However, s/he cannot move to the right and up at the same time.

Here is an argument based on recursion. (By now I hope you see that I like this method in particular).
\begin{align*}
\E{R_n^2} = \E{(R_{n-1} + X_n + Y_n)^{2}},
\end{align*}
but $R_{n-1}$ and $X_n+Y_n$ are independent, and $\E{(X_n + Y_n)^2} = 1$. Using the recursion, $\E{R^2_n} = n$.
\end{solution}
\end{exercise}


\setcounter{theorem}{57}
\begin{exercise}
BH.7.58.
This is a totally great exercise. First solve it yourself. In the solution, I'll explain why, in particular how to relate the concept of covariance to the determinant of a matrix.

\begin{hint}
a. Expand the brackets in the expression for the sample variance $r$ to see that
\begin{align*}
r = 1/n \sum_i x_i y_i - \bar x \bar y.
\end{align*}
Next, we choose with probability $1/n$ one the points $(x_i, y_{i})$.  Under this probability, $\E{X Y} = 1/n \sum_i x_i y_i$, $\E X = \bar x, \E Y = \bar y$. So, how do $\cov{X,Y}$ and $r$ relate?


b. Expand the brackets and use iid and linearity properties to show that the expected area spanned by two random points $(X,Y)$ and $(\tilde X, \tilde Y)$ satisfies
\begin{align*}
\E{(X-\tilde X)(Y-\tilde Y)} = 2\cov{X,Y}.
\end{align*}

\end{hint}
\begin{solution}

b. Use the hint. Then, if we choose two points at random from the sample, then $(x_i-x_j)(y_i-y_j)$ is the area spanned by these  two points.
More generally, I have $n$ choices for my first point, and also $n$ choices for the second point (if both points are the same, the area of the rectangle is 0, so we don't have to exclude such choices).
Hence, the expected area of the rectangle spanned by the two random points $(X,Y)$ and $(\tilde X, \tilde Y)$ is
\begin{align*}
\frac 1 {n^2} \sum_{i,j} (x_i-x_j)(y_i-y_j).
\end{align*}
Simplify this to show that
\begin{align*}
2 \frac 1 n \sum_i x_i y_i - 2 \bar x \bar y = 2 r
\end{align*}
Hence, by part a., the expected area is twice the covariance.

Why is $\cov{X,a}= 0$ for $a$ a constant? Because the `area' of rectangles, all with the same \(y\)-coordinate, is zero, i.e., they lie on a line.

c.  This is the part of the exercise that explains what the above is all about.
Since there is a direct relation between covariance and area, we can use geometric arguments to derive (and memorize!) all properties of covariance! Write property i. of covariance  as $\cov{X,Y} = \cov{Y,X}$. Suppose I flip the \(x\) and \(y\)-axis, does the area of a rectangle change?  For property ii., what happens to the area of rectangle if you stretch the sides? For property iii., realize that this is just a shift of a rectangle that leaves its area invariant. For property iv., what happens to the area if you put an extra rectangle on top or to the right?

BTW, property iii. follows directly from property iv. In iv., take $W_3$ equal to a constant $a_2$, in other words $\P{W_3=a_2}=1$. We know that $\cov{X, a} = 0$ for a constant $a$.

Here are  some final remarks.

Let's put all the above in a very general frame.  The covariance has a number of interesting properties:
\begin{enumerate}
\item  It is bilinear, that is, the covariance is linear in both arguments. The linearity in the first argument means that $\cov{X+Y, Z}=\cov{X,Z}+\cov{Y,Z}$ and $\cov{a X, Z}=a\cov{X,Z}$ for $a\in \R$. The linearity in the second argument means that $\cov{X, Y+Z}=\cov{X,Y}+\cov{X,Z}$ and $\cov{X, a Z}=a\cov{X,Z}$ for $a\in \R$.
\item It is symmetric: $\cov{X, Y}=\cov{Y,X}$, from which we define $\V X = \cov{X,X}$.
\item  $\cov{X,a} = 0$ for all $a\in R$.
\end{enumerate}
If you memorize the first two properties of covariance, all the rest follows.

Now we do some geometry. Take three vectors $x,y, z\in \R^2$ (it's easy to generalize to $\R^n)$. Then we know that the area $D(x,y)$ of the parallelogram spanned by vectors $x$ and $y$  satisfies the following properties.
\begin{enumerate}
\item  Area is bilinear. The linearity in the first argument means that $D(x+y, z) = D(x, z) + D(y, z)$ and $D(ax, z)=a D(x, z)$ for $a\in \R$. (Just make a drawing to convince you about this.) The linearity in the second argument means that  $D(x, y+z) = D(x, y) + D(x, z)$  and $D(x, a z)=a D(x, z)$ for $a\in \R$.
\item  $D(x,x)$ = 0; there is no area between $x$ and $x$.
\item $D( (1,0), (0,1)) = 1$; the area of the square with side 1 is 1.
\end{enumerate}
In fact, thex first property means that stretching vectors and stacking parallelograms result in stretching and adding areas.
The second says that the area of a parallelogram spanned by two parallel vectors is zero. The third specifies that the area of the unit square is 1.

Now it can be proven that there exists just one function $D$ that satisfies these properties. In fact, this is the determinant of the matrix with as columns the vectors that span the parallellogram. Moreover, it can be shown that the second property can be replaced by the skew-symmetric property: $D(x,y) = - D(x,y)$.
(Note that $D(x,x) = -D(x,x) \implies 2 D(x,x) = 0 \implies D(x,x) = 0$.)

Let us use the properties to compute the area of a parallelogram spanned by the vectors $x = (a,b)$ and $y =(c,d)$ in 2D. Then
\begin{align*}
D(x,y) &= D((a,b), (c,d)) =  D(a (1,0)) + b(0,1), c(1,0) + d(0,1)) \\
&= ad D((1,0), (0,1)) + b c D((0,1), (1,0)) = ad - b c,
\end{align*}
where we use bilinearity in the first step, and skew-symmetry in the second and third. And this is indeed the determinant of the matrix with $x$ and $y$ as columns.

So, all in all, this is what I remembered throughout the years: the covariance and the determinant are bi-linear forms, the first is symmetric, the second skew- (or anti-)symmetric.

Finally, I don't see why the areas of the rectangles have to have a sign in this problem. Interestingly, for the determinant, the areas of the parallelograms do have to have a sign to make the concept useful for physics.
\end{solution}
\end{exercise}

\setcounter{theorem}{58}
\begin{exercise}
BH.7.59.
Read this exercise, then read (and do) BH.5.53 for some further background.
You'll encounter these topics countless times in other courses!
The final answer is really nice and intuitive.

\begin{hint}
a. Use that expectation is linear.

b. Read the entire exercise in its entirety before trying to solve it. In this case trying to solve c.\/ seems simpler because of the extra iid assumption. You  might want to use this to formulate some simple guesses.

Thus first part c. It is given that the $X_i$ and $Y_j$ are iid. Then, if I could improve the estimator $\hat \theta$ by splitting the measurements into two sets $X_i$ and $Y_j$, then I would certainly do that.
And not only I would do that; anybody in his right mind would do that.
But, I never heard of this idea, and I am sure you have neither, so this must be impossible (because if it would, people would have been using this trick for ages.)
Hence, we can place this in the context of the maxim: `we cannot obtain information for free'.
For this case, this must imply that splitting iid measurements into smaller sets cannot help with improving the estimator. What does this idea imply for the weights?


Part b, continued. I always try to solve the problem myself without a hint. This lead to the following considerations, which gave me quite a bit of extra understanding beyond the problem itself.  As a next piece of advice, before doing hard work, I prefer to look at some corner cases to acquire some intuitive understanding. I also use the rvs of Part c.

Suppose  that $v_2:=\V{Y_j} = 0$, but $v_1 := \V{X_i} > 0$. (For instance, $Y_j$ is the $j$th measurement of a perfect machine and $X_j$ of an imperfect machine.)
Then we know that the set $\{Y_j\}$ forms a set of perfect measurements.
But then I am not interested in the $\{X_i\}$ measurements anymore; why should I as I have the perfect measurements $\{Y_j\}$ at my disposal.
So, then I put $w_1=0$, because I don't want the $\{X_i\}$ measurements to pollute my estimator.
In other words, the final result should be such that $v_2=0 \implies w_{1}=0$, and vice versa.


More generally, I learned from this  corner case that I want this for the final result:  when $v_2<v_{1} \implies w_1 < w_{2}$, and vice versa.

How would you choose the weights such that this requirement is satisfied, but also the condition imposed by Part c.?
\end{hint}


\begin{solution}
a. Follows directly from the hint.

Check the hint!

c.  If $X_i$ and $Y_j$ are iid, it must be that $w_{1} = n/(n+m)$.

b. Can we make some further progress, just by keeping a clear mind?
Well, in fact we can by using our insights of part c.
If we have $n+m$ iid measurements of which we call $n$ measurements of type $X_i$, and $m$ of type $Y_{j}$, then
\begin{align*}
\V{\hat \theta_{1}}  =  \E{\left(\frac{1}{n}\sum_{i}X_{i} - \theta\right)^2} = n^{-2}\E{\left(\sum_i (X_{i}-\theta)\right)^{2}} = n^{-2}\V{\sum_i X_i} = \V{X_1}/n = \sigma^{2}/n.
\end{align*}
So, $n=\sigma^{2}/\V{\hat \theta_1}$, and likewise $m=\sigma^{2}/\V{\hat \theta_2}$. Finally, plug this into our earlier expression for $w_1$ to  get
\begin{align*}
w_1 = \frac n {n+m} = \frac{\sigma^2/\V{\hat \theta_1}}{\sigma^2/\V{\hat \theta_1} + \sigma^2/\V{\theta_2}} = \frac{\V{\hat \theta_2}}{\V{\hat \theta_1} + \V{\theta_2}}.
\end{align*}
If we check our earlier insight, then we see that if $\V{Y_j}=0$, then $\V{\theta_2}=0$, hence $w_1=0$ in that case. This is precisely what we wanted.

Let us finally use the  hint of BH to check that the above expression for $w_1$ is correct.
\begin{align*}
\E{(\hat \theta -\theta)^{2}} =
\E{(w_1(\hat \theta_{1} - \theta) + w_2(\hat \theta_{2}-\theta))^{2}} =
\V{w_1 \hat \theta_{1}} + \V{w_2 \hat \theta_{2}},
\end{align*}
by independence. Take the $w$'s out of the variances, then write $w_2=1-w_1$, take $\partial_{w_1}$ of the expression,  set the result to 0, and solve for $w_1$. You'll get the above expression.
\end{solution}
\end{exercise}

\setcounter{theorem}{70}
\begin{exercise}
BH.7.71.

\begin{hint}

b. The people in the sample of size $n$ with an $A$ is $X_1+X_2$. But this is the same as $n-X_3$. Hence, what is $\P{X_3=n-i}$?


c. I found this a hard problem.
Here is my hint based on recursion.
Let $S_n$ be the number of $A$s in $n$ individuals.
We want to know $f_n(i) = \P{S_n=i}$.
A simple recursive idea, i.e., one-step analysis by conditioning on the phenotype of the $n$th person, gives that
\begin{align*}
f_n(i)=f_{n-1}(i-2) p^2 + f_{n-1}(i-1) 2p q + f_{n-1}(i)q^2,
\end{align*}
with $q=1-p$ as always. Now I was a bit stuck, but just to try to see whether I could see some structure, I tried a simpler case, namely, a recursion for the binomial distribution. Derive this, and then use this to solve the problem.


d. It is easiest to work with $f(p) = \log \P{X_1=k, X_2=l, X_3 = m}$, where $\P{X_1=k, X_2=l, X_3 = m}$ follows from a., and then differentiate with respect to $p$.

e. Follow the same scheme as for d.
\end{hint}

\begin{solution}
a. Multinomial.

b. With the hint we end up at $X_1+X_2\sim \Bin{n, p^2+2p(1-p)}$.

c. Here is a short intermezzo on finding a recursion for the sum of a number of Bernouilli rvs.  Let $S_n$ be the number of successes in the binomial, and write $g_n(i) = \P{S_n=i}$ for this case.
Then,
\begin{align*}
g_n(i)&= g_{n-1}(i-1)p +  g_{n-1}(i)q \\
&= (g_{n-2}(i-2)p + g_{n-2}(i-1)q)p + (g_{n-2}(i-1)p+g_{n-2}(i)q)q \\
&= g_{n-2}(i-2)p^2 + g_{n-2}(i-1)2p q + g_{n-2}(i)q^{2}.
\end{align*}
I also know that $g_n(i) = {n \choose i} p^iq^{n-i}$.
End of intermezzo.

Now compare the recursion with $f_n(i)$ for the genes tox the expression for the binomial.
They are nearly the same, except that in the genes case, the `n' seems to run twice as fast.
I then tried the guess $f_n(i) = {2n \choose i} p^i q^{2n-i}$.
For you, plug it in, and show that it works.

So, what was my overall approach?
I used recursion, but got stuck.
Then I used recursion for a simpler case whose solution I know by heart.
I compared the recursions for both cases to see whether I could recognize a pattern.
This lead me to a guess, which I verified by plugging it in.
Using recursion is not guaranteed to work, of course, but often it's worth a try.

Now, looking back, I realize that it is as if individual $n$ adds the outcome of two coin flips (with values in $AA$, $Aa$ or $a a$) to the sum $S_{n}$ of $A'$s. For you to solve: what is the distribution of two coin flips? Next, $S_n$ is just the sum of $n$ individual `double coin flips'. Hence, what must the distribution of $S_n$ be?

d. It is easiest to work with $f(p) = \log \P{X_1=k, X_2=l, X_3 = m}$. With part a. this can be written as
\begin{align*}
f(p) = C + (2k+l)\log p + (l+2m)\log(1-p),
\end{align*}
where $C$ is a constant (the log of the normalization constant). (BTW, with this you can check your answer for part a.)
Compute $\d f(p)/\d p = 0$, because at this $p$, $\log f$, hence $f$ itself, is maximal.  Observe that $C$ drops out of the computation, because when differentiating, it disappears.


e. Now we like to know what $p$  maximizes $\P{X_3=n-i}$. Take $g(q) = \log \P{X_3 = n-i}$, then
\begin{align*}
g(q) = C + i \log (1-q^{2}) + 2 (n-i)\log q.
\end{align*}
(With this, check your answer of part b.) Again, take the derivative (with respect to $q$), and solve for $q$.
\end{solution}
\end{exercise}

\setcounter{theorem}{85}
\begin{exercise}
BH.7.86. The concepts discussed here are a standard part of the education of GPs (i.e., medical doctors), and in data science in general.
\begin{hint}
The challenge for you is to try to understand the mathematics behind these concepts.
Read the exercise a number of times. I found it quite difficult to capture the concepts in formulas. (I solved it once. After two weeks,  I tried to solve it again, and found it just as hard as the first time.) Once you have the model, the technical part itself is simple.
\end{hint}
\begin{solution}
a. It is given that $\P{T\leq t\given D=1} = G(t)$ and $\P{T\leq t\given D=0} = H(t)$. From Theorem 5.3.1.i,  we have that we can associate a rv. to a CDF F. Sometimes we say that the CDF $F$ /induces/ a rv. $X$.  So let us use this here to say that $G$ induces the rv. $T_1$ and $H$ induces $T_0$.
So the /sensitivity/ is $\P{T_1>t_0} = 1-G(t_0)$ and the /specificity/ is $\P{T_1<t_0} = H(t_0)$.

To make the ROC plot, I first made two plots, one of the sensitivity and the other for 1 minus the specificity, i.e., $1-H(t_0)$.
Then, in the ROC plot, we put a specificity of $s$ on the \(x\)-axis, then we search for a $t$ such that $1-H(t) = s$, and then we plug this $t$ into $1-G(t)$ to get the sensitivity.
To help you understand this better, check that $s=0 \implies t = b \implies 1-G(t) = 0$.
Moreover, check that $s=1\implies t=a \implies 1- G(t) = 1$.
Hence, the ROC curve starts in the origin and stops at the point $(1,1)$.

With this insight, the area under the ROC curve can be written as
\begin{align*}
\int_0^1 (1-G(H^{-1}(1-s))) \d s  =
1 - \int_0^1 G(H^{-1}(1-s)) \d s  =
1 - \int_a^b G(t) h(t) \d t,
\end{align*}
where, in the last step, we use the 1D change of variable $H(t)=1-s \implies h(t) \d t = -\d s$. It remains to  interpret the integral, so let's plug in the definitions:
\begin{align*}
\int_a^b G(t) h(t) \d t =
\int_a^b \P{T_1\leq t} f_{T_0}(t) \d t =
\int_a^b \P{T_1\leq T_0\given T_0 = t} f_{T_0}(t) \d t =  \P{T_{1\leq T_0}}.
\end{align*}

\end{solution}
\end{exercise}



% \opt{check}{\Closesolutionfile{hint}
% \Closesolutionfile{ans}
% % \begin{Hint}{1.2}
			Note the difference between mean and median. This question sheds light on the link between our informal daily languages and formal mathematical concepts.
		
\end{Hint}
\begin{Hint}{1.5}
			For the median, use the fact that the Cauchy density function is symmetric about $0$.
		
\end{Hint}
\begin{Hint}{1.6}
			Given any random variable $X$ whose distribution is symmetric about some point $\mu$, you can construct a random variable $Y$ that is symmetric about 0. What can you say about $E(Y^3)$ and $E((-Y)^3)$?
		
\end{Hint}
\begin{Hint}{1.7}
			Check from the definition that a random variable $X$ has zero skewness if $E(X) = E(X^3) = 0$. Construct a random variable satisfying this property. The easiest option is to consider a discrete random variable with 3 values in its support.
		
\end{Hint}
\begin{Hint}{1.9}
			Note that variances (if exist) are always non-negative.
%			\textbf{Intuition:} ${\displaystyle \operatorname {E} (X)=\operatorname {P} (X<a)\cdot \operatorname {E} (X|X<a)+\operatorname {P} (X\geq a)\cdot \operatorname {E} (X|X\geq a)}$ where ${\displaystyle \operatorname {E} (X|X<a)}$  is larger than 0 as r.v. ${\displaystyle X}$ is non-negative and ${\displaystyle \operatorname {E} (X|X\geq a)}$  is larger than ${\displaystyle a}$ because the conditional expectation only takes into account of values larger than ${\displaystyle a}$ which r.v. ${\displaystyle X}$ can take. Hence intuitively ${\displaystyle \operatorname {E} (X)\geq \operatorname {P} (X\geq a)\cdot \operatorname {E} (X|X\geq a)\geq a\cdot \operatorname {P} (X\geq a)}$${\displaystyle \operatorname {E} (X)\geq \operatorname {P} (X\geq a)\cdot \operatorname {E} (X|X\geq a)\geq a\cdot \operatorname {P} (X\geq a)}$, which directly leads to ${\displaystyle \operatorname {P} (X\geq a)\leq {\frac {\operatorname {E} (X)}{a}}}$.
		
\end{Hint}
\begin{Hint}{1.10}
		\begin{enumerate}[i.]
			\item Use the fundamental bridge. Note that
			\begin{equation*}
				\begin{array}{cl}
						P(|X-\mu|\geq \epsilon) &= E(1_{\{\left( |X-\mu|\geq \epsilon\right) \}})= E(1_{\{  \frac{|X-\mu|}{\epsilon}\geq 1  \}})
				\end{array}
			\end{equation*}
		\item Show that $1_{\{ \frac{|X-\mu|}{\epsilon}\geq 1  \}}\leq \left( \frac{|X-\mu|}{\epsilon}\right)^2 $.
		\item The above two imply that $P(|X-\mu|\geq \epsilon)\leq E\left( \frac{|X-\mu|}{\epsilon}\right)^2$.
		\end{enumerate}
	
\end{Hint}
\begin{Hint}{1.12}
			Use the result from Ex \ref{ex:chap06:05}.
		
\end{Hint}
\begin{Hint}{1.13}
			First, derive the identity $\sum_{i = 1}^n (X_i - \mu)^2 = \sum_{i = 1}^n (X_i - \bar{X}_n)^2 + n (\bar{X}_n - \mu)^2$.
		
\end{Hint}
\begin{Hint}{1.14}
		Try to make use the fact that sample average of i.i.d. data goes to the expectation by decomposing $S^2$ as the sum of components with sample averages.  $$S_n^2 = \frac{n}{n - 1}\frac{1}{n} \sum_{i = 1}^n (X_i - \mu)^2 - \frac{n}{n - 1} (\bar{X}_n - \mu)^2.$$
	
\end{Hint}
\begin{Hint}{1.15}
			If you were to throw a fair coin a large number of times, what is the proportion of heads you would expect?
		
\end{Hint}
\begin{Hint}{1.16}
			FUse that if $X \sim N(\mu, \sigma^2)$, the MGF of $X$ is given by $M_X(t) = e^{\mu t} e^{\frac{1}{2} \sigma^2 t^2}$.
		
\end{Hint}
\begin{Hint}{1.18}
			Recall the formula for geometric series: for $|\rho| < 1$, $\sum_{k = 0}^{\infty} \rho^k = \frac{1}{1 - \rho}$.
		
\end{Hint}
\begin{Hint}{1.19}
			For $X \sim N(\mu, \sigma^2)$, the MGF of $X$ is given by $M_X(t) = e^{\mu t} e^{\frac{1}{2} \sigma^2 t^2}$. Now take derivatives.
		
\end{Hint}
\begin{Hint}{1.20}
			For $X \sim N(\mu, \sigma^2)$, the MGF of $X$ is given by $M_X(t) = e^{\mu t} e^{\frac{1}{2} \sigma^2 t^2}$. Now take derivatives.
		
\end{Hint}
\begin{Hint}{1.21}
		MGFs determines distributions. Show the MGF of the sum can not be written in the form of the Expo MGF.
	
\end{Hint}
\begin{Hint}{1.22}
		MGF!
	
\end{Hint}
\begin{Hint}{1.23}
		MGF!
	
\end{Hint}
\begin{Hint}{1.24}
		The sum of independent Gaussian is Gaussian, use the fact that $X_1+X_2\sim N(\mu_1+\mu_2, \sigma_1^2+\sigma_2^2)$ when $X_1,X_2$ are independent and $X_i\sim N(\mu_i, \sigma_i^2), i=1,2$.
	
\end{Hint}
\begin{Hint}{1.28}
		MGF!
	
\end{Hint}

% % \begin{Solution}{1.1}
			Let $X = 10^{100} B$, where $B \sim \text{Bern}(10^{-10})$. The mean $\mu$ of $X$ is $10^{100} \cdot 10^{-10} = 10^{90}$, which is very large. In contrast, the median is 0, which is closer to the value $X$ generally takes.
		
\end{Solution}
\begin{Solution}{1.2}
			The first sentence uses the ``Median'', and the ``average level'' refers to the ``Mean''.  The second sentence compares the ``median'' with the ``mean''.
		
\end{Solution}
\begin{Solution}{1.3}
~
			\begin{enumerate}
				\item Let $X$ be a random variable. We want to show that the value of $c$ that minimizes $E(X - c)^2$ is $c = \mu$, where $\mu$ denotes the mean of $X$. We have
				\begin{align*}
					E(X - c)^2 & = E((X - \mu) + (\mu - c))^2 \\
					& = E(X - \mu)^2 + 2 E((X - \mu)(\mu - c)) + E(\mu - c)^2 \\
					& = E(X - \mu)^2 + (\mu - c)^2
				\end{align*}
				It is easily seen that $E(X - c)^2$ is minimal for $c = \mu$.
				\item Let $X$ be a random variable. We want to show that the value of $a$ that minimizes $E|X - a|$ is $a = m$, where $m$ denotes the median of $X$. We want to evaluate $E|X - a|$ for $a \neq m$.
					
				Assume $m < a$. If $X \leq m$, then
				\begin{equation*}
					|X - a| - |X - m| = a - X - (m - X) = a - m.
				\end{equation*}
				If $X > m$, then
				\begin{equation*}
					|X - a| - |X - m| = X - a - (X - m) = m - a.
				\end{equation*}
				Now let $Y = |X - a| - |X - m|$ and let $I = 1$ if $X \leq m$ and $I = 0$ if $X > m$. Then
				\begin{align*}
					E(Y) & = E(YI) + E(Y(1 - I)) \\
					& \geq (a - m) E(I)	+ (m - a) E(1 - I) \\
					& = (a - m) \P{X \leq m} + (m - a) \P{X > m} \\
					& = (a - m) \P{X \leq m} - (a - m) (1 - \P{X \leq m}) \\
					& = (a - m) (2 \P{X \leq m} - 1).
				\end{align*}
				By the definition of a median, we have $2 \P{X \leq m} - 1 \geq 0$. Hence, $E(Y) \geq 0$, which implies $E(|X - m|) \leq E(|X - a|)$. Hence for all $E(|X - m|) \leq E(|X - a|)$ for all $m < a$. Repeat similar steps for $m > a$ and conclude $E|X - a|$ is minimal for $a = m$.
				\item Let $X \sim \text{Bern}(0.25)$. Then the mean of $X$ is $\mu = 0.25$, while the median of $X$ is $m = 0$. Note that $E(X - \mu)^2 = V(X) = 0.25(1 - 0.25) = 0.1875$ and $E(X - m)^2 = E(X)^2 = 0.25$; hence $E(X - \mu)^2 \leq E(X - m)^2$ as expected. Moreover, using LOTUS, $E|X - \mu| = |0 - 0.25|(1 - 0.25) + |1 - 0.25|0.25 = 0.375$ and $E|X - m| = E|X| = E(X) = 0.25$; thus, $E|X - m| \leq E(X - \mu)^2$, as expected.
			\end{enumerate}
		
\end{Solution}
\begin{Solution}{1.5}
			Let $X$ follow a standard Cauchy distribution. The PDF of $X$ is given by $f(x) = \frac{1}{\pi (1 + x^2)}$. Note that $f'(x) = -\frac{2x}{\pi (1 + x^2)}$; hence $f'(x) = 0 \iff x = 0$. It follows that $f$ has a maximum at $x = 0$. (Formally, you have to check $f''(0) < 0$, too.) Since this maximum is unique, the mode of $X$ is $0$. As $f(x) = f(-x) = 0$ for all $x \in \mathbb{R}$, the standard Cauchy distribution is symmetric about $0$. Therefore, $P(X \leq 0) = \int_{-\infty}^0 f_X(x) \mathrm{d}x = \int_{-\infty}^0 f_X(-x) \mathrm{d}x = \int_0^{\infty} f_X(y) \mathrm{d}y = \P{X \geq 0}$. As $\P{X \leq 0} + \P{X \geq 0} = 1$ it follows that $\P{X \leq 0} = \frac{1}{2}$. Hence, by definition, the median of the Cauchy distribution is $0$.
		
\end{Solution}
\begin{Solution}{1.6}
			Let $X$ be a random variable whose distribution is symmetric about its mean $\mu$. Then $Y = X - \mu$ is symmetric about 0. Due to symmetry, $Y$ and $-Y$ have the same distribution. That implies $E(Y^3) = E((-Y)^3)$. This in turn implies $E(Y^3) = 0$. It follows that $\text{Skew}(X) = E\left(\frac{X - \mu}{\sigma}\right)^3 = \frac{1}{\sigma^3} E(Y^3) = 0$.
		
\end{Solution}
\begin{Solution}{1.7}
			There are infinitely many possible asymmetric distributions with zero skewness. Zero skewness means that overall, the tails on both sides of the mean balance out. This occurs, for example, when one tail is ``long" but the other tail is ``fat". An easy example of an asymmetric distribution with zero skewness is obtained by considering a discrete random variable with 3 values in its support. Check from the definition that a random variable $X$ has zero skewness if $E(X) = E(X^3) = 0$. One random variable satisfying this property is the random variable $X$ with $\P{X = -3} = 0.1$, $\P{X = -1} = 0.5$ and $\P{X = 2} = 0.4$. The distribution of $X$ is asymmetric by construction. Verify yourself that $E(X) = E(X^3) = 0$.
		
\end{Solution}
\begin{Solution}{1.8}
			The $r$th central moment is given by
			\begin{align*}
				\mu_r & = \int_a^b \left[x - E(X)\right]^r \cdot f_X(x) \mathrm{d}x = \frac{1}{b - a} \int_a^b \left[x - \frac{b - a}{2}\right]^r \mathrm{d}x = \frac{1}{(b - a) 2^r} \int_a^b \left[2x - (a + b)\right]^r \mathrm{d}x \\
				&= \frac{1}{(b - a) 2^r} \left[\frac{(2x - (a + b))^{r + 1}}{2(r + 1)}\right]_a^b = \frac{1}{(b - a) 2^r} \cdot \frac{(b - a)^{r + 1} - (-1)^{r + 1} (b - a)^{r + 1}}{2(r + 1)}
			\end{align*}
			which is zero when $r$ is odd.
		
\end{Solution}
\begin{Solution}{1.9}
			Correct. Recall that the variance of a random variable $X$ is defined as $V(X) = E(X - E(X))^2$. Because $(X - E(X))^2$ is strictly non-negative, $V(X)$ can only be zero if $(X - E(X))^2$ is always zero (or with probability one). $(X - E(X))^2$ is always zero if and only if $X = E(X)$ with probability one. If $X = E(X)$, $X$ always has the same value, i.e. is constant with probability one. Hence, if a random variable is of zero variance, then it is a constant with probability one.
		
\end{Solution}
\begin{Solution}{1.10}
			\begin{enumerate}[i.]
			\item Use the fundamental bridge. Note that
			\begin{equation*}
				\begin{array}{cl}
					P(|X-\mu|\geq \epsilon) &= E(1_{\{\left( |X-\mu|\geq \epsilon\right) \}})= E(1_{\{  \frac{|X-\mu|}{\epsilon}\geq 1  \}})
				\end{array}
			\end{equation*}
			\item Show that $1_{\{ \frac{|X-\mu|}{\epsilon}\geq 1  \}}\leq \left( \frac{|X-\mu|}{\epsilon}\right)^2 $. For any $s\in S$, if $1_{\{ \frac{|X-\mu|}{\epsilon}\geq 1  \}}(s)=0$, then we know by the non-negativity of the square function $\left( \frac{|X-\mu|}{\epsilon}\right)^2(s)\geq 0=1_{\{ \frac{|X-\mu|}{\epsilon}\geq 1  \}}(s)$;  if $1_{\{ \frac{|X-\mu|}{\epsilon}\geq 1  \}}(s)=1$, then we know by the definition of the indicator function that $\frac{|X-\mu|}{\epsilon}(s)\geq 1 =1_{\{ \frac{|X-\mu|}{\epsilon}\geq 1  \}}(s)$. Therefore, for all outcomes $s\in S$, $1_{\{ \frac{|X-\mu|}{\epsilon}\geq 1  \}}(s)\leq \left( \frac{|X-\mu|}{\epsilon}\right)^2(s)$, and thus $$P\left\{1_{\{ \frac{|X-\mu|}{\epsilon}\geq 1  \}}\leq \left( \frac{|X-\mu|}{\epsilon}\right)^2\right\}=P(S)=1. $$
			\item The above two imply that $P(|X-\mu|\geq \epsilon)\leq E\left( \frac{|X-\mu|}{\epsilon}\right)^2$.
		\end{enumerate}
	
\end{Solution}
\begin{Solution}{1.11}
		We want to show that for some constant $c$ we have that for any $\varepsilon>0$ $\P{|\frac{1}{n}\sum_{i=1}^n(X_{i}-E(X_i))^2-c|>\varepsilon}\rightarrow 0$. Denote $E(X_i)=\mu$ and $Y_n=\frac{1}{n}\sum_{i=1}^n(X_{i}-\mu)^2$, then using the result from the previous exercise we obtain $\P{|Y_n-E(Y_n)|\geq\varepsilon}\leq \frac{Var(Y_n)}{\varepsilon^2}$. By independence of the $X_i$ $Var(Y_n)=Var(Y_n=\frac{1}{n}\sum_{i=1}^n(X_{i}-\mu)^2)=\frac{1}{n}Var((X_i-\mu)^2)\rightarrow 0$, because of $Var((X_i-\mu)^2)$ is finite by the finite fourth moment of $X_i$. We conclude that $\frac{1}{n}\sum_{i=1}^n(X_{i}-E(X_i))^2$ converges to $E(\frac{1}{n}\sum_{i=1}^n(X_{i}-E(X_i))^2)=Var(X_i)$.
	
\end{Solution}
\begin{Solution}{1.12}
			Let $X_1, \ldots, X_n$ be i.i.d. random variables with mean $\mu$ and variance $\sigma^2$. The sample mean is given by $\bar{X}_n = \frac{1}{n} \sum_{i = 1}^n X_i$. The variance of the sample mean is given by
			\begin{align*}
				V(X_n^2) & = V\left(\frac{1}{n} \sum_{i = 1}^n X_i\right) = \frac{1}{n^2} \sum_{i = 1}^n V(X_i) = \frac{1}{n^2} \cdot n \sigma^2 = \frac{\sigma}{n}.
			\end{align*}	
			It follows that $V(X_n^2) \to 0$ as $n \to \infty$. Now invoke the result of Ex \ref{ex:chap06:05} to conclude that the sample mean converges to a constant with probability one.
		
\end{Solution}
\begin{Solution}{1.13}
			Let $X_1, \ldots, X_n$ be i.i.d. random variables with mean $\mu$ and variance $\sigma^2$. The sample mean is given by $\bar{X}_n = \frac{1}{n} \sum_{i = 1}^n X_i$. The sample variance is given by $S_n^2 = \frac{1}{n - 1} \sum_{i = 1}^n (X_i - \bar{X}_n)^2$. First, we construct the identity
			\begin{align*}
				\sum_{i = 1}^n (X_i - \mu)^2 & = \sum_{i = 1}^n ((X_i - \bar{X}_n) + (\bar{X}_n - \mu))^2 \\
				& = \sum_{i = 1}^n (X_i - \bar{X}_n)^2 + 2 (\bar{X}_n - \mu) \sum_{i = 1}^n (X_i - \bar{X}_n) + \sum_{i = 1}^n (\bar{X}_n - \mu)^2 \\
				& = \sum_{i = 1}^n (X_i - \bar{X}_n)^2 + n (\bar{X}_n - \mu)^2
			\end{align*}
			(Here, we used that that $\sum_{i = 1}^n (X_i - \bar{X}_n) = \left(\sum_{i = 1}^n X_i\right) - n \bar{X}_n = n \bar{X}_n - n \bar{X}_n = 0$.) Rewriting this identity yields
			\begin{equation*}
				\sum_{i = 1}^n (X_i - \bar{X}_n)^2 = \sum_{i = 1}^n (X_i - \mu)^2 - n (\bar{X}_n - \mu)^2.
			\end{equation*}
			Note that $E(\sum_{i = 1}^n (X_i - \mu)^2) = n \sigma^2$ and $E(n (\bar{X}_n - \mu)^2) = n V(\bar{X}_n) = n \cdot \frac{\sigma}{n} = \sigma$. Hence,
			\begin{align*}
				E(S_n^2) & = E\left(\frac{1}{n - 1} \sum_{i = 1}^n (X_i - \bar{X}_n)^2\right) \\
				& = \frac{1}{n + 1} \left(E\left(\sum_{i = 1}^n (X_i - \mu)^2\right) - E\left(n\left(\bar{X}_n - \mu\right)^2\right)\right) = \frac{1}{n - 1} (n \sigma^2 - \sigma^2) = \sigma^2.
			\end{align*}
		
\end{Solution}
\begin{Solution}{1.14}
		By the result of Ex \ref{ex:chap06:04}, we have $\sum_{i = 1}^n (X_i - \bar{X}_n)^2 = \sum_{i = 1}^n (X_i - \mu)^2 - n (\bar{X}_n - \mu)^2$. As such, $S_n^2 = \frac{n}{n - 1}\frac{1}{n} \sum_{i = 1}^n (X_i - \mu)^2 - \frac{n}{n - 1} (\bar{X}_n - \mu)^2$. The latter term converges to $0$ as $\bar{X}_n$ converges to $\mu$. Hence, as $n \to \infty$, the term $\frac{1}{n} \sum_{i = 1}^n (X_i - \mu)^2$ which is the sample average of i.i.d. $Z_i=(X_i - \mu)^2$ converges to the expectation of $Z_i$, which in turn is the variance of $X_i$.
		
\end{Solution}
\begin{Solution}{1.15}
		If you were to throw a fair coin a large number of times, you would expect the proportion of heads to converge to 0.5 (see Ex \ref{ex:chap06:05}). So to verify whether the coin is fair, you could throw it a large number of time and assess whether the sample proportion of heads approximates 0.5. To illustrate, run the following code:
\begin{minted}{python}
import numpy as np
import matplotlib.pyplot as plt

nSeq = 5
nTrials = 10 ** 3
p = 0.5

for j in range(nSeq):
    x = np.zeros(nTrials + 1, float)
    Mean_list = []
    for i in range(nTrials):
        x[i] = np.random.binomial(1, p)
        xbar = np.mean(x[:i+1])
        Mean_list.append(xbar)

    plt.plot(range(nTrials), Mean_list, label='Sample_' + str(j + 1))
plt.ylabel('Estimated proportion of heads')
plt.xlabel('Trials')
plt.legend(loc=0, ncol=3, fontsize='small')
plt.show()
\end{minted}
		Indeed, after throwing a fair coin 1000 times, the sample proportion of heads is close to 0.5. (Check yourself what happens if $p \neq 0.5$!)
		
\end{Solution}
\begin{Solution}{1.16}
			For $X \sim N(\mu, \sigma^2)$, the MGF of $X$ is given by $M_X(t) = e^{\mu t} e^{\frac{1}{2} \sigma^2 t^2}$. To verify this, first write $X = \mu + \sigma Z$ for $Z \sim N(0,1)$, and calculate
			\begin{align*}
				M_Z(t) & = E(e^{tZ}) = \int_{-\infty}^{\infty} e^{tz} \cdot \frac{1}{\sqrt{2 \pi}} e^{-\frac{1}{2} z^2} \mathrm{d}z = \int_{-\infty}^{\infty} e^{tz} \cdot \frac{1}{\sqrt{2 \pi}} e^{-\frac{1}{2} z^2} \mathrm{d}z \\
				& = e^{\frac{1}{2} t^2} \int_{-\infty}^{\infty} \frac{1}{\sqrt{2 \pi}} e^{-\frac{1}{2} (z - t)^2} \mathrm{d}z = e^{\frac{1}{2} t^2}
			\end{align*}
			(The last step follows from recognizing $\frac{1}{\sqrt{2 \pi}} e^{-\frac{1}{2} (z - t)^2}$ as a PDF.) It then follows that
			\begin{align*}
				M_X(t) & = E\left(e^{tX}\right) = E\left(e^{t(\mu + \sigma Z)}\right) = e^{\mu t} E\left(e^{t \sigma Z}\right) = e^{\mu t} M_Z(\sigma t) = e^{\mu t} e^{\frac{1}{2} \sigma^2 t^2}.
			\end{align*}
			Now let $X_1\sim N(\mu_1, \sigma_1^2)$ and $X_2\sim N(\mu_2, \sigma_2^2)$. We have
			\begin{align*}
				M_{X_1 + X_2}(t) & = E\left(e^{t(X_1 + X_2)}\right) = E\left(e^{t X_1}\right) E\left(e^{t X_2}\right) = e^{\mu_1 t} e^{\frac{1}{2} \sigma_1^2 t^2} e^{\mu_2 t} e^{\frac{1}{2} \sigma_2^2 t^2} = e^{(\mu_1 + \mu_2) t} e^{\frac{1}{2} (\sigma_1^2 + \sigma_2^2) t^2}
			\end{align*}
			This is the MGF of the $N(\mu_1 + \mu_2, \sigma_1^2 + \sigma_2^2)$ distribution. It follows that $X_1 + X_2 \sim N(\mu_1 + \mu_2, \sigma_1^2 + \sigma_2^2)$.
		
\end{Solution}
\begin{Solution}{1.17}
			Let $X \sim \text{Expo}(\lambda)$ for some $\lambda > 0$. Let $Y = \lambda X$ for some $\lambda > 0$. The MGF of $X$ is given by
			\begin{align*}
				M_X(t) = E(e^{tX}) = \int_0^{\infty} e^{tx} \lambda e^{-\lambda x} \mathrm{d}x = \int_0^{\infty} \lambda e^{-x(\lambda - t)} \mathrm{d}x = \left[- \frac{\lambda}{\lambda - t} e^{-x(\lambda - t)}\right]_0^{\infty} = \frac{\lambda}{\lambda - t}, \quad t < \lambda.
			\end{align*}
			It follows that the MGF of $Y$ is given by
			\begin{align*}
				M_Y(t) = E(e^{tY}) = E(e^{\lambda t X}) = M_X(\lambda t) =  \frac{\lambda}{\lambda - \lambda t} = \frac{1}{1 - t}, \quad t < 1.
			\end{align*}
			This is the MGF of the $\text{Expo}(1)$ distribution. Hence, $Y \sim \text{Expo}(1)$.
		
\end{Solution}
\begin{Solution}{1.18}
			Let $X$ have the probability distribution $f(x) = e \left(\frac{1}{3}\right)^x$ for $x = 1, 2, \ldots$. Using LOTUS, the MGF is given by
		\begin{align*}
			M_X(t) = E(e^{tX}) = \sum_{x=1}^{\infty} e^{tx} f(x) = \sum_{x=1}^{\infty} e^{tx} \cdot 2 \left(\frac{1}{3}\right)^x = \sum_{x=0}^{\infty} e^{t(x+1)} \cdot 2 \left(\frac{1}{3}\right)^{x+1} = \frac{2 e^t}{3} \sum_{x=0}^{\infty} \left(\frac{e^t}{3}\right)^x = \frac{2 \left(\frac{e^t}{3}\right)}{1 - \left(\frac{e^t}{3}\right)} = \frac{2e^t}{3 - e^t},
		\end{align*}
		for $|t|<1$. Taking derivatives, we obtain
		\begin{align*}
			M_X'(t) & = \frac{(3 - e^t) 2e^t - 2e^t (-e^t)}{(3 - e^t)^2} = \frac{6e^t}{(3 - e^t)^2} \\
			M_X''(t) & = \frac{(3 - e^t) \cdot 6e^t - 6e^t \cdot 2(3 - e^t)(-e^t}{(3 - e^t)^4}.
		\end{align*}
		It follows that $E(X) = M_X'(0) = \frac{6}{4} = \frac{3}{2}$ and $E(X^2) = \frac{24 - 12 \cdot 2 \cdot -1}{16} = 3$. Therefore $V(X) = E(X^2) - E(X)^2 = 3 - \left(\frac{6}{4}\right)^2 = 3 - \frac{9}{4} = \frac{3}{4}$.
		
\end{Solution}
\begin{Solution}{1.19}
			Let $X \sim N(\mu, \sigma^2)$. The MGF of $X$ is given by $M_X(t) = e^{\mu t} e^{\frac{1}{2} \sigma^2 t^2}$. As such,
			\begin{align*}
				M_X'(t) & = (\mu + \sigma^2 t) e^{\mu t} e^{\frac{1}{2} \sigma^2 t^2} = (\mu + \sigma^2 t) M_X(t) \\
				M_X''(t) & = (\mu + \sigma^2 t)^2 M_X(t) + \sigma^2 M_X(t)
			\end{align*}
			We obtain $E(X) = M_X'(0) = \mu M_X(0) = \mu \cdot 1 = \mu$ and $E(X^2) = M_X''(0) = \mu^2 M_X(0) + \sigma^2 M_X(0) = \mu^2 + \sigma^2$. It follows that $V(X) = E(X^2) - E(X)^2 = (\mu^2 + \sigma^2) - \mu^2 = \sigma^2$.
		
\end{Solution}
\begin{Solution}{1.20}
			Let $X \sim N(\mu, \sigma^2)$. The MGF of $X$ is given by $M_X(t) = e^{\mu t} e^{\frac{1}{2} \sigma^2 t^2}$. As such,
			\begin{align*}
				M_X'(t) & = (\mu + \sigma^2 t) e^{\mu t} e^{\frac{1}{2} \sigma^2 t^2} = (\mu + \sigma^2 t) M_X(t) \\
				M_X''(t) & = (\mu + \sigma^2 t)^2 M_X(t) + \sigma^2 M_X(t) \\
				M_X^{(3)}(t) & = (\mu + \sigma^2 t)^3 M_X(t) + 3 \sigma^2 (\mu + \sigma^2 t) M_X(t) \\
				M_X^{(4)}(t) & = (\mu + \sigma^2 t)^4 M_X(t) + 3 \sigma^2 (\mu + \sigma^2 t)^2 M_X(t) + 3 \sigma^2 (\mu + \sigma^2 t)^2 M_X(t) + 3 \sigma^4 M_X(t)
			\end{align*}	
			We obtain $E(X) = M_X'(0) = \mu$, $E(X^2) = M_X''(0) = \mu^2 + \sigma^2$, $E(X^3) = 3 \mu \sigma^2 + \mu^3$ and $E(X^4) = \mu^4 + 6 \mu^2 \sigma^2 + 3 \sigma^4$.
			
			The skewness of $X$ is given by
			\begin{align*}
				\text{Skew}(X) = E\left(\frac{X - \mu}{\sigma}\right)^3 = \frac{E\left(X^3 - 3X^2 \mu + 3X \mu^2 - \mu^3\right)}{\sigma^3}.
			\end{align*}	
			Plugging in the obtained values for $E(X)$, $E(X^2)$ and $E(X^3)$ yields $\text{Skew}(X) = 0$.
			
			The kurtosis of $X$ is given by
			\begin{align*}
				\text{Kurt}(X) = E\left(\frac{X - \mu}{\sigma}\right)^4 = \frac{E\left(X^4 - 4 \mu X^3 + 6 \mu^2 X^2 - 3 \mu^4\right)}{\sigma^4}.
			\end{align*}	
			Plugging in the obtained values for $E(X)$, $E(X^2)$ and $E(X^3)$ yields $\text{Kurt}(X) = 3$.
		
\end{Solution}
\begin{Solution}{1.21}
		\begin{enumerate}
			\item   MGF of a Expo($\lambda$)-distributed $X$: $$M_X(t)=Ee^{tX}=\int_{0}^{\infty} e^{tx}e^{-\lambda x} dx=  \frac{1}{\lambda -t},$$ {which is finite for, e.g., $t\in (-\lambda/2, \lambda/2)$ (so the MGF is well-defined)}.
			\item {MGF of $Y_1+Y_2$} with $Y_i\sim i.i.d.\text{Expo}(1)$:
			\begin{align*}
				M_{Y_1+Y_2}(t) = M_{Y_1}(t)M_{Y_2}(t)=\frac{1}{(1 -t)^2}.
			\end{align*}
			{which is finite for, e.g., $t\in (-1/2, 1/2)$ (so the MGF is well-defined).}\\~\\
			\item $\frac{1}{(1 -t)^2}$ can not be written in the form of $\frac{1}{\lambda -t}$ for any $\lambda$'s. We prove by contradiction, suppose they are equal for some $\lambda$ then $\left.\frac{1}{(1 -t)^2}\right|_{t=0}=\left.\frac{1}{\lambda -t}\right|_{t=0}$ and thus $\lambda=1$, however, $\left.\frac{1}{(1 -t)^2}\right|_{t=a}\neq \left.\frac{1}{1 -t}\right|_{t=a}$ for any $a\neq 0$.
			\item {The MGF of  $Y_1+Y_2$ is not the MGF of a exponential distribution}, then we know  a sum of two i.i.d. Expo($1$)-distributed r.v.'s is not exponentially distributed as MGF determines distributions (different MGF forms, different distributions).
		\end{enumerate}
	
\end{Solution}
\begin{Solution}{1.22}
		\begin{enumerate}
				\item {MGF of a Pois($\lambda$)-distributed $X$}: $$M_X(t)=Ee^{tX} =e^{\lambda (e^t-1)},$$ {which is finite for, e.g., $t\in (-\lambda/2, \lambda/2)$ (so the MGF is well-defined)}.
			\item {MGF of $\sum_{i=1}^{n} Y_i$} with $Y_i\sim \textit{independent  Pois}(\lambda_i)$:
			\begin{align*}
				M_{\sum_{i=1}^{n}Y_i}(t) =\prod_{i=1}^{n} e^{\lambda_i (e^t-1) }=e^{\sum_{i=1}^{n} \lambda_i (e^t-1) }.
			\end{align*}
			{which is finite for, e.g., $t\in (-1/2, 1/2)$ (so the MGF is well-defined) and is the MGF of a Pois($\sum_{i=1}^{n} \lambda_i$)-distributed $X$.} \\~\\
			\item The MGF of   $\sum_{i=1}^{n} Y_i$ is  the MGF of Pois($\sum_{i=1}^{n} \lambda_i$), then we know a  sum of $n$ independent Pois($\lambda_i$), $i\leq n$ is still Poisson.
		\end{enumerate}
	
\end{Solution}
\begin{Solution}{1.23}
		\begin{enumerate}
					\item MGF of $X \sim N(\mu,\sigma^2)$.
			\begin{align*}
				{ M_X(t)}& =  {\color{red}e^{\mu t + \frac{1}{2}\sigma^2 t^2}},
			\end{align*}  {which is finite for $t\in \mathbb{R}$, so the MGF,  $M_X$, is well defined.}
			\item MGF of $\sum_{i=1}^{n} X_i$ with $X_i\sim \textit{independent }N(\mu_i,\sigma^2_i)$, $i\leq n$:
			\begin{align*}
				M_{\sum_{i=1}^{n}X_i}(t) =\prod_{i=1}^{n} e^{\mu_i t + \frac{1}{2}\sigma_i^2 t^2}=e^{\left(\sum_{i=1}^{n} \mu_i\right) t + \frac{1}{2}\left(\sum_{i=1}^{n}\sigma_i^2\right) t^2 }.
			\end{align*}
			which is finite for, e.g., $t\in (-1, 1)$ (so the MGF is well-defined) and is the MGF of a $N\left(\sum_{i=1}^{n}\mu_i, \sum_{i=1}^{n}\sigma_i^2 \right)$-distributed r.v. \\~\\
			\item The MGF of   $\sum_{i=1}^{n} X_i$ is  the MGF of $N\left(\sum_{i=1}^{n}\mu_i, \frac{1}{2}\left(\sum_{i=1}^{n}\sigma_i^2\right)\right)$, then we know a  sum of n independent $N(\mu_i,\sigma^2_i)$, $i\leq n$ is still normal.
		\end{enumerate}
	
\end{Solution}
\begin{Solution}{1.24}
		\begin{enumerate}
			\item $N(\mu, \frac{1}{n})$, when $n\rightarrow \infty$, $\bar{X}_n$ goes closer and closer to a constant value $\mu$.
			\item $N(0,1)$.
			\item Use the fact that  $Y=\sqrt{n}(\bar{X}_n-\mu)\sim N(0,1)$ for a given $\mu$. For example, in the case $n=10000$, $\mu=2$, we would need to calculate $P(Y\geq \sqrt{n}(2.01-\mu))=P(Y\geq 100)=\Phi(100)$, which is very small.
		\end{enumerate}
	
\end{Solution}
\begin{Solution}{1.26}
		\begin{enumerate}
			\item
			\begin{enumerate}
			\item {MGF of a Pois(1)-distributed $X$}: $$M_X(t)=Ee^{tX} =e^{(e^t-1)},$$ {which is finite for, e.g., $t\in (-1/2, 1/2)$ (so the MGF is well-defined)}.\\
			\item MGF of $Y=\frac{1}{\sqrt{n}}\sum_{i=1}^n (X_i-1),$
			\begin{align*}
				M_{Y}(t)= \prod_{i=1}^{n}M_{(X_i-1)}(t/\sqrt{n}) =(e^{-t/\sqrt{n}} M_{X_1}(t/\sqrt{n}))^n = {\color{red}e^{n\left(e^{\frac{t}{\sqrt{n}}}-1-\frac{t}{\sqrt{n}}\right)}},
			\end{align*} {which is finite for $t\in \mathbb{R}$, so the MGF,  $M_Y$, is well defined.}
			\item {MGF $M_\xi(t)$ for $\xi \sim N(\mu,\sigma^2)$.}
			\begin{align*}
				{ M_\xi(t)}& =  {\color{red}e^{\mu t + \frac{1}{2}\sigma^2 t^2}},
			\end{align*}  {which is finite for $t\in \mathbb{R}$, so the MGF,  $M_\xi$, is well defined.}
		\end{enumerate}
		\item Note that $\lim\limits_{n\rightarrow \infty }{\color{red}e^{n\left(e^{\frac{t}{\sqrt{n}}}-1-\frac{t}{\sqrt{n}}\right)}}= \lim\limits_{n\rightarrow \infty}{e^{n\left(\sum_{i=0}^{\infty}\left(\frac{t}{\sqrt{n}} \right)^i/i!   -1-\frac{t}{\sqrt{n}}\right)}}=  {\color{red}e^{\frac{1}{2}t^2}}$.
		\item{The MGFs of $\frac{1}{\sqrt{n}}\sum_{i=1}^n (X_i-1)$ and $N(0,1)$ are the same in the limit}, which implies in the limit these two distributions coincide.
	\end{enumerate}~\\
	
\end{Solution}
\begin{Solution}{1.28}
~\\
		\begin{enumerate}		
			\item \begin{enumerate}
				\item $X_i\sim \text{i.i.d. Bern}(p)$, $$M_{X_i}(t)=p(e^t-1)+1, t\in \mathbb{R};$$
				\item $Y=\sum_{i=1}^{n}X_i\sim \text{Bin}(n,p),$
				$$M_{Y}(t)= \prod_{i=1}^{n}M_{X_i}(t) =(M_{X_1}(t))^n = {\color{red}\left(1+p(e^t-1)\right)^n}, t\in \mathbb{R}.$$
				\item[] $Z\sim\text{Pois($\lambda$)}$, $$M_{Z}(t)=\sum_{n=0}^\infty e^{tn}e^{-\lambda} \lambda^n/n!=e^{-\lambda} \sum_{n=0}^\infty  (\lambda e^t)^n/n! =e^{-\lambda}e^{\lambda e^t}    = {\color{red}e^{\lambda (e^t-1) }}, t\in \mathbb{R}.$$
			\end{enumerate}
			\item Note that $\lim\limits_{n\rightarrow \infty, p=\lambda/n}{\color{red}\left(1+p(e^t-1)\right)^n}= \lim\limits_{n\rightarrow \infty}\left(1+ \frac{\lambda(e^t-1)}{n}\right)^n=  {\color{red}e^{\lambda (e^t-1) }}$.
			\item{ The MGFs are the same in the limit}, which implies in the limit these two distributions coincide.
		\end{enumerate}~\\
	
\end{Solution}
\begin{Solution}{1.29}
  	\begin{enumerate}
  		\item 		 The probability of having strictly more than one winner is zero (\textbf{use words to motivate is also okay, here I only show solutions motivated by formulas} Methods are not unique, using indicator functions is also okay.):
  		\begin{align*}
  			\mathbb{P}\left(\left\{ \bigcup_{i=1}^n \left\{ \max_{l\neq i} \{X_l\} =\max_{1\leq j \leq n} \{X_j\} \right\}    \right\}^c\right) \leq \mathbb{P}\left( \bigcup_{i\neq j} \{ X_i =X_j \} \right) \leq \sum_{i\neq j} \mathbb{P}\left( X_i =X_j \right)=0
  		\end{align*}	
  		where the last equation is due to the fact that $X_i-X_j$ is one continuous r.v. and thus $\mathbb{P}\left( \{ X_i =X_j \} \right) = \mathbb{P}\left( \{ X_i-X_j =0\} \right)=0$ (the probability of one continuous r.v. equal to one fixed constant is zero).
  		\item			By symmetry of continuous random variables we know
  		\begin{align*}
  			\mathbb{P}\left(X_{a_1}<X_{a_2}<\cdots < X_{a_j} \right) =1/j!
  		\end{align*}	
  		for arbitrary permutation $(a_1,\cdots, a_j)$ of $(1,\cdots, j)$. Now among all permutations, there are $(j-1)!$ permutations of the format  $(a_1,\cdots, a_{j-1}, j)$: therefore,
  		\begin{align*}
  			\mathbb{P}\left(\bigcup_{(a_1,\cdots, a_{j-1})}\left\{X_{a_1}<X_{a_2}<\cdots < X_{a_{j-1}}<X_j\right\} \right) =1/j
  		\end{align*}
  	\item 			Note that  $X_1+X_2$ follows the same distribution as $1-X_{n-1}+1-X_n$ (they have the same PDF). Therefore, by the LOTUS,
  	\begin{align*}
  		&\mathbb{E} \left(\log\left(\frac{X_1+X_2}{2-(X_{n-1}+X_{n})} \right) \right)=\mathbb{E}\left(\log\left({X_1+X_2} \right) \right)- \mathbb{E}\left(\log\left( {2-(X_{n-1}+X_{n})} \right) \right) \\&= \int_{-\infty}^{+\infty} \log(x)g(x)dx- \int_{-\infty}^{+\infty} \log(x)g(x)dx =0
  	\end{align*}
  \item 			We first derive the MGF  of $X_1$ (for $t\neq 0$):
  \begin{align*}
  	\mathbb{E}[e^{tX_1}] =\int_{0}^{1} e^{tx} dx =  (e^t-1)/t
  \end{align*}	
  for $t=0$, we know $	\mathbb{E}[e^{tX_1}]=	\mathbb{E}[1]=1$	
  Therefore the MGF is well defined. Next, for $t\neq 0$
  \begin{align*}
  	M_{a+b X_1}(t) = e^{at}M_{X_1}(bt)= e^{at}\left(e^{bt}-1\right)/t
  \end{align*}	
  for $t=0$, $	M_{a+b X_1}(t) = 1$.
  \item 						The variance (by independence) is $Var(X^*)=\sum_{i=1}^nVar(X_i)/n = 1/12$.\\~\\
  For $t\neq 0$,
  \begin{align*}
  	&M_{X^*}(t) = M_{\left(\sum_{i=1}^n X_i\right) -\sqrt{n}/2  }(t/\sqrt{n}) =e^{-\sqrt{n}t/2 }\prod_{i=1}^nM_{ X_i}(t/\sqrt{n})\\
  	=& e^{-\sqrt{n}t/2 } \left(\sqrt{n}\left(e^{t/\sqrt{n}} -1 \right)/t\right)^n= \left(e^{-t/(2\sqrt{n}) }\sqrt{n}\left(e^{t/\sqrt{n}} -1 \right)/t\right)^n\\
  	=&  \left(\sqrt{n}\left(e^{t/(2\sqrt{n})} -e^{-t/(2\sqrt{n}) } \right)/t\right)^n =_{\textit{large n}} \left((t+t^3/24/n)/t\right)^n\\ \rightarrow_{n\rightarrow \infty}& e^{t^2/24}
  \end{align*}	
  When $t=0$, $M_{X^*}(t) =1$. 	
  The limiting distribution should be $N(0,1/12)$ by the format of our derived limiting MGF, since the normal distribution has the MGF $e^{\mu t + \sigma^2t^2/2 }$.	
  	\end{enumerate}
  ~\\~\\~\\
  Some simulation codes to help you with this exercise, also you may need some coding skills for the PD course.
  For continuouse random varaible, we can redraw many of its realised values, and then draw a histogram. Histogram can be regarded as a sample-version of the PDF (actually, it is indeed one estimator for the PDF). You can see the more data you use to draw the histogram, the histogram is smoother and closer to its PDF.

  We may use the following codes to draw $S=2,10,100,5000$ realised values from Unif(0,1) and then draw histogram:
  \begin{minted}[]{R}
set.seed(12)
n=1
sigma2=1/12
par(mfrow=c(2,2))
require(ggplot2)
loop.vector <- c(2,10,100,5000)
plot_list = list()
j=0
for(i in loop.vector){
	S=i
	j=j+1
	Unif_draws= runif(n*S, min = 0, max = 1)
	
	plot <- ggplot(data.frame(Unif_draws), aes(Unif_draws)) +
	geom_histogram(aes(y=..density..)) +
	ggtitle("n=",n)
	plot_list[[j]] = plot
}
#library(gridExtra)
require(gridExtra)
grid.arrange(plot_list[[1]], plot_list[[2]],plot_list[[3]],plot_list[[4]], nrow = 2)
  \end{minted}
 The output is the following figure~\\
 \begin{figure}[htbp!]
 	\includegraphics[width=0.7\textwidth]{0}
 \end{figure}  		   ~\\
 and indeed the larger the $S$ the closer the histogram to the PDF of Unif(0,1).\\~\\~\\

 \begin{minipage}{0.45\textwidth}
 	\includegraphics[width=0.9\textwidth]{1n2} \captionof{figure}{n=\textbf{2},S=1000}
 	\includegraphics[width=0.9\textwidth]{1n20} \captionof{figure}{n=\textbf{20},S=1000}
 \end{minipage}
 \begin{minipage}[b]{0.45\textwidth}
 	From the comparison between the histograms of $X^*_n$ (blue) and normal density (red), we can see it indeed gets closer and closer as $n$ increases. \\~\\
 	Here we fix a large $S$ to make sure the histograms are good approximations for the density of $X_n^*$.
 \end{minipage}
\newline
Each line of code can be executed using $\text{ctrl}+\text{Enter}$ in Rstudio.
~\\~\\
This draws 4 values from Unif(0,1)
\begin{minted}[]{R}
runif(4, min = 0, max = 1)
\end{minted}
Here are the codes that you may find how many are the largest number
\begin{minted}[]{R}
rand.a=runif(4, min = 0, max = 1)
sum(rand.a==max(rand.a))
\end{minted}
We may repeat this game many many times and check how many more than 1 winner in S repeated games via the following codes:
\begin{minted}[]{R}
n=4;
S=100;
rand.Srepeats=matrix(runif(n*S, min = 0, max = 1),ncol=S)
colMax<- apply(rand.Srepeats, 2, max)
# this gives you the a True and false matrix, and only true if the max in that column
result=colMax==t(matrix(rep((colMax),n),nrow=S))
# how many more than 1 winner in S repeated games
sum(colSums(result)>1)
\end{minted}

Here we comparing the histogram of X*
(we generate S values drawn from the same distribution as the one of X* and then draw histogram)
with the density function of a normal distribution.
\begin{minted}[]{R}
set.seed(12)
require(ggplot2)
n=4
S=100
sigma2=1/12
# we generate S realised values of X* (simu_Sbar) by drawing from Unif distribution
simu_Sbar= colSums(matrix(runif(n*S, min = 0, max = 1)-1/2,ncol=S))/sqrt(n)

ggplot(data.frame(simu_Sbar), aes(simu_Sbar)) +
geom_histogram(aes(y=..density..)) +
stat_function(fun=function(x)1/sqrt(2*pi*sigma2)*exp(-(x)^2/sigma2/2),
color=rgb(0.6, 0.2, 0.2, 0.35), size=2)
\end{minted}

Here are also Python codes
\begin{minted}{python}
import numpy as np
np.random.seed(10)

# now again this draws 4 values from Unif(0,1)
np.random.uniform(0,1,4)

n=4;
S=100;
rand_Srepeats= np.random.uniform(0,1,n*S).reshape(-1,S)
# how many more than 1 winner in S repeated games
np.sum(np.sum(rand_Srepeats==rand_Srepeats.max(axis=0), axis=0)>1)


# X* distribution analysis
n=4
S=100
sigma2=1/12

rand_Srepeats= np.random.uniform(0,1,n*S).reshape(-1,S) -1/2
simu_Sbar= np.sum(rand_Srepeats.reshape(-1,S),axis=0)/np.sqrt(n)
#  Histogram
import seaborn as sns
sns.distplot(simu_Sbar, hist=True, kde=True,
bins=int(180/5), color = 'darkblue',
hist_kws={'edgecolor':'black'},
kde_kws={'linewidth': 4})

# we add some normal density
import math
ax=sns.distplot(simu_Sbar, hist=True, kde=True,
bins=int(180/5), color = 'darkblue',
hist_kws={'edgecolor':'black'},
kde_kws={'linewidth': 4})
# calculate the pdf
sigma2=1/12
x0, x1 = ax.get_xlim()  # extract the endpoints for the x-axis
x_pdf = np.linspace(x0, x1, 100)

y_pdf = 1/(np.sqrt(2*math.pi*sigma2))*np.exp(-np.power(x_pdf,2)/(2*sigma2))

ax.plot(x_pdf, y_pdf, 'r', lw=2, label='pdf')
ax.legend()
\end{minted}


\end{Solution}

% }

\end{document}


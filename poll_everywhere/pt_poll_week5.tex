\documentclass[poll_tutorial_format]{subfiles}
\begin{document}
	\maketitle
	\section{PT Week 5 (Continuous) random variables}
	
	\subsection{Set things up}
	\label{sec:set-things-up}
	
	
	
	\setcounter{theorem}{-1}
	\begin{exercise}
		Have you helped your neighbors to set up their polleverywhere app? 
		\begin{enumerate}
			\item Yes
			\item No
		\end{enumerate}
	\end{exercise}
	
	\subsection{Real questions}
	\label{sec:start-real-questions pt week 4}
	\begin{exercise}
		Suppose we have a sample space $S$ with a subset/event $A$. Denote $X$ the random variable such that $X(s)=1$ for $s\in A$ otherwise $0$ 
		Which one of these statements could be incorrect:%TODO $P(X=0|A)=1$ should be 0. 
		\begin{enumerate}
			\item $EX=P(X=1)$.
			\item $EX=P(A)$.
			\item The support of $X$ is $\{0,1\}$.
			\item $P(X=0|A)=1$. 
		\end{enumerate}
	\end{exercise}
	
	
	\begin{exercise}
		(Coin tossing problem) A fair coin is flipped two times, consider the sample space $S=\{HH, HT, TH, TT\}$ (H stands for head, and T stands for tail). Denote r.v. $X$ the number of heads within two tosses.
		Choose one of these answers that is incorrect: %TODO ans:$EX^2=(\sum_{i=0}^2 iP(X=i))^2$ (slightly tricky as $EX^2=\sum_{i=0}^9 iP(X^2=i)$ include some terms that are simply zero).
		\begin{enumerate}
			\item $EX=\sum_{i=0}^2 iP(X=i)$.
			\item $EX^2=\left(\sum_{i=0}^2 iP(X=i) \right)^2$
			\item $(EX)^2=\left(\sum_{i=0}^2 iP(X=i)\right)^2$
			\item $EX^2=\sum_{i=0}^2 i^2P(X=i)$
			\item $EX^2=\sum_{i=0}^9 iP(X^2=i)$
			\item $EX^2=\sum_{i=0,1,4,9} iP(X^2=i)$
		\end{enumerate}
	\end{exercise}
	
	
	
	\begin{exercise}
		(Coin tossing problem) A fair coin is flipped two times, event A represents two tosses landed head and B represents the event that the first toss landed tail. Denote r.v. $X$ the number of heads within two tosses.
		Choose one of these answers that is incorrect: %TODO ans:$ \{X=1\}=B$
		\begin{enumerate}
			\item $\{X=2\}=A$
			\item $ \{X=1\}=B$
			\item $\{X^2 =4\} =\{X=2\} $
			\item $ \{X=0\} \subseteq B$
		\end{enumerate}
	\end{exercise}
	
	
	
	\begin{exercise}
		Choose one of these answers that is incorrect:%TODO ans: For a discrete random variable, its support may contain infinite numbers, but at most countably infinite. 
		\begin{enumerate}
			\item For a discrete random variable, its support may contain infinite numbers, but at most countably infinite.
			\item The probability mass function
			(PMF) of a discrete r.v., X, is a function that maps events of type $\{X=a\}$  to real numbers in $[0,1]$.
			\item The support of $X$ is $a_1, a_2, a_3, \dots$, it is possible that $P(X=a_{3})=0$.
		\end{enumerate}
	\end{exercise}
	
	
	\begin{exercise}
		Choose one of these answers that is incorrect:%TODO ans: For a discrete random variable, its support may contain infinite numbers, but at most countably infinite. 
		\begin{enumerate}
			\item A random variable is a function that maps each outcome in the sample space to a number.  
			\item CDF functions are right-continuous functions increasing from 0 to 1.
			\item The CDF function of r.v. $X$ is a functions that map events of type $\{X\leq a\}$  for  to real numbers in $[0,1]$.
			\item When we talk about the distribution of a random variable, we are solely referring to its PMF function.
		\end{enumerate}
	\end{exercise}
	
	
	
	\begin{exercise}
		Denote $X$ a discrete random variable following the Discrete Uniform distribution with support $\{-2,-1,1,2\}$.
		Choose one of these answers that is incorrect: %TODO ans: The support of $f(X)$ with $f(x)=x^2$ is $\{-1,-2,1,2\}$  
		\begin{enumerate}
			\item The PMF function $P(X=a)=1/4$ for $a\in \{-2,-1,1,2\}$ otherwise 0 describes the how the probability is distributed among events generated by $X$. 
			\item The support of $X^2$ is $\{1,4\}$
			\item The support of $f(X)$ with $f(x)=x^2$ is $\{-1,-2,1,2\}$  
			\item The CDF function of $X$ describes how the probability is distributed among events generated by $X$.  
			\item The first and the forth choices.
		\end{enumerate}
	\end{exercise}
	
	
	\begin{exercise}
		Let $X_1,\cdots , X_n$ follow independent Bernoulli distribution with the same successful rate $1/2$, 
		Choose one of these answers that is incorrect: %TODO ans:The support of $\sum_{i=1}^n X_n$ is $\{1,2,3,4,...,n\}$.
		\begin{enumerate}
			\item The PMF of $X_i$ is $P(X_i=1)=1/2$ and $P(X_i=0)=1/2$ otherwise 0.
			\item The CDF of $X_i$ is $F(x)=0$ for $x<0$; $F(x)=1/2$ for $0\leq x <1$; and $F(x)=1$ for $x\geq 1$.  
			\item $\sum_{i=1}^n X_n$ follows  Bin(n,1/2).
			\item The support of $\sum_{i=1}^n X_n$ is $\{1,2,3,4,...,n\}$.
		\end{enumerate}
	\end{exercise}
	
	
	
	\begin{exercise}
		(Coin tossing problem) A fair coin is flipped two times, event A represents two tosses landed head and B represents the event that the first toss landed tail. 
		Choose one of these answers that is incorrect: %TODO ans: $P(A|B^c)=P(B^c|A)$
		\begin{enumerate}
			\item $P(B\cup B^c )=P(B)+P(B^c)$
			\item $P(B\cup B^c |A)=P(B|A)+P(B^c|A)$
			\item $P(A|B^c)=P(B^c|A)$
			\item $P(A|B)=0$
			\item $P(A|B^c)=P(B)$
		\end{enumerate}
	\end{exercise}
	
	
	\begin{exercise}
		Here are some statements regarding the conditional probability (where $P(A)\neq 0$, and $S$ is the sample space).
		Choose one of these answers that may be incorrect: %TODO ans: $P(B|A)+P(B|A^c)=1$ 
		\begin{enumerate}
			\item $P(S|A)=1$
			\item $P(\cup_{i=1}^\infty B_i|A) = \sum_{i=1}^\infty P(B_i|A)$ for disjoint events $B_i$'s
			\item $P(B|A)+P(B^c|A)=1$
			\item $P(B|A)+P(B|A^c)=1$ 
			\item Suppose that $A$ and $B$ are independent then $P(B|A)=P(B)$ and $P(B|A)P(A)=P(B)P(A)$ 
		\end{enumerate}
	\end{exercise}
	
	
	\begin{exercise}
		(Coin tossing problem) A fair coin is flipped two times, event A represents two tosses landed head, B represents the event that the first toss landed tail and C represents the event that the second toss landed tail. 
		Choose one of these answers that is incorrect: %TODO ans: (indepednence not necessarily conditional independence)	$P(B^c\cap C^c|A)=P(B^c|A)P(C^c|A) $
		\begin{enumerate}
			\item $P(A)=P(A|B)P(B)+P(A|B^c)P(B^c)$ % LOTP
			\item $P(A|B^c)=1/2$
			\item $P(B^c\cap C^c)=P(B^c)P(C^c)$ % Independence
			\item $P(B^c\cap C^c|A)=P(B^c|A)P(C^c|A) $  
		\end{enumerate}
	\end{exercise}
	
	
	
	\begin{exercise}
		Which one of these answers is not part of the 
		Choose one of these answers: %TODO ans: $2^{10}$
		\begin{enumerate}
			\item $10^2$
			\item $2^{10}$
			\item ${10\choose 2}$
			\item $2! * 10$
			\item $10!/(10-2)!$
		\end{enumerate}
	\end{exercise}
	
	
	
	
	
	
\end{document}

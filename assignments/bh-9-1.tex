\subsection{BH.9.1}
\label{sec:bh.9.1}

Here is the code.

\begin{minted}[]{python}
import numpy as np
from scipy.stats import randint, norm

mus = np.array([2, 5, 8])
stds = np.array([0.1, 0.5, 1])


m2 = stds ** 2 + mus ** 2 # this
v = m2.mean() - (mus.mean()) ** 2 # this
print(mus.mean(), v)

routes = []
for i in range(3):
    routes.append(norm(mus[i], stds[i])) # this


num = 200
options = randint(0, 3).rvs(num) # this

times = np.zeros(num)  # this
for i in range(num):
    times[i] = routes[options[i]].rvs() # this

print(times.mean(), times.var())
\end{minted}

\begin{minted}[]{R}
mus = c(2, 5, 8)
stds = c(0.1, 0.5, 1)

m2 = stds ** 2 + mus ** 2 #this
v = mean(m2) - (mean(mus)) ** 2 #this
print(paste(mean(mus), v))

num = 200
options = sample(1:3, num, replace = TRUE) #this

times = rep(0, num)
for(i in 1:num){
  times[i] = rnorm(1, mus[options[i]],stds[options[i]]) # this
}

print(paste(mean(times),var(times)))
\end{minted}

\begin{exercise}
Explain the lines marked as `this', i.e., how does the program work?
\end{exercise}

\begin{exercise}
Run the code and explain the results. Are the simulated results in line with the theory?
\end{exercise}

\begin{exercise}
Modify the code so that you can plot the histogram of the \verb|times| vector. Include your code and a figure. Set also a seed equal to, e.g.,  the day of the month in which you do the assignment.
\end{exercise}


\begin{exercise}
Set $\mu_8$ to 80 instead of $8$. Rerun the code and explain what happens to the mean and the variance. Then change to 800 and include a histogram of the PDF of the times.
\end{exercise}


\begin{exercise}
Change $\sigma_1$ from $0.1$ to $10$.  Run the code and describe what you get. What's wrong with such a large std?
\begin{solution}
The route lengths can become negative.
\end{solution}
\end{exercise}

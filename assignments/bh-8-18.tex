\subsection{BH.8.18: Division of rvs}


% Use Python uncertainties module.\footnote{\url{https://pythonhosted.org/uncertainties/}}.


% We can demonstrate the use of Taylor's equation to see how uncertaintly should propagate\footnote{\url{https://en.wikipedia.org/wiki/Propagation_of_uncertainty}}.

While BH.8.18 uses Jacobians to deal with the fraction $X/Y$ of two rvs $X$ and $Y$, here we use simulation for this purpose
(You don't have to solve BH.8.18 as this year (2023) we skip the Jacobians.)

Here is the basic question: How many ping pong balls fit into an Airbus Beluga?


One way to answer this is as follows.
According to this \href{https://en.wikipedia.org/wiki/Airbus\_Beluga}{wiki-page} the cargo volume $V$ of this airplane is $1500 \m^{3}$.
But this number is based on the physical dimensions that is available to store containers, tanks, and so on.
So, I estimate the volume as about twice that amount, i.e., $V = 2500 \m^{3}$.
The volume of a ping pong ball is $v = 4 \pi r^3/3  = \py{4*3.14*8/3} \cm^{3}$ with $r=2$ cm.
A plain division gives \py{2500/33.5} ping pong balls.
Note, I left out the $10^{6}$ conversion from meters to cm, and I do not take into  account the sphere packing factor.
Besides that, I hope you agree with me that providing a result with the precision as given here is plain ridiculous.\footnote{For reasons incomprehensible to me, even professional econometricians sometimes  report results with 10 digits or more, without questioning the precision.}


In fact, I know that the volumes of an airplane and a ping pong ball is an estimate, rather than a precise number as assumed above.
It seems to be better to approximate $V$ and $v$ as rvs.
Let's assume that
   \begin{align*}
V \sim N(2500, 500^{2}), & v \sim N(33.5, 0.5^{2}),
\end{align*}
where the variances express my trust in my guess work.
What is now the mean of $N = V/v$ and its std? Can we get the CDF?
From BH.8.18 you know that finding the closed form expression for the distribution of $N$ is not entirely simple.
However, with simulation it's easy to get an estimates for the mean and the standard deviation.


\begin{exercise}
Explain lines marked as `this'.

\begin{minted}[]{python}
import numpy as np
from scipy.stats import norm

num = 500

np.random.seed(3)

V = norm(loc=2500, scale=500) # this
v = norm(loc=33.5, scale=0.5) # this

print(V.mean(), V.std()) # just a check

N = V.rvs(num) / v.rvs(num) # this
print(N.mean(), N.std())

print(2500/33.5)
print(np.sqrt(500*0.5))
\end{minted}

\begin{minted}[]{R}
num <- 500

set.seed(3)

V = rnorm(num, 2500, 500) # this
v = rnorm(num, 33.5, 0.5) # this

N = V / v  # this
print(paste(mean(N), sd(N)))

\end{minted}
\end{exercise}


\begin{exercise}
Contrary to BH.7.1.25 if you run the code above, you'll see that $\E N < \infty$, and is very near to the deterministic answer.
But isn't this strange: we divide two normal random variables, just like BH.7.1.25, but there the expectation is infinite.
Explain why it works for the Beluga case, but not for the case discussed in BH.7.1.25.
\end{exercise}

\begin{exercise}
Include code to make a histogram of the PDF of $N$.
\end{exercise}



\begin{remark}
The numerical results suggest the interesting guess $\V N \approx \V V * \V v$. Whether this is true more generally will be discussed in the Challenge of week 4.
\end{remark}

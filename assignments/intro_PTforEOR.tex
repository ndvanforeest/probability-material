\section*{Introduction}
\addcontentsline{toc}{section}{Introduction}

Here we just provide the exercises of the assignments. All these exercises are based on the textbook exercises with variations. 


The rules:
\begin{enumerate}
	\item For each assignment, you may work in a group of at most two people, and have to turn in a pdf document typeset in \LaTeX{}. Include a title, group number, student names and ids, and date.
	\item Exercises are modified based on exercises from \textbf{Introduction to Probability, Second Edition by J.K. Blitzstein and J. Hwang and published by CRC Press}, which is referred to as \textbf{[BH]}: \textbf{first} answer the original question \textbf{in you own words}, and \textbf{then} answer the additional questions modified based on the original exercises.\\~\\ Due to copyright issues, we could not include textbook exercises' main contexts but only a rough description of the original exercise and its index. The index is referred to the exercise section index from \textbf{[BH]}: e.g., the first exercise in Assignment 1, Ex 1.1, is the exercise 1.9.7 from  \textbf{[BH].}Chapter 1. 
	\item 
	We expect \textbf{brief} answers \textbf{in your own words}, part of the exercises' solutions (\textbf{[BH]} exercises' solutions)  are provided in our course materials as references when you find exercises challenging but do not copy directly (otherwise, no points)
	\item If you want to include some graphs, photos/scans of hand drawings are certainly okay.
	\item Default point is 1 per assignment even if you did not hand in anything, if you answer the original exercises with the solution we provide, that's at most 4. 
\end{enumerate}

For some exercises, we provide simulation codes for you to understand the uncertainty better. There are many programming languages, and they all exist for certain purposes. For applications in Econometrics,  you might have heard of Matlab, Stata, and R, which are widely used in the field for their specifically designed packages/build-in default functions. Python can be used in a more general scenario,  though less convenient for some specific purposes/cases where the other languages may be optimized for. Hence, we provide codes either in R or Python.\\~\\ 
Coding is \textbf{not} part of the exam material but could be helpful in your later courses, and the earlier you get in touch with the better.  
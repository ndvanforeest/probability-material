\subsection{BH.8.54}
\label{sec:bh.8.54}


Read and solve BH.8.54. Perhaps you should read the hints in the study guide on this problem too.
First we consider the case with $r=1$, and we need to tackle some technical details.

\begin{exercise}
Just as in the problem, let $Y=p X/ q$. Why is $\E Y = 1$?
\end{exercise}

\begin{exercise}
If $Z\sim\Exp{1}$, then explain that $M_Z(s) = 1/(1-s)$ for $s<1$.
\end{exercise}

\begin{exercise}
Use the mathematical solution of the problem to explain that $M_Y(s)\to M_Z(s)$ as $p\to 0$.
\end{exercise}


\begin{exercise}
Here is code to compare $M_Y(s)$ to $M_{Z}(s)$. Explain what the function $M(s)$ does. Then explain why in the line marked as `this' we multiply with $1-s_{i}$.  Explain why we limit $s$ to the interval $[0, 0.8]$. Finally, explain why the graph goes through the point $x=0, y = 1$.

\begin{minted}[]{python}
import numpy as np
from scipy.stats import geom, gamma
import matplotlib.pyplot as plt

np.random.seed(3)

p = 0.1
q = 1 - p
n = 10000


def M(s):
    X = geom(p).rvs(n)
    Y = p * X / q
    return np.exp(s * Y).mean()


num = 30
m = np.zeros(num)
s = np.linspace(0, 0.8, num)
for i in range(num):
    m[i] = M(s[i]) * (1 - s[i])  # this

plt.plot(s, m, label="MY")
plt.tight_layout()
plt.savefig("figures/moments.pdf")
\end{minted}


\begin{minted}[]{R}
set.seed(3)

p = 0.1
q = 1 - p
n = 100000

M = function(s){
    X = rgeom(n,p) - 1
    Y = p * X / q
    return (mean((exp(s * Y))))
}


num = 30
m = rep(0,num)
s = seq(0,0.8,length.out = num)
for (i in 1:num){
    m[i] = M(s[i]) * (1 - s[i]) # this
}

pdf(file="figures/moments.pdf",
    width = 8, height = 7, bg = "white",          
    colormodel = "cmyk", paper = "A4")   

plot(s,m, type = "l", ann = FALSE, col = "blue")

dev.off()
\end{minted}
\end{exercise}

\begin{exercise}
The match is not terrric when $p=0.1$. Rerun the code, but now with $p=1/1000$. Which line to you have to change for this? Include your graph. Is the match better now?
\end{exercise}

\begin{exercise}
The problem asks for the general $r$. What lines of code have to change to deal with general $r$? Modify the code appropriately, and explain why your code is correct.
\begin{solution}
The limiting mgf is the mgf of the gamma distribution. This is $(1/(1-s)^{r}$. Also the mgf of the geom must be raised to the power $r$.
\begin{minted}[]{python}
    m[i] = (M(s[i]) * (1 - s[i]))**r
\end{minted}

\begin{minted}[]{R}
    m[i] = (M(s[i]) * (1 - s[i]))**r
\end{minted}
\end{solution}
\end{exercise}

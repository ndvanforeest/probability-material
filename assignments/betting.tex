\subsubsection{Challenge: Betting}
\label{sec:betting}

Consider the setting of BH.9.25, which you also studied in the coding section. We use the notation from that exercise.
In this exercise we will discuss how to set $f$, the betting fraction. In particular, we will discuss the \textit{Kelly criterion}, which states that the betting fraction should be $f = 2p-1$ if $p > \tfrac12$ is the winning probability.

We discuss its relationship to expected utility theory, which you will also study in Introduction to Mathematical Economics. Expected utility theory states that bets should be chosen to maximize expected utility. So we solve $\max\limits_{0 \leq f \leq 1} \E{U(X_{n+1}) \given X_n}$.



\begin{exercise}
Show that solving the maximization problem for the utility function $U(x) = \log(x)$ yields the betting fractions from the Kelly criterion, $f = 2p-1$ if $p > \tfrac12$ and $f=0$ if $p \leq \tfrac12$.
\end{exercise}

Other people may have a different utility function, which yields a different betting fraction.

\begin{exercise}
Calculate the utility maximizing betting fraction $f$ if $U(x) = \sqrt{x}$.
\end{exercise}

Note that for both of these utility functions, the betting fraction $f$ does not depend on the wealth $X_n$ before the gamble, but in general $f$ does depend on $X_n$.

Now that we have two different betting fractions, we compare them. For that, we first need the following result:

\begin{exercise}
Assume that $f$ does not depend on $X_n$. Let $x_0 = 1$. Show that there exist constants $a, b$ such that $\log (X_n) = a W + b$ and $W  \sim \Bin{n, p}$, and determine $a$ and $b$ in terms of $f$.
\end{exercise}

Theorem 10.3.6. states that (for sufficiently large $n$) we can approximate a random variable with the binomial distribution $W  \sim \Bin{n, p}$ by a random variable with the normal distribution $\Norm{np, np(1-p)}$. While you will only learn about the proof of this next week, we are already going to use this approximation here.

\begin{exercise}
Two people (Carl and Daria) participate in $n$ rounds of this betting game. Their games are independent. Carl's initial wealth is $x_0 = 1$ and Daria's initial wealth is $y_0 = 1$. We denote Carl's wealth after $n$ rounds by $X_n$ and Daria's wealth after $n$ rounds by $Y_n$. Carl chooses $f$ according to the Kelly criterion, i.e. $f = 2p-1$.  Daria chooses $f$ to be the  utility maximizing betting fraction for  $U(x) = \sqrt{x}$.
Use the previously mentioned normal approximation to derive an approximation for the difference $\log(Y_n) - \log(X_n)$.
\end{exercise}

Kelly's criterion does not mention utility functions, it just recommends to set $f = 2p-1$ regardless of one's utility function. The next exercise is meant to give some insight why.


\begin{exercise}
Use \texttt{pnorm} in R, or \texttt{norm.cdf} in Python, to approximate $P(X_n > Y_n)$ for some chosen values for $n$ and $p$. What do you think that happens if $n \to \infty$ for a fixed $p$? Explain why this is an argument to use the Kelly criterion  regardless of one's utility function. Also, explain why maximizing utility suggests a different $f$ in spite of this result.
\end{exercise}

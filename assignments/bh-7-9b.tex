\subsection{Assignment based on exercise BH.7.9}
\label{sec:bh.7.9}

First read and solve BH.7.9. Here we provide a simulation.

\begin{exercise}
Include your hand written solution for this problem. Read the course manual  on why we want you to do it like this.
\end{exercise}

\begin{exercise}
  In Python (or R), devise a monte carlo test using the \texttt{min()} function to determine whether \texttt{scipy.stats.geom} (or the \texttt{\_geom()} functions in R) uses the ``geometric'' or ``first succes'' distribution.

  What is the probability that your devised test fails to give the correct result?
\end{exercise}

\begin{exercise}
  The naming convention of ``first succes'' is BH specific.  Expecting
  the other distribution can lead to some nasty bugs or leave you
  needlessly wondering where you messed up in your calculations.

  Provide a (simple) solution to obtain the correct/other distribution.
\end{exercise}

In Python one can raise various types of errors if such a check fails, resulting in more resilient code:\

\begin{minted}[]{python}
if ... include your test here ...:
    raise ValueError("Not a geometric distribution!")
\end{minted}

\begin{exercise}
Explain the lines marked with `this line' in the code of ~\cref{BH.7.9.p} or ~\cref{BH.7.9.r}. You already explained the other parts so you don't have to do that again.
\end{exercise}

\begin{listing}[!ht]
\begin{minted}[]{python}
import numpy as np
from scipy.stats import randint, geom
import matplotlib.pyplot as plt

np.random.seed(3)

p, num = 0.1, 100000
q = 1 - p
rv = geom(p, loc=-1)
X = rv.rvs(num)

Y = rv.rvs(num)
N = X + Y

fig, axes = plt.subplots(3, 3, sharey="all", figsize=(6, 3))  # this
n = 7  # gives nice results, after some experimentation
for ax in axes.flatten():  # this
    Xn = X[np.argwhere(N == n)]
    y, bins = np.histogram(Xn, bins=np.linspace(0, n + 1, n + 2)) # this
    ax.plot(range(n + 1), y, 'bo', ms=3, label='N=8')  # this
    ax.vlines(range(n + 1), 0, y, colors='b', lw=2, alpha=0.5) # this
    ax.set_title(f'{n=} ({len(Xn)})')
    ax.set_ylim(0) # this
    n += 1
\end{minted}
\caption{BH.7.9, Python code.}
\label{BH.7.9.p}
\end{listing}


\begin{listing}[!ht]
\begin{minted}[]{R}
library(ggplot2)
library(gridExtra)

p = 0.1
num = 100000
q = 1 - p
X = rgeom(num, p)

Y = rgeom(num, p)
N = X + Y

pdf(file="bh-7-9.pdf", width = 8, height = 7,
    bg = "white", colormodel = "cmyk", paper = "A4")

plots <-list()
par(oma= c(0,0,0,0))

i = 1
n = 7  # gives nice results, after some experimentation
while (i <= 9){  #this
  Xn = X[which(N == n)]
  df = data.frame(Xn)
  plots[[i]] =  ggplot()+
                geom_histogram(df, mapping = (aes(x = Xn)),
                bins = n + 1, color = "blue")+ # this
                ylim(0,600)+  # this
                labs(title=paste0("n=", n, ", (", length(Xn) ,")"))+
                theme(axis.title.x=element_blank(),
                axis.title.y = element_blank())
  n = n + 1
  i = i + 1
}

do.call("grid.arrange", c(plots, ncol=3))
dev.off()
\end{minted}
\caption{BH.7.9, R code.}
\label{BH.7.9.r}

\end{listing}

\begin{exercise}
Explain the results. Do they match with the theoretical results?
\end{exercise}

\begin{exercise}
  Include a handwritten solution of BH.7.29 and verify your results
  using Python or R.  You don't need to plot a 3-D graph, checking
  whether a few values are in the same ballpark is enough.
\end{exercise}

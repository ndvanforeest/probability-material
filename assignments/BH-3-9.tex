\begin{exercise}
	\textbf{[BH.9]} $F(x) = p F_1(x)+ (1 - p) F_2(x)$ for all $x$.
	\begin{enumerate}
		\item First answer the original questions in your own words.
	\end{enumerate}
Extensions from this exercise, Let r.v.'s $X_1$, $X_2$, $X$ and $Y$ follow distribution $F_1$, $F_2$, $F$ and Bern$(p)$ respectively. Suppose $X_1,X_2$ and $Y$ are independent.
\begin{enumerate}
	\item Express the CDF (one way to describe the distribution) of the new r.v. $\tilde{X}_1=\mu+ \sigma X_1$, $F_{\tilde{X}_1}$, using $F_1$;
	\item Denote $\tilde{Y}=YX_1+(1-Y)X_2$, what the connection between the distribution of $Y$ and the distribution of $\tilde{Y}$?
\end{enumerate}
	\begin{solution}~
		\begin{enumerate}
			\item The properties of a valid CDF are
			\begin{enumerate}
				\item Increasing: If $x_{1}\leq x_{2}$, then $F(x_{1})\leq F(x_{2})$.
				\item Right-continuous: $F(a)=\lim_{x\rightarrow a^{+}} F(x)$.
				\item Convergence to 0 and 1 in the limits: $\lim_{x\rightarrow-\infty}F(x)=0$ and $\lim_{x\rightarrow\infty}F(x)=0$.
			\end{enumerate}
			To show (i), notice that $F(x_{1})= pF_{1}(x_{1}) + (1-p)F_{2}(x_{1})\leq  pF_{1}(x_{2}) + (1-p)F_{2}(x_{2})=F(x_{2})$, where the inequality holds because $F_{1}$ and $F_{2}$ are CDFs \textit{and} $0<p<1$. 
			To show (ii) 
			\begin{align*}
				\lim_{x\rightarrow a^{+}} F(x) &= \lim_{x\rightarrow a^{+}}(pF_1(x) + (1-p)F_{2}(x))\\
				&= p\cdot \lim_{x\rightarrow a^{+}}F_1(x) + (1-p)\cdot \lim_{x\rightarrow a^{+}}F_{2}(x)\tag*{(Algebraic Limit Theorem)}\\
				& = p F_1(a) + (1-p)F_{2}(a)\\
				& = F(a).
			\end{align*}
			To show (iii)
			\begin{align*}
				\lim_{x\rightarrow-\infty}F(x)& = \lim_{x\rightarrow-\infty} pF_1(x) + (1-p)F_{2}(x)\\
				& =p\cdot  \lim_{x\rightarrow-\infty} F_1(x) + (1-p) \cdot  \lim_{x\rightarrow-\infty} F_{2}(x)\\
				& = 0,\\
				\lim_{x\rightarrow \infty}F(x)& = \lim_{x\rightarrow \infty} pF_1(x) + (1-p)F_{2}(x)\\
				& =p\cdot  \lim_{x\rightarrow-\infty} F_1(x) + (1-p) \cdot  \lim_{x\rightarrow-\infty} F_{2}(x)\\
				& = p + (1-p)\\
				& = 1.
			\end{align*}
			\item This uses the definition of the CDF and the LOTP. Define $X_{1}$ as the the random variable with distribution $F_{1}$ and $X_{2}$ as the random variable with distribution $F_{2}$ and $X$ as the random variable with distribution $F$. Define $Y$ as the random variable equal to 1 if the coin lands heads and equal to 0 if the coin lands tails. Then,
			\begin{align*}
				F(x)=P(X\leq x)& = P(X\leq x|Y=1)P(Y=1) +  P(X\leq x|Y=0)P(Y=0)\tag*{(\text{LOTP})}\\
				&= P(X_{1}\leq x)P(Y=1) + P(X_{2}\leq x)P(Y=0)\\
				&= p\cdot F_{1}(x) + (1-p)\cdot F_{2}(x).
			\end{align*} 
		\end{enumerate}
	Extensions from this exercise, Let r.v.'s $X_1$, $X_2$, $X$ and $Y$ follow distribution $F_1$, $F_2$, $F$ and Bern$(p)$ respectively. Suppose $X_1,X_2$ and $Y$ are independent.
	\begin{enumerate}
		\item $F_{\tilde{X}_1}(x)=F_1((x-\mu)/\sigma)$;
		\item Identical distribution.
	\end{enumerate}
	\end{solution}
\end{exercise}

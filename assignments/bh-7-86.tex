\subsection{BH.7.86}

In line with Exercise BH.7.86, we are now going to analyze the effect on \(\P{D\given T}\) when the sensitivity is not known exactly.
So, why is this interesting?
In Example BH.2.3.9 the sensitivity is given, but in fact, in `real' experiments, this is not always known as accurately as assumed in this example.
For example, in this paper: \href{https://www.thelancet.com/journals/lanres/article/PIIS2213-2600(20)30453-7/fulltext}{False-positive COVID-19 results: hidden problems and costs} it is claimed that `The current rate of operational false-positive swab tests in the UK is unknown; preliminary estimates show it could be somewhere between 0·8\% and 4·0\%.'.
Hence, even though it is claimed that PCR tests `have analytical sensitivity and specificity of greater than 95\%', it may be 4\% lower.
Simply put, the specificity and sensitivity are not precisely known, hence this must affect \(\P{D\given T}\).

%To help you, we show how to make one graph. Then we ask you to make a few on your own, and comment on them.



\begin{exercise}
Turn in your hand written solution for this problem.
\end{exercise}


I write \texttt{p\_D\_g\_T} for \(\P{D\given T}\). Here is how  Example 2.3.9 can be implemented.

\begin{minted}[]{python}
sensitivity = 0.95
specificity = 0.95
p_D = 0.01

p_T = sensitivity * p_D + (1-specificity)*(1-p_D)
p_D_g_T  = sensitivity * p_D/p_T
p_D_g_T
\end{minted}

\begin{minted}[]{R}
sensitivity = 0.95
specificity = 0.95
p_D = 0.01

p_T = sensitivity * p_D + (1-specificity)*(1-p_D)
p_D_g_T  = sensitivity * p_D/p_T
p_D_g_T
\end{minted}


\begin{exercise}
\begin{enumerate}
\item Make a plot of \(\P{D\given T}\) in which you vary the sensitivity from 0.9 to 0.99. Explain what you see.
\item Make a plot of \(\P{D\given T}\) in which you vary the specificity from 0.9 to 0.99. Explain what you see.
\item Make a plot of \(\P{D\given T}\) in which you vary \(\P{D}\) from 0.01 to 0.5. Explain what you see.
\end{enumerate}
\end{exercise}

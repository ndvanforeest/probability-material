\section*{Question}

\begin{exercise}[0.5]
Consider a random variable $X\sim\mathcal{N}(\mu,\sigma^2)$. Write down the PDF of the random variable $Y=e^X$. You do not have to elaborate on your answer, but make sure to get everything correct.
\begin{solution}
The PDF of $Y$ is given by
\begin{align*}
    &= \frac{1}{\sqrt{2\pi}\sigma y}\exp{\left(-\frac{1}{2\sigma^2}(\ln{y}-\mu)^2\right)}
\end{align*}
for $y>0$.
\\\\
Grading scheme:
\begin{itemize}
    \item Correct 0.5pt.
\end{itemize}
\end{solution}
\end{exercise}

\begin{exercise}[1.5]
Consider now the random variable $W_k = \frac{k}{5Y^2}$. What is the distribution of $W_k$? You can use results from the book here.
\begin{solution}
Notice that, since $Y$ is a normal rv, the log of $Y$ is log-normal. Then, taking the $\ln$ on both sides, we get that
\begin{align*}
    \ln{W_k} = \ln{k}-\ln{5}-2\ln{Y_k}.
\end{align*}
From the book, we know that if $\ln{Y}\sim\mathcal{N}(\mu,\sigma^2)$, then it must be that
\begin{align*}
    -2\ln{Y}\sim\mathcal{N}(-2\mu, 4\sigma^2),
\end{align*}
and that
\begin{align*}
    \ln{W_k} = \ln{k}-\ln{5}-2\ln{Y}\sim\mathcal{N}(\ln{k}-\ln{5}-2\mu, 4\sigma^2).
\end{align*}
Thus, $W_k\sim\mathcal{L}\mathcal{N}(\ln{k}-\ln{5}-2\mu, 4\sigma^2)$
\\\\
Grading scheme:
\begin{itemize}
    \item The idea to take logs 0.5pt.
    \item The rest correct 1pt.
\end{itemize}
\end{solution}
\end{exercise}


\begin{exercise}[1]
Calculate $\P{\frac{W_k}{W_l}=\frac{k}{l}}$ for some $l>k>0$. Are $W_kW_l$ and $\frac{W_k}{W_l}$ independent?
\begin{solution}
Clearly, $W_k=\frac{k}{l}W_l$, and thus we can see that
\begin{align*}
    W_kW_l &= \frac kl W_k^2,\\
    \frac{W_k}{W_l} &= \frac kl.
\end{align*}
These are independent, since one is a constant.
\\\\
Grading scheme:
\begin{itemize}
    \item Correct probability 0.5pt.
    \item Correct conclusion 0.5pt.
\end{itemize}
\end{solution}
\end{exercise}


\begin{exercise}[2]
Let $X_1,X_2\sim X$ be IID random variables, where $X$ has PDF
\begin{align*}
    f_X(x) = \frac{1}{\sqrt{2\pi x}}\exp{\left(-\frac{x}{2}\right)}
\end{align*}
for $x>0$. Find the joint PDF of the random variables $U = X_1+X_2$ and $V = X_1-X_2$.
\begin{solution}
Since $U = X_1 +X_2$ and $V = X_1-X_2$, we can write the inverse functions $X_1 = \frac12(U+V)$ and $X_2 = \frac12(U-V)$. These functions are one-to-one and $C^1$, so we can write the Jacobian matrix
\begingroup
\renewcommand*{\arraystretch}{1.5}
\begin{align*}
    J = \begin{pmatrix}
    \frac12 & \frac12\\
    \frac12 & -\frac12
    \end{pmatrix},
\end{align*}
\endgroup
which has absolute determinant $\frac12$. Since $X_1$ and $X_2$ are independent,
\begin{align*}
    f_{X_1,X_2}(x_1,x_2) = \frac{1}{2\pi\sqrt{x_1x_2}}\exp{\left(-\frac{x_1+x_2}{2}\right)}
\end{align*}
for $x_1,x_2\in\bf{R}_+$. Then, by the transformation theorem, we have that
\begin{align*}
    f_{U,V}(u,v) &= f_{{X_1,X_2}}\left(\frac12(u+v),\frac12(u-v)\right)\frac{1}{2}\\
    &= \frac{1}{2\pi \sqrt{u^2-v^2}}\exp{\left(-\frac{u}{2}\right)}\\
\end{align*}
For $-\infty<v<u<\infty$.
\\\\
Grading scheme:
\begin{itemize}
    \item Correct inverses 0.5pt.
    \item Correct absolute determinant of the Jacobian matrix 0.5pt.
    \item Correct application of transformation theorem 0.5pt.
    \item Most of: the correct bounds, no mistakes in calculation, noticing the theorem is applicable, the inverses are 1-1 and $C^1$, independence of $X_1$ and $X_2$ etc. 0.5pt.
\end{itemize}
\end{solution}
\end{exercise}
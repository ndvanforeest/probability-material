\section*{Question}

A mouse is trapped in a pit with three tunnels.
When the mouse takes tunnel A, the time to get out of the pit is 2 minutes.
Tunnel B leads back to the pit (in other words, the mouse cannot escape when it takes tunnel B) and takes 3 minutes.
Tunnel C leads also back to the pit, and takes 4 minutes.
Every time the mouse is in the pit, it selects a tunnel at random with equal probability.
(This mouse much dumber than a real mouse.)
Write $X$ for the tunnel selected by the mouse, and let $T$ be the time until the mouse escapes.
The travel times of the tunnels are constant.

For the moment, assume that $\E T < \infty$.


\begin{exercise}[1]
  Explain that $\E{T|X=B} = 3 + \E{T}$.
\begin{solution}
After selecting tunnel $B$, which takes 3 minutes to travel, the mouse is back in the pit again, and the process starts over again.
\end{solution}
\end{exercise}


\begin{exercise}[1]
  Compute $\E T$.
\begin{solution}
    \begin{align}
\E{T} &= \E{T|X=A}/3 + \E{T|X=B}/3 + \E{T|X=C}/3.\\
\E{T|X=A} &= 2\\
\E{T|X=B} &= 3 + \E T\\
\E{T|X=C} &= 4 + \E T.
    \end{align}
Solving gives $\E T=9$.

Grading
\begin{itemize}
\item Not using the result of subquestion 1: no points.
\end{itemize}
\end{solution}
\end{exercise}


\begin{exercise}[2]
Compute $\V T$.
\begin{solution}
  \begin{align}
\V{T|X=A} &= 0\\
\V{T|X=B} &= \V T\\
\V{T|X=C} &= \V T\\
\E{\V{T|X}} &= \V T 2/ 3.\\
\E{T|X} &= 2 \1{X=A} + (3 + \E T) \1{X=B} + (4 + \E T)\1{X=C} \\
 &= 2 \1{X=A} + 12 \1{X=B} + 13\1{X=C} \\
\V{\E{T|X}} &= 4\cdot 2/9  + 144\cdot 2/9 + 169\cdot 2/9 =: \alpha\\
\V T &= \V T 2/3 + \alpha  \quad \text{EVE}\\
\V T &= 3 \alpha.
  \end{align}
Here we use that $\1{X=A}$ etc are independent and Bernoulli distributed with success probability $p$, hence  $\V{\1{X=A}}= p q = 1/3\cdot 2/3$.

Grading
\begin{itemize}
\item Not using EVE: no points.
\item I saw this: $\E{T^{2}} = \ldots + (3 + \E )^{2} 1/3 + \ldots$. This is not correct of course.
\end{itemize}
\end{solution}
\end{exercise}

\begin{exercise}[1]
Why was it actually allowed to assume that $\E T <\infty$?
\begin{solution}
By the strong law of large numbers, any sequence of tunnel selections that excludes tunnel $A$ has probability zero.

Grading:
\begin{itemize}
\item Mention the LLN somehow. If not: 0 points.
\end{itemize}
\end{solution}
\end{exercise}

\section{Question}


We have a population of $X(t)$ individuals at time $t$. At time $t$, the time to the next birth is $Z\sim\Exp{\lambda X(t)+\theta}$, and the time to the next death is $Y\sim \Exp{\mu X(t)}$; $\lambda, \mu, \theta \geq 0$, and  rvs $Y$ and $Z$ are independent. Write $B(h)$ for the number of births during an interval of length $h$, and $D(h)$ for the number of deaths. (Hint, recall the relation between the Poisson distribution and the exponential distribution.)

In the sequel, take $h$ positive, but very, very small, i.e, $h\ll 1$. With this, we use the shorthand $o(h)$ to capture all terms of a polynomial in $h$ with a power higher than $1$, for instance,
\begin{equation}
  2h + 3 h^2+ 44 h^{21} = 2h + o(h).
\end{equation}
Like this we can hide all nonlinear terms of a polynomial in the $o(h)$ function. This is easy when we want to take limits, for example,
\begin{equation}
  \lim_{h\to 0} \frac{2h + 3h^{2} + 44 h^{21}} h
= 2 +  \lim_{h\to 0} \frac{o(h)} h  = 2 + 0.
\end{equation}
In other words, when computing this limit for $h\to 0$, we don't care about the details in $o(h)$ because $o(h)/h\to0$ anyway.

\begin{exercise}[1]
Provide intuitive motivation about the correctness of the following equality:
\begin{equation}
\P{B(h)=1, D(h)=0|X(0) = n } = (\lambda n  +\theta) h e^{-(\lambda n+\theta)h} e^{-\mu n h} + o(h).
\end{equation}
The $o(h)$ here is a subtlety to get the mathematics correct, but you  don't have to explain why this term is necessary.
\begin{solution}
 % Since the time to the next event is the minimum of two exponential distributions, this time is also exp distributed, with rate $\lambda X(0) + \theta + \mu X(0)$.
Since births and deaths  are exponentially distributed, we can use that $B(h) \sim \Pois{(\lambda X(t) + \theta)h}$ and $D(h) \sim \Pois{\mu X(0) h}$ when $X(0) = n$.

The subtlety is due to the fact that  during the time $h$ also multiple arrivals and  departures  can occur, but since these rates depend on the number people in the system, these rates need not be constant during the time interval $h$.
However, since such events have very small, in fact have $o(h)$ probability, we can capture all such details in the $o(h)$ terms.

Grading: mention the use of exponential and Poisson distribution: +1/2.
\end{solution}
\end{exercise}

\begin{exercise}[1]
Use the first degree Taylor's expansion, $f(h) \approx f(0) + hf'(0) + o(h)$, to show  that
\begin{equation}
\P{B(h)=0, D(h)=1|X(0) = n} = n \mu h + o(h).
\end{equation}
\begin{solution}
\begin{equation}
\P{B(h)=0, D(h)=1|X(0) = n } = e^{-(\lambda n + \theta)h} \mu n h e^{-\mu n h} = (1 - (\lambda n + \theta) h) \mu n h (1-\mu n h) + o(h) = \mu n h + o(h).
\end{equation}

Grading:
\begin{itemize}
\item Skipping the algebra: -1/2.
\end{itemize}
\end{solution}
\end{exercise}

\begin{exercise}[2]
Explain that
\begin{equation}
\E{X(t+h) | X(t) =n }   = n  + (\lambda n + \theta - \mu n )h + o(h).
\end{equation}
\begin{solution}
\begin{align}
\E{X(t+h) | X(t)=n }
  &= n \P{B(h)=0, D(h) = 0} + (n+1)\P{B(h)=1, D(h) = 0} \\
  &\quad + (n-1)\P{B(h)=0, D(h) = 1}  + o(h) \\
 &= n e^{-(\lambda n + \theta) h} e^{-\mu n h} + (n+1) (\lambda n + \theta) h + (n -1)\mu n h + o(h) \\
 &= n (1-(\lambda n + \theta) h) (1-\mu n h)  + (n +1) (\lambda n + \theta) h + (n -1)\mu n h + o(h) \\
 &= n + (\lambda n + \theta - \mu n) h + o(h).
\end{align}

Grading:
\begin{itemize}
\item Show also how to simplify the results of the first question. If not, -1/2.
\end{itemize}
\end{solution}
\end{exercise}

Write $M(t) = \E{X(t)}$.
\begin{exercise}[1]
Derive that
\begin{equation}
M(t+h)  =  M(t) + (\lambda  - \mu) M(t) h + \theta h + o(h).
\end{equation}
\begin{solution}
  Replace $n$ by $X(t)$ in $\E{X(t+h)|X(t)}$ to see that
\begin{equation}
\E{X(t+h) | X(t) }   = X(t)  + (\lambda  - \mu) X(t) h + \theta h + o(h).
\end{equation}
Take expectations left and right and use Adam's law.

Grading:
\begin{itemize}
\item No points for not mentioning Adam's law, or showing in some way that you used it.
\item Using Adam's law in the wrong way, i.e, not replacing the $n$ by $X(t)$ at most 1/2.
\end{itemize}

\end{solution}
\end{exercise}

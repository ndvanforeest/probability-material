\section*{Question}
\begin{exercise}[1]
Let $X$ follow the student's $t$ distribution with $\nu$ degrees of freedom. Consider the random variable $Y = \frac{1}{X}$. Find the CDF of $Y$, $F_Y(y)$, in terms of $\P{X\leq\frac{1}{y}}$.
\begin{solution}
We start by trying to find a formula for $F_Y(y)$. After drawing the function $y=\frac{1}{x}$ in the $x,y$-plane, it becomes obvious that
\begin{align*}
    F_Y(y) = \P{Y\leq y} = \begin{cases}
    \P{X\leq 0} + \P{X\geq\frac{1}{y}}&\quad\text{if}\;y>0\\
    \P{\frac{1}{y}\leq X\leq 0}&\quad\text{if}\;y<0
    \end{cases}.
\end{align*}
Draw it if this is not clear!\\
To calculate less, we notice that $\P{X\leq 0}=\frac12$, by symmetry. Then,
\begin{align*}
    F_Y(y) = \begin{cases}
    \frac12 + \P{X\geq\frac{1}{y}}&\quad\text{if}\;y>0\\
    \frac12&\quad\text{if}\;y=0\\
    \frac12 - \P{X\leq\frac1y}&\quad\text{if}\;y<0
    \end{cases}.
\end{align*}
See also exercise 8.9.11.
\\\\
Grading scheme:
\begin{itemize}
    \item Correct cases 0.5pt.
    \item No mistakes etc. 0.5pt.
\end{itemize}
\end{solution}
\end{exercise}

\begin{exercise}[1]
Show that $f_Y(y) = \frac{1}{y^2}f_X(\frac{1}{y})$ for all $y\neq0$.
\begin{solution}
\begin{align*}
    f_Y(y) = \frac{\d}{\d y} F_Y(y) &= \begin{cases}
    \frac{\d}{\d y}(\frac12 + \P{X\geq\frac{1}{y}})&\quad\text{if}\;y>0\\
    \frac{\d}{\d y}(\frac12 - \P{X\leq\frac1y})&\quad\text{if}\;y<0
    \end{cases}\\
    &=\begin{cases}
    -\frac{\d}{\d y}\P{X\leq\frac{1}{y}}&\quad\text{if}\;y>0\\
    -\frac{\d}{\d y}\P{X\leq\frac{1}{y}}&\quad\text{if}\;y<0
    \end{cases}.
\end{align*}
We can write, by definition:
\begin{align*}
    \P{X\leq\frac{1}{y}} = \int_{-\infty}^{\frac1y}{f_X(s)\d s},
\end{align*}
such that by the FTC,
\begin{align*}
    -\frac{\d}{\d y}\P{X\leq\frac{1}{y}} = \frac{1}{y^2}f_X(\frac1y)
\end{align*}
for $y\neq0$.\\
\\\\
Grading scheme:
\begin{itemize}
    \item Correct calculations 0.5pt.
    \item No mistakes etc. 0.5pt.
    \item Alternatively, use the transformation theorem to show this, if you didn't use the result from part (a). Be careful to correctly apply it.
\end{itemize}
\end{solution}
\end{exercise}

\begin{exercise}[1]
Let $Y$ be as in the previous question. What distribution does $Y$ follow when $\nu=1$?
\begin{solution}
When $\nu=1$, $X$ follows a Cauchy distribution. Then, $Y$ must also be Cauchy.
\\\\
Grading scheme:
\begin{itemize}
    \item Correct 1pt.
\end{itemize}
\end{solution}
\end{exercise}

\begin{exercise}[0.5]
What happens to the $t$ distribution when $\nu\rightarrow\infty$?
\begin{solution}
It converges to the standard normal distribution.
\\\\
Grading scheme:
\begin{itemize}
    \item Correct 0.5pt.
\end{itemize}
\end{solution}
\end{exercise}

\begin{exercise}[1.5]
Let $\nu>1$. For what value(s) of $y$ is $f_Y(y)$ maximal? You may neglect the possibility that $y=0$.
\begin{solution}
For $\nu>1$, we have that
\begin{align*}
    f_Y(y)\propto \left(y+\frac{1}{\nu y}\right)^{\frac{\nu}{2}+\frac12}.
\end{align*}
The FOC tells us that the mode is at
\begin{align*}
    &\frac{\d}{\d y} \left(y+\frac{1}{\nu y}\right)^{\frac{\nu}{2}+\frac12} = \left(y+\frac{1}{\nu y}\right)^{\frac{\nu}{2}-\frac12}\left(1-\frac{1}{\nu y^2}\right) = 0\implies\\
    &1-\frac{1}{\nu y^2} = 0\implies\\
    &y=\pm\frac{\sqrt{\nu}}{\nu}.
\end{align*}
Clearly, these must be the maximum values $f_Y$ takes on; if they were minima the PDF would not integrate to unity, and they cannot be saddle points (the only other option as the PDF is symmetric) since then there would be a different maximum, which the FOC would show. Alternatively, you could look at the second derivative.
\\\\
Grading scheme:
\begin{itemize}
    \item Correct FOC 1pt.
    \item Something about it being a maximum (lenient) 0.5pt.
\end{itemize}
\end{solution}
\end{exercise}

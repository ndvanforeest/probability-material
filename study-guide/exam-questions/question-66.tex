\section{Question}
$N$ number of employees will participate in the pension system. There are currently two types of pension schemes, Plan A and Plan B. Each employee independently chooses Plan A with probability $p$, and Plan B with probability $1-p.$
% covariance, multivariate normal
\begin{exercise}[2]
 Suppose $N\sim\text{Pois}(\lambda).$ Let $X_A$ be the number of people that choose Plan A and $X_B=N-X_A$ be the number of people that choose Plan B. Find $Var(X_A-X_B)$ and $\rho_{X_B,N}.$
%Find $\cov{2N_1+N_2, 2N_1-N_2}.$

\begin{solution}
First notice that $X_A|N \sim \text{Bin}(N, p)$ and $X_B |N \sim \text{Bin}(N, 1-p)$.By the chicken-egg theory, $X_A\sim\text{Pois}(\lambda p)$ and $X_B \sim \text{Pois}(\lambda (1-p))$ are independent. (0.5 points) \\
Then it follows
\begin{align*}
    Var(X_A-X_B)=Var(X_A)+Var(X_B)=\lambda p+\lambda(1-p)=\lambda  \cdot \cdot \cdot \text{ (0.5 points)}
\end{align*}
And \begin{align*}
    &\cov{X_B,N}\\
    =&\cov{X_B,X_A+X_B}\\
    =&\cov{X_B,X_A}+\cov{X_B,X_B}\\
    =&Var(X_B)\\
    =&\lambda (1-p)  \cdot \cdot \cdot \text{ (0.5 points)}\\
\end{align*}
Then $$  \rho_{X_B,N}
    =\frac{\cov{X_B,N}}{sd(X_B)sd(N)}
    =\frac{\lambda (1-p)}{\sqrt{\lambda (1-p)}\sqrt{\lambda}}
    =\sqrt{1-p}\cdot \cdot \cdot \text{ (0.5 points)}$$
\end{solution}
\end{exercise}

\begin{exercise}[2]
Suppose $N=500.$ Two new pension schemes are now introduced, called Plan C and Plan D. Each of the 500 employees now independently chooses one of the four pension schemes with equal probabilities $\frac{1}{4}.$  Let $X_i$ be the number of employees that choose Plan $i$, i=A,B,C,D, $\sum_{i}X_i=N=500$. Find $\cov{X_B,X_C}$ and $\rho_{X_B,X_C}.$
\begin{solution}
First notice that $X_j \sim \text{Bin}(500,\frac{1}{4}), j=A, B, C, D.$(0.5 points)\\
Then we know $\boldsymbol{X}=(X_A,X_B,X_C, X_D) \sim Mult_4(500,\frac{1}{4}).$(0.5 points)\\
Using the property of a Multinomial distribution,
\begin{align*}
    \cov{X_B,X_C}=-\frac{500}{4^2},  \cdot \cdot \cdot \text{ (0.5 points)}
\end{align*}
\begin{align*}
       \rho_{X_B,X_C}=\frac{ \cov{X_B,X_C}}{sd(X_B)sd(X_C)}=-\frac{500/4^2}{500(1/4)(3/4)}=-\frac{1}{3}. \cdot \cdot \cdot \text{ (0.5 points)}
\end{align*}
\end{solution}
\end{exercise}

\noindent Consider the following codes:
\begin{minted}{R}
library(mvtnorm)
set.seed(999)
A<-c(1,2)
B<-c(2,3)
C<-A+B
D<-cbind(A,B)
X<-rmvnorm(200,mean=C,sigma=D)
output<-colMeans(X)
\end{minted}

\begin{minted}{python}
import random
import numpy as np
random.seed(999)
A = np.array([1,2])
B = np.array([2,3])
C = A+B
D = np.transpose([A,B])
X = np.random.multivariate_normal(C, D, size = 200)
output = X.mean(axis=0)
\end{minted}


\begin{exercise}[1]
Explain in detail the purpose of \textbf{Line 1, 2, 5, 6, 7, 8} of the above codes.
\begin{solution}
Line 1: Load the package "mvtnorm" so that we can use the function \textit{rmvnorm}.\\
Line 2: Set a random seed to reproduce the same results.\\
Line 5: Generate a vector $C=(3,5)$.\\
Line 6: Generate a 2$\times$2 matrix of D=$\begin{pmatrix}1&2\\2&3\end{pmatrix}$.\\
Line 7: Generate 200 multivariate normally distributed variables $\textbf{X}={x_1,x_2}$ with mean equal to $(3,5)$ and variance equal to matrix D.\\
Line 8: Calculate the average value for each column of matrix $X$.\\
\textbf{(0.5 points for mentioning at least 3 of the above.)}
\end{solution}
\end{exercise}

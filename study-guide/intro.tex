\chapter{Introduction}
\label{sec:orgb865fed}
\addcontentsline{toc}{chapter}{Introduction}

This study guide contains material (additional exercises and discussions) organized per chapter of BH. \\~\\
For Probability Theory for EOR (Chapter 1- Chapter 6 Section 6.3): 
\begin{enumerate}
	\item The simple questions and exercises are based on each section of BH and are meant to practice while you read.
	Most questions are (often much) simpler than exam ones but help you read and study well. It takes time and attention to understand definitions and notation. Mind that good notation and good understanding strongly correlate.
	\item When you get stuck, it might also be helpful to watch videos provided on Brightspace to help you better understand the relevant concepts. In knowledge clips, we also discuss some exercises. 
	\item We provide additional remarks when there are common confusions around certain concepts.
	\item Some problems are not easy, which may be slightly above exam level. It is useful sometimes to challenge yourself and not be frightened, and we all make mistakes when trying something difficult, which we should also be proud of.
	\item Coding is not in the exam range, but since you will use it in many other courses (e.g., Probability Distribution), we also provide some codes when useful to give you some taste of \textit{(Monte Carlo) simulations}.  
\end{enumerate}
For Probability Distribution: 
\begin{enumerate}
\item The simple questions and exercises are based on each section of BH and  are meant to practice while you read.
These questions are (often much) simpler than exam questions, but just help you to read and study well. It takes time and attention to understand definitions and notation. Mind that good notation and good understanding strongly correlate.
\item The part related to the obligatory exercises of BH provides motivational comments, hints and solutions.
\item The third part contains challenges.
These problems are (quite a bit) above exam level, hence optional.
However, if you like to be intellectually challenged, then you'll like these problems a lot.
\end{enumerate}


In general, when working on the exercises, try first hard to find the answer and \emph{write it down} on paper.  Only after having written your answer on paper, meticulously compare your work with ours. Like this you' ll get a lot of feedback, and you'll see that it is quite hard to get the details right.

Finally, we included many old exam questions. Of course you are not expected to make them all. Instead  do a just few until you feel comfortable with the level.

The selection of exercises in the table above are the bare minimum; I advice you to do more.
To assure you, I found the problems quite hard at times; probability never `comes for free'; not for you, not for me, not for anybody.
You can expect to spend between 30 minutes (and sometimes more) per problem; if you are serious.


Here is a list of good, and important, advice when making the exercises.
(As a student I did not always do this, partly because I was not aware about how useful this advice is. Hopefully you are smart enough to avoid making the same mistakes as I did as a student.)
\begin{itemize}
\item  Read an example in the book. Close the book, and try to redo the example. When I try, I often fail. Why is that? Simple: I did not really   think about the example while just reading it, I skimmed it.  But you should get used to the fact that reading requires pen and paper.
\item  Before trying to solve an exercise, read all parts of it, i.e., part a, b, etc. Ensure you understand the problem.
\item Before actually solving  an exercise, \emph{make a plan on how to solve it}. A first step is to look for simple corner cases (set things to zero, make certain probabilities equal to one, and so on), make extra assumptions that simplify the problem, and solve the problem under these simplifying (stronger) assumptions. Then drop an assumption, and try to generalize to a pattern or some property you expect to hold. You'll be astonished to see how many problems you can actually solve by following this strategy. And even if you cannot solve it with this approach, the corner cases help to check throughout whether you're still working in the right direction. Also, reduce the problem to simpler cases you do understand. Try to solve the simpler problem first, and then generalize.
\item Carry out your plan, and \emph{relax} if  you cannot directly find the answer.
\item Look back after solving the problem, and try to find a general pattern you used to solve the problem. Can you use this for other problems too?
\item Look back again at the problem some time later. In other words, do not solve  a problem just once, but also a few weeks later again. This is often very revealing.
\item Work every day a reasonable amount of time. This is much more effective than working 10 h on one day, and not at all the next. The concept is often called `Kaizen': try to improve every day a little bit. Over the course of time, you'll be amazed how much you can achieve.
\item Finally, when I am stuck, this piece of advice of Jim Rohn (an author on personal development) helps: `Don't wish it was easier, wish you were better.'
\end{itemize}

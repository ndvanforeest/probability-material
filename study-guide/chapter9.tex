% arara: pdflatex: { shell: yes }
% arara: pythontex: {verbose: yes, rerun: modified }
% arara: pdflatex: { shell: yes }
% arara: move: {  files: ['chapter9.pdf'], target: ['..'] }

\documentclass[bh_problems_check]{subfiles}

\opt{check}{
\Opensolutionfile{hint}
\Opensolutionfile{ans}
}

\begin{document}



\chapter{Chapter 9: Exercises and remarks}

\section{Section 9.1}

\begin{remark}
Skip BH.9.1.6.
\end{remark}

\begin{exercise}
BH.9.1.7. We will also simulate this in an assignment.

I  like this example as it  shows how to make optimal decisions under uncertainty, but I have to admit that I don't understand the reasoning, or the use of conditional probability to solve this problem. Here is how I would solve the problem.
\begin{enumerate}
\item Explain that $W=(V-b)\1{b \geq \alpha V}$, with $\alpha=2/3$, is the rv that models our payoff.
\item Why is this wrong: $\E W = \E{(V-b)\1{b \geq \alpha V}} = V \E{\1{b \geq \alpha V}} -b \E{\1{b \geq \alpha V}}$?
\item Compute $\E W$, and provide a bound on $\alpha$ to ensure that $\E W > 0$.
\end{enumerate}
\begin{solution}
\begin{enumerate}
  \item The indicator function is zero when the bid is rejected, hence there is a zero payoff. When the bid is accepted the indicator function is one and the resulting payoff is $V-b$, i.e., the prize value minus the offer.
  \item $V$ is a rv, hence cannot be taken out the expectation.
  \item Since $W$ is a function of $V$ we can apply LOTUS and find the expectation by integration. First we assume $b/\alpha < 1$.
\end{enumerate}
\begin{align*}
\E W &= \int_0^1 (v-b) \1{b\geq \alpha v} \d v = \int_0^{b/\alpha} v \d v - b \int_0^{b/\alpha}  \d v \\
  &= b^2/2\alpha^2 - b^2/\alpha = \frac{b^{2}}{\alpha^{2}} \frac{1-2\alpha}{2}.
\end{align*}
Clearly, we need to assume $\alpha<1/2$ to ensure that $\E W > 0$.

Next, assume that $b>\alpha$. Then the expected payout is $\E{W} = 1/2 -b $, as any bid is accepted (why?). Thus, if $b<1/2$ the expected payout is positive. But we assumed that $b>\alpha$, hence $\alpha < b < 1/2$, so again $\alpha<1/2$ for a positive payout.
\end{solution}
\end{exercise}

\begin{exercise}
  BH.9.1.7. continued. By the previous exercise, we know that only when $\alpha<1/2$ we should participate in the game. In other words, when our bid $b$ should be larger than $\alpha V$ to be accepted. If we know $\alpha$, what would be our optimal bid $b$?
\begin{solution}
  When $\alpha < 1/2$ the expected payoff is positive so it becomes interesting. We can write the total payout as:
 \begin{align*}
   \E W &= (1/2 - b) \1{b>\alpha} + \frac{1-2\alpha}{2\alpha^2}b^2 \1{b\leq \alpha}\\
  \end{align*}
  Assume $0<\alpha<1/2$. The expected payoff is a piecewise continuous function since $\lim_{b\uparrow \alpha} \E W  = 1/2 - \alpha = \lim_{b\downarrow \alpha} \E W$.
Now, the left term in the RHS of the equation above, i.e., $(1/2 - b)\1{b>\alpha}$, decreases strictly for $b>\alpha$, while in the right term $b^{2}$ increases strictly for $b<\alpha$. Hence, $\E W$ achieves its maximum for $b=\alpha$.
Therefore, the optimal betting amount is $\alpha$ and results in an expected payoff of $1/2 - \alpha$.
\end{solution}
\end{exercise}

\begin{exercise}
BH.9.1.8. Apply the same type of argumentation to find $\E X$ when $X\sim \FS{p}$.
\begin{solution}
$\E X = 1 + q\E X$, because we have to throw at least once, and with probability $q$, we start again. Hence, $\E X = 1/(1-q) = 1/p$.
\end{solution}
\end{exercise}

\begin{exercise}
BH.9.1.8. Use first step analysis to  find $N_{r} := \E X$ when $X\sim \NBin{r,p}$.
\begin{solution}
Suppose the first throw is a success, then we need $r-1$ more successes, if the first throw is a failure, we are back at `hole one'. Thus, $N_{r} = p N_{r-1}+q(1+N_{r})$. Simplifying (and using that $p/(1-q)=1$) gives $N_{r} = N_{r-1} + q/p$, which implies $N_{r} = r q/p$.
\end{solution}
\end{exercise}

\begin{exercise}
BH.9.1.9. I reason slightly differently here. Write $N_{r}$ for the number of throws required to reach $r$ heads in row. Then I need $N_{r-1}$ throws in expectation to reach the state in which there are $r-1$ heads in row. Suppose now that we are in this state, i.e.,  there are $r-1$ heads in row. Then, if I throw heads, with probability $p$, I reach the state with $r$ heads in row, and I am done. However, if I throw tails, with probability $q$, I have to start all over again.  Use this argument to derive the recursion $N_{r} = N_{r-1} + p\cdot 1 + q(1+N_{r})$.  Solve this to obtain
\begin{equation}
N_{r} = \sum_{i=1}^{r} 1/p^i.
\end{equation}
\begin{hint}
  If this is new to you, check out appendix A.4
\end{hint}
\begin{solution}
\begin{equation}
N_{r} = N_{r-1} + p\cdot 1 + q(1+N_{r}) \implies N_{r} = N_{r-1}/p + 1/p \implies N_{r} = \sum_{i=1}^{r} 1/p^i.
\end{equation}
\end{solution}
\end{exercise}

\begin{exercise}
Compute the expected outcome of a die throw (with a 6-sided die), given that the outcome is even. Introduce proper notation for random variables and events.
\begin{solution}
Let $X$ be the outcome of the die throw (note that $X$ is a random variable) and let $A$ be the event that the outcome is even. Then
\begin{equation*}
\E{X\given A} = 2 \P{X=2|A} + 4 \P{X=4|A} + 6 \P{X=6|A} = \tfrac13 \cdot (2+4+6)  = 4.
\end{equation*}
We conclude that $\E{X\given A} = 4$.
\end{solution}
\end{exercise}

\begin{exercise}
  BH.9.1.10.  Let $p_i$, $0\leq i\leq b$, be the probability to hit $b$ before $0$, with $i$ being the current position of the drunkard.
\begin{enumerate}
\item Why is $p_0=p_1/2$?
\item Why is $p_1=p_2/2$?
\item Why is $p_{b}=1$?
\item Explain the recursion $p_i=p_{i-1}/2 + p_{i+1}/2$ for $1 < i <b$?
\item Show that the previous points imply that $p_i =  \alpha i$ for $0<i<b$ for any $\alpha$ we like.
\item Combine the fact that $p_{b}=1$ with $p_{i} = \alpha i$ for all $0<i<b$ to see that $\alpha=1/b$.
\item Conclude that $p_0=1/2b$.
\end{enumerate}
\begin{hint}
  Read the gambler's ruin problem BH.2.7.3
\end{hint}

\begin{solution}
  \begin{enumerate}
    \item As in BH, condition on the first step. If the drunkard starts at zero, and makes a step to the left, he is at position $-1$. To get to $b$, with $b>0$, requires to pass $0$ again. Hence, in this case, the drunkard did not reach $b$ before 0.

    When the drunkard makes a step to the right, which occurs with probability 1/2, then the drunkard moves to position $1$. Writing $p_{1}$ for the probability to reach $b$ before $0$, it must be that $p_0 = p_1/2$.

    Written in another way: $p_0=p_{-1}\cdot 1/2 + p_1\cdot 1/2$, but $p_{-1}=0$, hence $p_0=p_1/2$.
    \item The reasoning is the same as in the  previous step. When the drunkard is in state $1$ and moves to the left, he hits 0 before $b$ so the process stops. If he moves to the right, then the process continues. Therefore $p_1=p_2/2$.
    \item If the drunkard starts in $b$, the probability to hit $b$ before $0$ is 1. (In other words, what is the probability to be in $b$ when the drunkard starts in $b$?)
    \item In some state $i$, $2\leq i < b$, condition again on whether the inebriate moves to the left or to the right.  The probability $p_{i}$ of reaching $b$ before 0 is $p_{i+1}$ when he makes a step to the right and $p_{i-1}$ when he moves to the left.  Since a step in both directions is equally likely, the equation of the problem follows.
    \item Plug in $p_i = \alpha i$ in the RHS:
      \begin{equation*}
    \frac{\alpha(i-1)}{2} + \frac{\alpha(i+1)}{2} = \alpha i - \frac{\alpha}{2} + \frac{\alpha}{2} = \alpha i = p_i
      \end{equation*}
    This shows $p_i = \alpha i$ is a solution to this recursive equation for any choice of $\alpha$.
    \item Use the boundary condition $ p_b = 1$ in the expression $p_{i}\alpha = i$ for $i=b-1$. Solving for $\alpha$ in $\alpha(b-1) = \alpha(b-2)/2+1/2$,      gives $\alpha=1/b$.
    \item Since $p_1=p-2/2 = 2/b\cdot 1/2 = 1/b$ and $p_0 = p_1 / 2$, we get that $p_{0}=1/2b$.  \end{enumerate}

  Another way you might have found $p_0$ is by noticing the following pattern when plugging in the previous values in the equation in $3$.
\begin{align*}
\textrm{As noted before } \quad & p_1 = p_2/2 \\
\textrm{Plugging the above result } \quad & p_2 = p_1/2 + p_3/2 = p_2/4 + p_3/2 \\
  \textrm{This implies } \quad & p_2 = 2/3 p_3 \\
  \textrm{Continuing substituting } \quad & p_3 = 1/3 p_3 + p_4 / 2\\
  & p_3 = 3/4 p_4 \\
  \textrm{Notice a pattern? } \quad & \dots \\
  & p_{b-1} = \frac{b-1}{b} p_b = \frac{b-1}{b}\\
  \textrm{As we already knew} \quad & p_b = 1 \\
\textrm{Now we can recursively substitute what we know to find } \quad & p_1 = \frac{1}{2} \frac{2}{3} \frac{3}{4} \dots \frac{b-1}{b} = \frac{1}{b} \\
\textrm{Then using the result from $1$} \quad & p_0 = p_1/2 = \frac{1}{2b}
\end{align*}
\end{solution}
\end{exercise}



\section{Section 9.2}
\label{sec:section-9.2}

\begin{remark} About BH. 9.2.1. This definition is subtle, and it takes time to understand.
Here is a slightly different explanation; perhaps it's useful for you.

Take some random variable $X$, say.
Then, as in Chapter 7,  we can be interested in  $\E{g(X)}$, i.e., the expectation of the rv $g(X)$.

When $Y$ is continuous we can compute $\E{Y \given X=x}$ with  the conditional CDF
\begin{align*}
\E{Y\given X=x } =\int f_{Y| X}(y | x) \d y.
\end{align*}
(For discrete rv., replace the integral by the PMF.)
Observe that this is just a function of $x$; define this function as $g(x) = \int f_{Y| X}(y | x)\d y$. And now, as before, we consider the random variable $g(X)$, and we  \emph{call this rv the conditional expectation} of $Y$ given $X$.

It is true that $X$ plays some sort of double role---first we use it in the conditioning in the definition of the function $g$, and then we plug it into  $g$ again---and this is perhaps confusing. But I finally `got  it', when I understood that $g$ can be interpreted as just some function of $x$. And then we compute $\E{g(X)}$, and so on.
\end{remark}



\begin{exercise}
BH.9.2.2. Is $\E{Y\given \1{A}}$ a number or a rv?
\begin{hint}
  Is $\1{A}$ a rv or an event?
\end{hint}
\begin{solution}
  It is a rv as the indicator function isn't crystallized, so we are conditioning on a rv ($\1{A} \sim \Bern{\P{A}}$) We could write $\E{Y\given \1{A}} = \E{Y\given A}\1{A} + \E{Y\given A^C}(1-\1{A}).$
\end{solution}
\end{exercise}


\section{Section 9.3}
\label{sec:section-9.3}

\begin{remark}
On BH.9.3.2 (Taking out what is known.) Perhaps it is easier to crystallize  $X$ into $x$. Then $g(x) = \E{h(x) Y|X} = h(x) \E{Y|X}$, because $h(x)$ is just a function. The rvs $\E{H(X) Y|X}$ and $h(X) \E{Y|X}$ are then both  equal to $g(X)$.
\end{remark}

\begin{exercise}
On BH.9.3.9. Show that $\cov{Y-\E{Y|X}, \E{Y|X}}=0$.
\begin{solution}
We have that $\E{Y-\E{Y|X}}=0$. Hence, $\E{Y-\E{Y|X}}\E X = 0$. Then define $h(X) = \E{Y|X}$ and apply BH.9.3.9 to see that $\E{(Y-\E{Y|X})h(X)} = 0$. From the definition of the covariance, $\cov{W, Z} = \E{WZ} - \E W \E Z$, we have shown that both terms are zero.
\end{solution}
\end{exercise}


\section{Section 9.4}
\label{sec:section-9.4}


\begin{exercise}
Consider a casino where, for any $a>0$, it is possible to pay $a$ euro and get a chance of $\tfrac15$ on receiving $4a$ euro and a chance of $\tfrac45$ of receiving nothing.
Adam enters the casino with $b$ euros, and bets half of his money on this gamble.
Let $X$ be the amount of money he has after the gamble.
After that, he again bets half of the money he then has (i.e.
half of $X$) on this gamble.
Let $Y$ be the amount of money he has after the second gamble.
\begin{enumerate}
\item Compute $\E{X}$.
\item Compute $\E{Y|X}$.
\item Compute $\E{Y}$.
\end{enumerate}
Explicitly mention the laws/rules you use.
\begin{solution}

1. Since Adam keeps $b/2$ and does the gamble with $a = b/2$, we have
\begin{equation*}
\E{X} = b/2 + \tfrac15 \cdot 4(b/2) + \tfrac45 \cdot 0 = 0.9 b.
\end{equation*}
2. The computation is the same as in part 1., but with $X$ instead of $b$:
\begin{equation*}
\E{Y|X} = X/2 + \tfrac15 \cdot 4(X/2) + \tfrac45 \cdot 0 = 0.9 X.
\end{equation*}
Note that the result is a random variable. \\
3. Using Adam's law (and linearity of expectation), we conclude that:
\begin{equation*}
\E{Y} = \E{\E{Y|X}} = \E{0.9X} = 0.9\E{X} = 0.81b.
\end{equation*}
In general, if Adam would do this $n$ times, the expected amount of money he has after $n$ such gambles would be $0.9^n b$. This would be very difficult to show without Adam's law!
\end{solution}
\end{exercise}


\begin{exercise}
Let $N \sim \Pois{\lambda}$, and let $X|N \sim \Bin{N, p}$, where $p \in (0,1)$ and $\lambda > 0$ are known constants.
Compute $\E{X}$ using Adam's law.
Check your answer using the chicken-egg story; with this story you can also obtain the distribution of $X$.
\begin{solution}
We have $\E{X|N} = Np$, so using Adam's law (and linearity of expectation), we conclude that $\E{X} = \E{\E{X|N}} = \E{Np} = \E{N}p = \lambda p$.

This is in accordance with $X \sim \Pois{\lambda p}$, which was shown in the chicken-egg story.

Some students reported answers like $\lambda^{2}p$. This is wrong, and can be immediately seen by checking units: the unit of $\lambda$ being 1 per time.

Others wrote $\E{X\given N= n} n p$, hence $\E X = \E{\E{X\given N}}= \E{np} = np$.

Apparently, such students are not aware of the idea that $\E{X\given N}$ is a random variable.
When this happens during the exam, you will score 0 points for that particular part of a question.

\end{solution}
\end{exercise}


\begin{exercise}
Correct? If $A$ is an event and $\1{A}$ is its indicator, then for all random variables $X$ we have $\E{X \given A} = \E{X \given \1{A}}$.
\begin{solution}
Incorrect:  $\E{X \given A}$ is a number since $A$ is an event, whereas $\E{X \given \1{A}}$ is a random variable since $\1{A}$ is a random variable. A correct statement is $\E{X \given A} = \E{X \given \1{A} = 1}$.
\end{solution}
\end{exercise}

\begin{exercise}
Correct? If $X$ and $Y$ are independent, then $\V{\E{Y \given X}} = 0$.
\begin{solution}
Correct, if  $X$ and $Y$ are independent, then $\E{Y \given X} = \E{Y}$ which is a constant (formally, a degenerate random variable).
Since the variance of a constant is 0, we conclude that $\V{\E{Y \given X}} = 0$.
\end{solution}
\end{exercise}

\begin{exercise}
Let $X \sim \Exp{\lambda}$, and let $a$ be a constant.
\begin{enumerate}
\item Compute $\E{X \given X \geq a}$ using an integral and an indicator.
\item Explain the answer using a property of the exponential distribution.
\end{enumerate}
\begin{solution}
1. We compute $\E{X \given X \geq a}$ as follows:
\begin{align*} \E{X \given X \geq a} &= \int_0^\infty y f(y|A) \d y
\\ &= \int_0^\infty y \frac{\lambda e^{-\lambda y} \1{y \geq a}}{e^{-\lambda a}} \d y
\\ &= \lambda \int_a^\infty y e^{-\lambda (y-a)} \d y
\\ &= -y e^{-\lambda (y-a)}\biggr|_a^\infty +  \int_a^\infty e^{-\lambda (y-a)} \d y
\\ &= a   - \frac1{\lambda} e^{-\lambda (y-a)}\biggr|_a^\infty   = a+\frac1\lambda.    \end{align*}

2.
The result also follows from the memoryless property, which states that conditional on the event that $X \geq a$, we have that $X-a | X \geq a \sim\Exp{\lambda}$.
\end{solution}
\end{exercise}

\begin{exercise}
A hat contains 9 fair coins and one coin that lands heads with probability 0.8. You pick a coin from the hat uniformly at random and toss it 10 times. Let $A$ be the event that you pick a fair coin, and let $X$ be the number of heads. Let $B$ be the event that the first four tosses all show heads.

\begin{enumerate}
\item Compute $\E{X \given A}$.
\item Compute $\E{X \given A^c}$.
\item Compute $\E{X}$.
\item Compute $\P{B}$.
\item Compute $\P{A \given B}$.
\item Compute $\E{X \given B}$.
\item Compute $\E{X \given B^c}$. \textit{Hint}: it is not necessary to compute $\P{A \given B^c}$.
\end{enumerate}

\begin{solution}
1. Note that $X \, |\, A \sim \Bin{10, 0.5}$, so $\E{X \given A} = 10 \cdot 0.5 = 5$. \\
2. Note that $X \, |\,  A^c \sim \Bin{10, 0.8}$, so $\E{X \given A^c} = 10 \cdot 0.8 = 8$. \\
3. By LOTE we have $\E{X} = \P{A}\E{X \given A} +\P{A^c} \E{X \given A^c}  = 0.9 \cdot 5 + 0.1\cdot 8 = 5.3$. \\
4. Note that  $\P{B \given A}  = 0.5^4$ and $\P{B \given A^c}  = 0.8^4$.
By LOTP we have
\begin{equation*}
\P{B} = \P{A}\P{B \given A} +\P{A^c} \P{B \given A^c}  = 0.9 \cdot 5 + 0.1\cdot 8 = 0.09721.
\end{equation*}
5. By Bayes' rule $\P{A \given B} = \frac{\P{B \given A}\P{A}}{\P{B}} \approx 0.57864$. \\
6. Note that $\E{X \given A, B} = 4 + 6 \cdot 0.5 = 7$ and $\E{X \given A^c, B} = 4 + 6 \cdot 0.8 = 8.8$. By LOTP with extra conditioning we have
\begin{equation*}
\P{X \given B} = \P{A \given B}\E{X \given A, B} + \P{A^c \given B}\E{X \given A^c, B} \approx 7.75844.
\end{equation*}
7. By LOTE we have $\P{B}\E{X \given B} +\P{B^c} \E{X \given B^c}  = \E{X} = 5.3$. We know $\P{B}$ and $\E{X \given B}$, so solving this for $\E{X \given B^c}$ yields $\E{X \given B^c} \approx 5.035$.

One or more students wrote the LOTE as $\E X = \sum_{Y} \E{X\given Y} \P{Y}$. This is wrong, as you cannot sum over a rv. This is correct: $\E{X} = \sum_{y} \E{X\given Y=y} \P{Y=y}$, so sum over the \emph{outcomes} of a rv.

\end{solution}
\end{exercise}


\begin{exercise}
Consider random variables $X, Y \in [0,1]^2$ with joint PDF $f_{X,Y}(x,y) = 2 \1{x \leq y}$.

Determine $\E{Y \given X}$ and $\E{X \given Y}$.
\begin{solution}
The marginal density of $X$ is given by $f_X(x) = 2(1-x)$.

So the conditional density is given by $f_{Y|X}(y|x) = \frac{f_{X,Y}(x,y)}{f_X(x)} = \frac{\1{x \leq y}}{1-x}$. Hence,
\begin{equation*}
\E{Y \given X = x} = \int_0^1 y \frac{\1{x \leq y}}{1-x} \d y  = \frac1{1-x}  \int_x^1 y \d y = \frac1{1-x} \left[\tfrac12y^2\right]_x^1 = \frac{\tfrac12\left(1-x^2\right)}{1-x} =  \tfrac12 \left(1+x\right) .
\end{equation*}
We conclude that $\E{Y \given X} = \tfrac12(1+X)$.


The marginal density of $Y$ is given by $f_Y(y) = 2y$.

So the conditional density is given by $f_{X|Y}(x|y) = \frac{f_{X,Y}(x,y)}{f_Y(y)} = \frac{\1{x \leq y}}{y}$. So
\begin{equation*}
\E{X \given Y = y} = \int_0^1 x \frac{\1{x \leq y}}{y} \d x  = \frac1{y}   \int_0^y x \d x = \tfrac12 y.
\end{equation*}
We conclude that $\E{X \given Y} = \tfrac12Y$.

Some students wrote for instance $\E{X\given Y} = y/2$.
Apparently, such students are not aware of the idea that $\E{X\given N}$ is a random variable.
When this happens during the exam, you will score 0 points for that particular part of a question.

\end{solution}
\end{exercise}





\begin{exercise}
Prove that $\E{X \given X \geq a} > \E{X}$ for any $a$ with $0 < \P{X \geq a} < 1$.
\begin{solution}
Note that $\E{X \given X \geq a} \geq a > \E{X \given X < a}$. By LOTE:
\begin{align*}
\E{X} &= \P{X \geq a}  \E{X \given X \geq a} + \P{X <a}  \E{X \given X < a}
\\  &< \P{X \geq a}  \E{X \given X \geq a} + \P{X <a}  \E{X \given X \geq a}
\\  &= \E{X \given X \geq a},
\end{align*}
where the inequality is strict since $ \P{X <a} > 0$.
\end{solution}
\end{exercise}




\begin{exercise}
Let $N \sim \Pois{\lambda}$ and let  $X|N \sim \Bin{N, p}$, where $p \in (0,1)$ and $\lambda > 0$ are known constants. Find $\E{N \given X}$.
\begin{hint}
  For a smart argument, use the chicken-egg story.
  Recall that the number of hatched eggs and the number of unhatched eggs are independent (since $N \sim \Pois{\lambda}$); i.e.
  $N-X$ and $X$ are independent.
\end{hint}
\begin{solution}
With the hint,
\begin{equation*} \E{N \given X} = \E{N-X \given X} + \E{X \given X} = \E{N-X} + X = \lambda(1-p) + X. \end{equation*}
As a check, $\E{ \E{N \given X}} = \E{\lambda(1-p) + X} = \lambda(1-p)  + \lambda p = \lambda  = \E{N}$.

Here is straightforward computation.
You should check each and every step as they are based on pattern recognition.
\begin{align}
  \label{eq:22}
\E{N \given X=k}
&= \sum_{n=k}^{\infty} n \P{N=n \given X=k } \\
&= \frac{1}{\P{X=k}} \sum_{n=k}^{\infty} n e^{-\lambda}\frac{\lambda^{n}}{n!} {n \choose k} p^{k}(1-p)^{n-k} \\
&= \frac{1}{\P{X=k}}\sum_{n=k}^{\infty} n e^{-\lambda}\frac{1}{n!} \frac{n!}{k!(n-k)!}(\lambda p)^{k}(\lambda(1-p))^{n-k} \\
&= \frac{e^{-\lambda p} (\lambda p)^{k}/k!}{\P{X=k}}\sum_{n=k}^{\infty} n e^{-\lambda(1-p)}\frac{1}{(n-k)!}(\lambda(1-p))^{n-k} \\
&= \sum_{n=k}^{\infty} n e^{-\lambda(1-p)}\frac{1}{(n-k)!}(\lambda(1-p))^{n-k} \\
&= \sum_{n=0}^{\infty} (n+k) e^{-\lambda(1-p)}\frac{1}{n!}(\lambda(1-p))^{n} \\
&= k  + \sum_{n=0}^{\infty} n e^{-\lambda(1-p)}\frac{1}{n!}(\lambda(1-p))^{n} \\
&= k  + \lambda (1-p).
\end{align}
Hence, $\E{N\given X}= \lambda(1-p) + X$. Since $\E X = \lambda p$, we get $\E N = \lambda$ with Adam's law, as above.
\end{solution}
\end{exercise}

\section{Section 9.5}
\label{sec:section-9.5}

\begin{exercise}
BH.9.5.1. Is $\V{Y|X}= \V{\E{Y|X}}$?
\begin{hint}
\end{hint}
\begin{solution}
\end{solution}
\end{exercise}



\begin{exercise}
Use Eve's law to show that $\V Y \geq \V{\E{Y\given X}}$.
\begin{solution}
By Eve's law,
\begin{align}
    \V Y = \E{\V{Y\given X}} + \V{\E{Y\given X}} \geq \V{\E{Y\given X}},
\end{align}
since $\V{Y\given X} \geq 0$ for all $X$, which implies that $\E{\V{Y\given X}} \geq 0$.
\end{solution}
\end{exercise}


\begin{exercise}
Let $Z\sim \mathcal{N}(\mu, \sigma^2)$ and $Y=\sqrt{Z}+Z^2.$ Find $\V{Y|Z}$.
\begin{solution}
Conditional on $Z$, $Y$ is a constant, and the variance of a constant is 0. Hence, $\V{Y|Z} = 0$.
\end{solution}
\end{exercise}


\begin{exercise}
Correct? $\V{Y}=\V{Y|A}\P{A}+\V{Y|A^c}\P{A^c}$ for any random variable $Y$ and event $A$.

\begin{solution}
Incorrect. Counterexample: Let $Y\sim$ Bern(1/2) and $A$ be the event $Y=0$. then Var($Y|A$) and Var($Y|A^c$) are both 0, but Var($Y$)=1/4.
\end{solution}
\end{exercise}


\begin{exercise}
Let $X, Y$ be random variables. Explain the difference between $\V{Y|X}$ and $\V{Y|X=x}$.

\begin{solution}
$\V{Y|X}$ is a random variable, but $\V{Y|X=x}$ is a constant.
\end{solution}
\end{exercise}


\begin{exercise}
Show that $\E{(Y - \E{Y |X})^2|X} = \E{Y^2|X} - (\E{Y |X})^2$.

\begin{solution}
Define $g(X) = \E{Y|X}$. Then,
\begin{align}
    \E{(Y - \E{Y |X})^2|X}&= \E{(Y - g(X))^2|X}\\
    &=\E{Y^2-2Yg(X)+g(X)^2|X}\\
    &=\E{Y^2|X}-2\E{Yg(X)|X}+\E{g(X)^2|X}\\
    &=\E{Y^2|X}-2g(X)\E{Y|X}+g(X)^2\\
    &=\E{Y^2|X}-2g(X)^2+g(X)^2\\
    &=\E{Y^2|X}-(\E{Y |X})^2
\end{align}
\end{solution}
\end{exercise}


\begin{exercise}
Let $X\sim \mathcal{N}(\mu,\sigma^2)$ and $W|X \sim \mathcal{N}(0,X^2)$. Find $\V{W}.$
\begin{solution}
Using Eve's Law we have
\begin{align}
    \V{W} = \V{\E{W|X}}+\E{\V{W|X}} =\V{0} + \E{X^2} = 0+  \mu^2 + \sigma^2 = \mu^2 + \sigma^2.
\end{align}

\end{solution}
\end{exercise}


% \opt{check}{\Closesolutionfile{hint}
% \Closesolutionfile{ans}
% % \begin{Hint}{1.2}
			Note the difference between mean and median. This question sheds light on the link between our informal daily languages and formal mathematical concepts.
		
\end{Hint}
\begin{Hint}{1.5}
			For the median, use the fact that the Cauchy density function is symmetric about $0$.
		
\end{Hint}
\begin{Hint}{1.6}
			Given any random variable $X$ whose distribution is symmetric about some point $\mu$, you can construct a random variable $Y$ that is symmetric about 0. What can you say about $E(Y^3)$ and $E((-Y)^3)$?
		
\end{Hint}
\begin{Hint}{1.7}
			Check from the definition that a random variable $X$ has zero skewness if $E(X) = E(X^3) = 0$. Construct a random variable satisfying this property. The easiest option is to consider a discrete random variable with 3 values in its support.
		
\end{Hint}
\begin{Hint}{1.9}
			Note that variances (if exist) are always non-negative.
%			\textbf{Intuition:} ${\displaystyle \operatorname {E} (X)=\operatorname {P} (X<a)\cdot \operatorname {E} (X|X<a)+\operatorname {P} (X\geq a)\cdot \operatorname {E} (X|X\geq a)}$ where ${\displaystyle \operatorname {E} (X|X<a)}$  is larger than 0 as r.v. ${\displaystyle X}$ is non-negative and ${\displaystyle \operatorname {E} (X|X\geq a)}$  is larger than ${\displaystyle a}$ because the conditional expectation only takes into account of values larger than ${\displaystyle a}$ which r.v. ${\displaystyle X}$ can take. Hence intuitively ${\displaystyle \operatorname {E} (X)\geq \operatorname {P} (X\geq a)\cdot \operatorname {E} (X|X\geq a)\geq a\cdot \operatorname {P} (X\geq a)}$${\displaystyle \operatorname {E} (X)\geq \operatorname {P} (X\geq a)\cdot \operatorname {E} (X|X\geq a)\geq a\cdot \operatorname {P} (X\geq a)}$, which directly leads to ${\displaystyle \operatorname {P} (X\geq a)\leq {\frac {\operatorname {E} (X)}{a}}}$.
		
\end{Hint}
\begin{Hint}{1.10}
		\begin{enumerate}[i.]
			\item Use the fundamental bridge. Note that
			\begin{equation*}
				\begin{array}{cl}
						P(|X-\mu|\geq \epsilon) &= E(1_{\{\left( |X-\mu|\geq \epsilon\right) \}})= E(1_{\{  \frac{|X-\mu|}{\epsilon}\geq 1  \}})
				\end{array}
			\end{equation*}
		\item Show that $1_{\{ \frac{|X-\mu|}{\epsilon}\geq 1  \}}\leq \left( \frac{|X-\mu|}{\epsilon}\right)^2 $.
		\item The above two imply that $P(|X-\mu|\geq \epsilon)\leq E\left( \frac{|X-\mu|}{\epsilon}\right)^2$.
		\end{enumerate}
	
\end{Hint}
\begin{Hint}{1.12}
			Use the result from Ex \ref{ex:chap06:05}.
		
\end{Hint}
\begin{Hint}{1.13}
			First, derive the identity $\sum_{i = 1}^n (X_i - \mu)^2 = \sum_{i = 1}^n (X_i - \bar{X}_n)^2 + n (\bar{X}_n - \mu)^2$.
		
\end{Hint}
\begin{Hint}{1.14}
		Try to make use the fact that sample average of i.i.d. data goes to the expectation by decomposing $S^2$ as the sum of components with sample averages.  $$S_n^2 = \frac{n}{n - 1}\frac{1}{n} \sum_{i = 1}^n (X_i - \mu)^2 - \frac{n}{n - 1} (\bar{X}_n - \mu)^2.$$
	
\end{Hint}
\begin{Hint}{1.15}
			If you were to throw a fair coin a large number of times, what is the proportion of heads you would expect?
		
\end{Hint}
\begin{Hint}{1.16}
			FUse that if $X \sim N(\mu, \sigma^2)$, the MGF of $X$ is given by $M_X(t) = e^{\mu t} e^{\frac{1}{2} \sigma^2 t^2}$.
		
\end{Hint}
\begin{Hint}{1.18}
			Recall the formula for geometric series: for $|\rho| < 1$, $\sum_{k = 0}^{\infty} \rho^k = \frac{1}{1 - \rho}$.
		
\end{Hint}
\begin{Hint}{1.19}
			For $X \sim N(\mu, \sigma^2)$, the MGF of $X$ is given by $M_X(t) = e^{\mu t} e^{\frac{1}{2} \sigma^2 t^2}$. Now take derivatives.
		
\end{Hint}
\begin{Hint}{1.20}
			For $X \sim N(\mu, \sigma^2)$, the MGF of $X$ is given by $M_X(t) = e^{\mu t} e^{\frac{1}{2} \sigma^2 t^2}$. Now take derivatives.
		
\end{Hint}
\begin{Hint}{1.21}
		MGFs determines distributions. Show the MGF of the sum can not be written in the form of the Expo MGF.
	
\end{Hint}
\begin{Hint}{1.22}
		MGF!
	
\end{Hint}
\begin{Hint}{1.23}
		MGF!
	
\end{Hint}
\begin{Hint}{1.24}
		The sum of independent Gaussian is Gaussian, use the fact that $X_1+X_2\sim N(\mu_1+\mu_2, \sigma_1^2+\sigma_2^2)$ when $X_1,X_2$ are independent and $X_i\sim N(\mu_i, \sigma_i^2), i=1,2$.
	
\end{Hint}
\begin{Hint}{1.28}
		MGF!
	
\end{Hint}

% % \begin{Solution}{1.1}
			Let $X = 10^{100} B$, where $B \sim \text{Bern}(10^{-10})$. The mean $\mu$ of $X$ is $10^{100} \cdot 10^{-10} = 10^{90}$, which is very large. In contrast, the median is 0, which is closer to the value $X$ generally takes.
		
\end{Solution}
\begin{Solution}{1.2}
			The first sentence uses the ``Median'', and the ``average level'' refers to the ``Mean''.  The second sentence compares the ``median'' with the ``mean''.
		
\end{Solution}
\begin{Solution}{1.3}
~
			\begin{enumerate}
				\item Let $X$ be a random variable. We want to show that the value of $c$ that minimizes $E(X - c)^2$ is $c = \mu$, where $\mu$ denotes the mean of $X$. We have
				\begin{align*}
					E(X - c)^2 & = E((X - \mu) + (\mu - c))^2 \\
					& = E(X - \mu)^2 + 2 E((X - \mu)(\mu - c)) + E(\mu - c)^2 \\
					& = E(X - \mu)^2 + (\mu - c)^2
				\end{align*}
				It is easily seen that $E(X - c)^2$ is minimal for $c = \mu$.
				\item Let $X$ be a random variable. We want to show that the value of $a$ that minimizes $E|X - a|$ is $a = m$, where $m$ denotes the median of $X$. We want to evaluate $E|X - a|$ for $a \neq m$.
					
				Assume $m < a$. If $X \leq m$, then
				\begin{equation*}
					|X - a| - |X - m| = a - X - (m - X) = a - m.
				\end{equation*}
				If $X > m$, then
				\begin{equation*}
					|X - a| - |X - m| = X - a - (X - m) = m - a.
				\end{equation*}
				Now let $Y = |X - a| - |X - m|$ and let $I = 1$ if $X \leq m$ and $I = 0$ if $X > m$. Then
				\begin{align*}
					E(Y) & = E(YI) + E(Y(1 - I)) \\
					& \geq (a - m) E(I)	+ (m - a) E(1 - I) \\
					& = (a - m) \P{X \leq m} + (m - a) \P{X > m} \\
					& = (a - m) \P{X \leq m} - (a - m) (1 - \P{X \leq m}) \\
					& = (a - m) (2 \P{X \leq m} - 1).
				\end{align*}
				By the definition of a median, we have $2 \P{X \leq m} - 1 \geq 0$. Hence, $E(Y) \geq 0$, which implies $E(|X - m|) \leq E(|X - a|)$. Hence for all $E(|X - m|) \leq E(|X - a|)$ for all $m < a$. Repeat similar steps for $m > a$ and conclude $E|X - a|$ is minimal for $a = m$.
				\item Let $X \sim \text{Bern}(0.25)$. Then the mean of $X$ is $\mu = 0.25$, while the median of $X$ is $m = 0$. Note that $E(X - \mu)^2 = V(X) = 0.25(1 - 0.25) = 0.1875$ and $E(X - m)^2 = E(X)^2 = 0.25$; hence $E(X - \mu)^2 \leq E(X - m)^2$ as expected. Moreover, using LOTUS, $E|X - \mu| = |0 - 0.25|(1 - 0.25) + |1 - 0.25|0.25 = 0.375$ and $E|X - m| = E|X| = E(X) = 0.25$; thus, $E|X - m| \leq E(X - \mu)^2$, as expected.
			\end{enumerate}
		
\end{Solution}
\begin{Solution}{1.5}
			Let $X$ follow a standard Cauchy distribution. The PDF of $X$ is given by $f(x) = \frac{1}{\pi (1 + x^2)}$. Note that $f'(x) = -\frac{2x}{\pi (1 + x^2)}$; hence $f'(x) = 0 \iff x = 0$. It follows that $f$ has a maximum at $x = 0$. (Formally, you have to check $f''(0) < 0$, too.) Since this maximum is unique, the mode of $X$ is $0$. As $f(x) = f(-x) = 0$ for all $x \in \mathbb{R}$, the standard Cauchy distribution is symmetric about $0$. Therefore, $P(X \leq 0) = \int_{-\infty}^0 f_X(x) \mathrm{d}x = \int_{-\infty}^0 f_X(-x) \mathrm{d}x = \int_0^{\infty} f_X(y) \mathrm{d}y = \P{X \geq 0}$. As $\P{X \leq 0} + \P{X \geq 0} = 1$ it follows that $\P{X \leq 0} = \frac{1}{2}$. Hence, by definition, the median of the Cauchy distribution is $0$.
		
\end{Solution}
\begin{Solution}{1.6}
			Let $X$ be a random variable whose distribution is symmetric about its mean $\mu$. Then $Y = X - \mu$ is symmetric about 0. Due to symmetry, $Y$ and $-Y$ have the same distribution. That implies $E(Y^3) = E((-Y)^3)$. This in turn implies $E(Y^3) = 0$. It follows that $\text{Skew}(X) = E\left(\frac{X - \mu}{\sigma}\right)^3 = \frac{1}{\sigma^3} E(Y^3) = 0$.
		
\end{Solution}
\begin{Solution}{1.7}
			There are infinitely many possible asymmetric distributions with zero skewness. Zero skewness means that overall, the tails on both sides of the mean balance out. This occurs, for example, when one tail is ``long" but the other tail is ``fat". An easy example of an asymmetric distribution with zero skewness is obtained by considering a discrete random variable with 3 values in its support. Check from the definition that a random variable $X$ has zero skewness if $E(X) = E(X^3) = 0$. One random variable satisfying this property is the random variable $X$ with $\P{X = -3} = 0.1$, $\P{X = -1} = 0.5$ and $\P{X = 2} = 0.4$. The distribution of $X$ is asymmetric by construction. Verify yourself that $E(X) = E(X^3) = 0$.
		
\end{Solution}
\begin{Solution}{1.8}
			The $r$th central moment is given by
			\begin{align*}
				\mu_r & = \int_a^b \left[x - E(X)\right]^r \cdot f_X(x) \mathrm{d}x = \frac{1}{b - a} \int_a^b \left[x - \frac{b - a}{2}\right]^r \mathrm{d}x = \frac{1}{(b - a) 2^r} \int_a^b \left[2x - (a + b)\right]^r \mathrm{d}x \\
				&= \frac{1}{(b - a) 2^r} \left[\frac{(2x - (a + b))^{r + 1}}{2(r + 1)}\right]_a^b = \frac{1}{(b - a) 2^r} \cdot \frac{(b - a)^{r + 1} - (-1)^{r + 1} (b - a)^{r + 1}}{2(r + 1)}
			\end{align*}
			which is zero when $r$ is odd.
		
\end{Solution}
\begin{Solution}{1.9}
			Correct. Recall that the variance of a random variable $X$ is defined as $V(X) = E(X - E(X))^2$. Because $(X - E(X))^2$ is strictly non-negative, $V(X)$ can only be zero if $(X - E(X))^2$ is always zero (or with probability one). $(X - E(X))^2$ is always zero if and only if $X = E(X)$ with probability one. If $X = E(X)$, $X$ always has the same value, i.e. is constant with probability one. Hence, if a random variable is of zero variance, then it is a constant with probability one.
		
\end{Solution}
\begin{Solution}{1.10}
			\begin{enumerate}[i.]
			\item Use the fundamental bridge. Note that
			\begin{equation*}
				\begin{array}{cl}
					P(|X-\mu|\geq \epsilon) &= E(1_{\{\left( |X-\mu|\geq \epsilon\right) \}})= E(1_{\{  \frac{|X-\mu|}{\epsilon}\geq 1  \}})
				\end{array}
			\end{equation*}
			\item Show that $1_{\{ \frac{|X-\mu|}{\epsilon}\geq 1  \}}\leq \left( \frac{|X-\mu|}{\epsilon}\right)^2 $. For any $s\in S$, if $1_{\{ \frac{|X-\mu|}{\epsilon}\geq 1  \}}(s)=0$, then we know by the non-negativity of the square function $\left( \frac{|X-\mu|}{\epsilon}\right)^2(s)\geq 0=1_{\{ \frac{|X-\mu|}{\epsilon}\geq 1  \}}(s)$;  if $1_{\{ \frac{|X-\mu|}{\epsilon}\geq 1  \}}(s)=1$, then we know by the definition of the indicator function that $\frac{|X-\mu|}{\epsilon}(s)\geq 1 =1_{\{ \frac{|X-\mu|}{\epsilon}\geq 1  \}}(s)$. Therefore, for all outcomes $s\in S$, $1_{\{ \frac{|X-\mu|}{\epsilon}\geq 1  \}}(s)\leq \left( \frac{|X-\mu|}{\epsilon}\right)^2(s)$, and thus $$P\left\{1_{\{ \frac{|X-\mu|}{\epsilon}\geq 1  \}}\leq \left( \frac{|X-\mu|}{\epsilon}\right)^2\right\}=P(S)=1. $$
			\item The above two imply that $P(|X-\mu|\geq \epsilon)\leq E\left( \frac{|X-\mu|}{\epsilon}\right)^2$.
		\end{enumerate}
	
\end{Solution}
\begin{Solution}{1.11}
		We want to show that for some constant $c$ we have that for any $\varepsilon>0$ $\P{|\frac{1}{n}\sum_{i=1}^n(X_{i}-E(X_i))^2-c|>\varepsilon}\rightarrow 0$. Denote $E(X_i)=\mu$ and $Y_n=\frac{1}{n}\sum_{i=1}^n(X_{i}-\mu)^2$, then using the result from the previous exercise we obtain $\P{|Y_n-E(Y_n)|\geq\varepsilon}\leq \frac{Var(Y_n)}{\varepsilon^2}$. By independence of the $X_i$ $Var(Y_n)=Var(Y_n=\frac{1}{n}\sum_{i=1}^n(X_{i}-\mu)^2)=\frac{1}{n}Var((X_i-\mu)^2)\rightarrow 0$, because of $Var((X_i-\mu)^2)$ is finite by the finite fourth moment of $X_i$. We conclude that $\frac{1}{n}\sum_{i=1}^n(X_{i}-E(X_i))^2$ converges to $E(\frac{1}{n}\sum_{i=1}^n(X_{i}-E(X_i))^2)=Var(X_i)$.
	
\end{Solution}
\begin{Solution}{1.12}
			Let $X_1, \ldots, X_n$ be i.i.d. random variables with mean $\mu$ and variance $\sigma^2$. The sample mean is given by $\bar{X}_n = \frac{1}{n} \sum_{i = 1}^n X_i$. The variance of the sample mean is given by
			\begin{align*}
				V(X_n^2) & = V\left(\frac{1}{n} \sum_{i = 1}^n X_i\right) = \frac{1}{n^2} \sum_{i = 1}^n V(X_i) = \frac{1}{n^2} \cdot n \sigma^2 = \frac{\sigma}{n}.
			\end{align*}	
			It follows that $V(X_n^2) \to 0$ as $n \to \infty$. Now invoke the result of Ex \ref{ex:chap06:05} to conclude that the sample mean converges to a constant with probability one.
		
\end{Solution}
\begin{Solution}{1.13}
			Let $X_1, \ldots, X_n$ be i.i.d. random variables with mean $\mu$ and variance $\sigma^2$. The sample mean is given by $\bar{X}_n = \frac{1}{n} \sum_{i = 1}^n X_i$. The sample variance is given by $S_n^2 = \frac{1}{n - 1} \sum_{i = 1}^n (X_i - \bar{X}_n)^2$. First, we construct the identity
			\begin{align*}
				\sum_{i = 1}^n (X_i - \mu)^2 & = \sum_{i = 1}^n ((X_i - \bar{X}_n) + (\bar{X}_n - \mu))^2 \\
				& = \sum_{i = 1}^n (X_i - \bar{X}_n)^2 + 2 (\bar{X}_n - \mu) \sum_{i = 1}^n (X_i - \bar{X}_n) + \sum_{i = 1}^n (\bar{X}_n - \mu)^2 \\
				& = \sum_{i = 1}^n (X_i - \bar{X}_n)^2 + n (\bar{X}_n - \mu)^2
			\end{align*}
			(Here, we used that that $\sum_{i = 1}^n (X_i - \bar{X}_n) = \left(\sum_{i = 1}^n X_i\right) - n \bar{X}_n = n \bar{X}_n - n \bar{X}_n = 0$.) Rewriting this identity yields
			\begin{equation*}
				\sum_{i = 1}^n (X_i - \bar{X}_n)^2 = \sum_{i = 1}^n (X_i - \mu)^2 - n (\bar{X}_n - \mu)^2.
			\end{equation*}
			Note that $E(\sum_{i = 1}^n (X_i - \mu)^2) = n \sigma^2$ and $E(n (\bar{X}_n - \mu)^2) = n V(\bar{X}_n) = n \cdot \frac{\sigma}{n} = \sigma$. Hence,
			\begin{align*}
				E(S_n^2) & = E\left(\frac{1}{n - 1} \sum_{i = 1}^n (X_i - \bar{X}_n)^2\right) \\
				& = \frac{1}{n + 1} \left(E\left(\sum_{i = 1}^n (X_i - \mu)^2\right) - E\left(n\left(\bar{X}_n - \mu\right)^2\right)\right) = \frac{1}{n - 1} (n \sigma^2 - \sigma^2) = \sigma^2.
			\end{align*}
		
\end{Solution}
\begin{Solution}{1.14}
		By the result of Ex \ref{ex:chap06:04}, we have $\sum_{i = 1}^n (X_i - \bar{X}_n)^2 = \sum_{i = 1}^n (X_i - \mu)^2 - n (\bar{X}_n - \mu)^2$. As such, $S_n^2 = \frac{n}{n - 1}\frac{1}{n} \sum_{i = 1}^n (X_i - \mu)^2 - \frac{n}{n - 1} (\bar{X}_n - \mu)^2$. The latter term converges to $0$ as $\bar{X}_n$ converges to $\mu$. Hence, as $n \to \infty$, the term $\frac{1}{n} \sum_{i = 1}^n (X_i - \mu)^2$ which is the sample average of i.i.d. $Z_i=(X_i - \mu)^2$ converges to the expectation of $Z_i$, which in turn is the variance of $X_i$.
		
\end{Solution}
\begin{Solution}{1.15}
		If you were to throw a fair coin a large number of times, you would expect the proportion of heads to converge to 0.5 (see Ex \ref{ex:chap06:05}). So to verify whether the coin is fair, you could throw it a large number of time and assess whether the sample proportion of heads approximates 0.5. To illustrate, run the following code:
\begin{minted}{python}
import numpy as np
import matplotlib.pyplot as plt

nSeq = 5
nTrials = 10 ** 3
p = 0.5

for j in range(nSeq):
    x = np.zeros(nTrials + 1, float)
    Mean_list = []
    for i in range(nTrials):
        x[i] = np.random.binomial(1, p)
        xbar = np.mean(x[:i+1])
        Mean_list.append(xbar)

    plt.plot(range(nTrials), Mean_list, label='Sample_' + str(j + 1))
plt.ylabel('Estimated proportion of heads')
plt.xlabel('Trials')
plt.legend(loc=0, ncol=3, fontsize='small')
plt.show()
\end{minted}
		Indeed, after throwing a fair coin 1000 times, the sample proportion of heads is close to 0.5. (Check yourself what happens if $p \neq 0.5$!)
		
\end{Solution}
\begin{Solution}{1.16}
			For $X \sim N(\mu, \sigma^2)$, the MGF of $X$ is given by $M_X(t) = e^{\mu t} e^{\frac{1}{2} \sigma^2 t^2}$. To verify this, first write $X = \mu + \sigma Z$ for $Z \sim N(0,1)$, and calculate
			\begin{align*}
				M_Z(t) & = E(e^{tZ}) = \int_{-\infty}^{\infty} e^{tz} \cdot \frac{1}{\sqrt{2 \pi}} e^{-\frac{1}{2} z^2} \mathrm{d}z = \int_{-\infty}^{\infty} e^{tz} \cdot \frac{1}{\sqrt{2 \pi}} e^{-\frac{1}{2} z^2} \mathrm{d}z \\
				& = e^{\frac{1}{2} t^2} \int_{-\infty}^{\infty} \frac{1}{\sqrt{2 \pi}} e^{-\frac{1}{2} (z - t)^2} \mathrm{d}z = e^{\frac{1}{2} t^2}
			\end{align*}
			(The last step follows from recognizing $\frac{1}{\sqrt{2 \pi}} e^{-\frac{1}{2} (z - t)^2}$ as a PDF.) It then follows that
			\begin{align*}
				M_X(t) & = E\left(e^{tX}\right) = E\left(e^{t(\mu + \sigma Z)}\right) = e^{\mu t} E\left(e^{t \sigma Z}\right) = e^{\mu t} M_Z(\sigma t) = e^{\mu t} e^{\frac{1}{2} \sigma^2 t^2}.
			\end{align*}
			Now let $X_1\sim N(\mu_1, \sigma_1^2)$ and $X_2\sim N(\mu_2, \sigma_2^2)$. We have
			\begin{align*}
				M_{X_1 + X_2}(t) & = E\left(e^{t(X_1 + X_2)}\right) = E\left(e^{t X_1}\right) E\left(e^{t X_2}\right) = e^{\mu_1 t} e^{\frac{1}{2} \sigma_1^2 t^2} e^{\mu_2 t} e^{\frac{1}{2} \sigma_2^2 t^2} = e^{(\mu_1 + \mu_2) t} e^{\frac{1}{2} (\sigma_1^2 + \sigma_2^2) t^2}
			\end{align*}
			This is the MGF of the $N(\mu_1 + \mu_2, \sigma_1^2 + \sigma_2^2)$ distribution. It follows that $X_1 + X_2 \sim N(\mu_1 + \mu_2, \sigma_1^2 + \sigma_2^2)$.
		
\end{Solution}
\begin{Solution}{1.17}
			Let $X \sim \text{Expo}(\lambda)$ for some $\lambda > 0$. Let $Y = \lambda X$ for some $\lambda > 0$. The MGF of $X$ is given by
			\begin{align*}
				M_X(t) = E(e^{tX}) = \int_0^{\infty} e^{tx} \lambda e^{-\lambda x} \mathrm{d}x = \int_0^{\infty} \lambda e^{-x(\lambda - t)} \mathrm{d}x = \left[- \frac{\lambda}{\lambda - t} e^{-x(\lambda - t)}\right]_0^{\infty} = \frac{\lambda}{\lambda - t}, \quad t < \lambda.
			\end{align*}
			It follows that the MGF of $Y$ is given by
			\begin{align*}
				M_Y(t) = E(e^{tY}) = E(e^{\lambda t X}) = M_X(\lambda t) =  \frac{\lambda}{\lambda - \lambda t} = \frac{1}{1 - t}, \quad t < 1.
			\end{align*}
			This is the MGF of the $\text{Expo}(1)$ distribution. Hence, $Y \sim \text{Expo}(1)$.
		
\end{Solution}
\begin{Solution}{1.18}
			Let $X$ have the probability distribution $f(x) = e \left(\frac{1}{3}\right)^x$ for $x = 1, 2, \ldots$. Using LOTUS, the MGF is given by
		\begin{align*}
			M_X(t) = E(e^{tX}) = \sum_{x=1}^{\infty} e^{tx} f(x) = \sum_{x=1}^{\infty} e^{tx} \cdot 2 \left(\frac{1}{3}\right)^x = \sum_{x=0}^{\infty} e^{t(x+1)} \cdot 2 \left(\frac{1}{3}\right)^{x+1} = \frac{2 e^t}{3} \sum_{x=0}^{\infty} \left(\frac{e^t}{3}\right)^x = \frac{2 \left(\frac{e^t}{3}\right)}{1 - \left(\frac{e^t}{3}\right)} = \frac{2e^t}{3 - e^t},
		\end{align*}
		for $|t|<1$. Taking derivatives, we obtain
		\begin{align*}
			M_X'(t) & = \frac{(3 - e^t) 2e^t - 2e^t (-e^t)}{(3 - e^t)^2} = \frac{6e^t}{(3 - e^t)^2} \\
			M_X''(t) & = \frac{(3 - e^t) \cdot 6e^t - 6e^t \cdot 2(3 - e^t)(-e^t}{(3 - e^t)^4}.
		\end{align*}
		It follows that $E(X) = M_X'(0) = \frac{6}{4} = \frac{3}{2}$ and $E(X^2) = \frac{24 - 12 \cdot 2 \cdot -1}{16} = 3$. Therefore $V(X) = E(X^2) - E(X)^2 = 3 - \left(\frac{6}{4}\right)^2 = 3 - \frac{9}{4} = \frac{3}{4}$.
		
\end{Solution}
\begin{Solution}{1.19}
			Let $X \sim N(\mu, \sigma^2)$. The MGF of $X$ is given by $M_X(t) = e^{\mu t} e^{\frac{1}{2} \sigma^2 t^2}$. As such,
			\begin{align*}
				M_X'(t) & = (\mu + \sigma^2 t) e^{\mu t} e^{\frac{1}{2} \sigma^2 t^2} = (\mu + \sigma^2 t) M_X(t) \\
				M_X''(t) & = (\mu + \sigma^2 t)^2 M_X(t) + \sigma^2 M_X(t)
			\end{align*}
			We obtain $E(X) = M_X'(0) = \mu M_X(0) = \mu \cdot 1 = \mu$ and $E(X^2) = M_X''(0) = \mu^2 M_X(0) + \sigma^2 M_X(0) = \mu^2 + \sigma^2$. It follows that $V(X) = E(X^2) - E(X)^2 = (\mu^2 + \sigma^2) - \mu^2 = \sigma^2$.
		
\end{Solution}
\begin{Solution}{1.20}
			Let $X \sim N(\mu, \sigma^2)$. The MGF of $X$ is given by $M_X(t) = e^{\mu t} e^{\frac{1}{2} \sigma^2 t^2}$. As such,
			\begin{align*}
				M_X'(t) & = (\mu + \sigma^2 t) e^{\mu t} e^{\frac{1}{2} \sigma^2 t^2} = (\mu + \sigma^2 t) M_X(t) \\
				M_X''(t) & = (\mu + \sigma^2 t)^2 M_X(t) + \sigma^2 M_X(t) \\
				M_X^{(3)}(t) & = (\mu + \sigma^2 t)^3 M_X(t) + 3 \sigma^2 (\mu + \sigma^2 t) M_X(t) \\
				M_X^{(4)}(t) & = (\mu + \sigma^2 t)^4 M_X(t) + 3 \sigma^2 (\mu + \sigma^2 t)^2 M_X(t) + 3 \sigma^2 (\mu + \sigma^2 t)^2 M_X(t) + 3 \sigma^4 M_X(t)
			\end{align*}	
			We obtain $E(X) = M_X'(0) = \mu$, $E(X^2) = M_X''(0) = \mu^2 + \sigma^2$, $E(X^3) = 3 \mu \sigma^2 + \mu^3$ and $E(X^4) = \mu^4 + 6 \mu^2 \sigma^2 + 3 \sigma^4$.
			
			The skewness of $X$ is given by
			\begin{align*}
				\text{Skew}(X) = E\left(\frac{X - \mu}{\sigma}\right)^3 = \frac{E\left(X^3 - 3X^2 \mu + 3X \mu^2 - \mu^3\right)}{\sigma^3}.
			\end{align*}	
			Plugging in the obtained values for $E(X)$, $E(X^2)$ and $E(X^3)$ yields $\text{Skew}(X) = 0$.
			
			The kurtosis of $X$ is given by
			\begin{align*}
				\text{Kurt}(X) = E\left(\frac{X - \mu}{\sigma}\right)^4 = \frac{E\left(X^4 - 4 \mu X^3 + 6 \mu^2 X^2 - 3 \mu^4\right)}{\sigma^4}.
			\end{align*}	
			Plugging in the obtained values for $E(X)$, $E(X^2)$ and $E(X^3)$ yields $\text{Kurt}(X) = 3$.
		
\end{Solution}
\begin{Solution}{1.21}
		\begin{enumerate}
			\item   MGF of a Expo($\lambda$)-distributed $X$: $$M_X(t)=Ee^{tX}=\int_{0}^{\infty} e^{tx}e^{-\lambda x} dx=  \frac{1}{\lambda -t},$$ {which is finite for, e.g., $t\in (-\lambda/2, \lambda/2)$ (so the MGF is well-defined)}.
			\item {MGF of $Y_1+Y_2$} with $Y_i\sim i.i.d.\text{Expo}(1)$:
			\begin{align*}
				M_{Y_1+Y_2}(t) = M_{Y_1}(t)M_{Y_2}(t)=\frac{1}{(1 -t)^2}.
			\end{align*}
			{which is finite for, e.g., $t\in (-1/2, 1/2)$ (so the MGF is well-defined).}\\~\\
			\item $\frac{1}{(1 -t)^2}$ can not be written in the form of $\frac{1}{\lambda -t}$ for any $\lambda$'s. We prove by contradiction, suppose they are equal for some $\lambda$ then $\left.\frac{1}{(1 -t)^2}\right|_{t=0}=\left.\frac{1}{\lambda -t}\right|_{t=0}$ and thus $\lambda=1$, however, $\left.\frac{1}{(1 -t)^2}\right|_{t=a}\neq \left.\frac{1}{1 -t}\right|_{t=a}$ for any $a\neq 0$.
			\item {The MGF of  $Y_1+Y_2$ is not the MGF of a exponential distribution}, then we know  a sum of two i.i.d. Expo($1$)-distributed r.v.'s is not exponentially distributed as MGF determines distributions (different MGF forms, different distributions).
		\end{enumerate}
	
\end{Solution}
\begin{Solution}{1.22}
		\begin{enumerate}
				\item {MGF of a Pois($\lambda$)-distributed $X$}: $$M_X(t)=Ee^{tX} =e^{\lambda (e^t-1)},$$ {which is finite for, e.g., $t\in (-\lambda/2, \lambda/2)$ (so the MGF is well-defined)}.
			\item {MGF of $\sum_{i=1}^{n} Y_i$} with $Y_i\sim \textit{independent  Pois}(\lambda_i)$:
			\begin{align*}
				M_{\sum_{i=1}^{n}Y_i}(t) =\prod_{i=1}^{n} e^{\lambda_i (e^t-1) }=e^{\sum_{i=1}^{n} \lambda_i (e^t-1) }.
			\end{align*}
			{which is finite for, e.g., $t\in (-1/2, 1/2)$ (so the MGF is well-defined) and is the MGF of a Pois($\sum_{i=1}^{n} \lambda_i$)-distributed $X$.} \\~\\
			\item The MGF of   $\sum_{i=1}^{n} Y_i$ is  the MGF of Pois($\sum_{i=1}^{n} \lambda_i$), then we know a  sum of $n$ independent Pois($\lambda_i$), $i\leq n$ is still Poisson.
		\end{enumerate}
	
\end{Solution}
\begin{Solution}{1.23}
		\begin{enumerate}
					\item MGF of $X \sim N(\mu,\sigma^2)$.
			\begin{align*}
				{ M_X(t)}& =  {\color{red}e^{\mu t + \frac{1}{2}\sigma^2 t^2}},
			\end{align*}  {which is finite for $t\in \mathbb{R}$, so the MGF,  $M_X$, is well defined.}
			\item MGF of $\sum_{i=1}^{n} X_i$ with $X_i\sim \textit{independent }N(\mu_i,\sigma^2_i)$, $i\leq n$:
			\begin{align*}
				M_{\sum_{i=1}^{n}X_i}(t) =\prod_{i=1}^{n} e^{\mu_i t + \frac{1}{2}\sigma_i^2 t^2}=e^{\left(\sum_{i=1}^{n} \mu_i\right) t + \frac{1}{2}\left(\sum_{i=1}^{n}\sigma_i^2\right) t^2 }.
			\end{align*}
			which is finite for, e.g., $t\in (-1, 1)$ (so the MGF is well-defined) and is the MGF of a $N\left(\sum_{i=1}^{n}\mu_i, \sum_{i=1}^{n}\sigma_i^2 \right)$-distributed r.v. \\~\\
			\item The MGF of   $\sum_{i=1}^{n} X_i$ is  the MGF of $N\left(\sum_{i=1}^{n}\mu_i, \frac{1}{2}\left(\sum_{i=1}^{n}\sigma_i^2\right)\right)$, then we know a  sum of n independent $N(\mu_i,\sigma^2_i)$, $i\leq n$ is still normal.
		\end{enumerate}
	
\end{Solution}
\begin{Solution}{1.24}
		\begin{enumerate}
			\item $N(\mu, \frac{1}{n})$, when $n\rightarrow \infty$, $\bar{X}_n$ goes closer and closer to a constant value $\mu$.
			\item $N(0,1)$.
			\item Use the fact that  $Y=\sqrt{n}(\bar{X}_n-\mu)\sim N(0,1)$ for a given $\mu$. For example, in the case $n=10000$, $\mu=2$, we would need to calculate $P(Y\geq \sqrt{n}(2.01-\mu))=P(Y\geq 100)=\Phi(100)$, which is very small.
		\end{enumerate}
	
\end{Solution}
\begin{Solution}{1.26}
		\begin{enumerate}
			\item
			\begin{enumerate}
			\item {MGF of a Pois(1)-distributed $X$}: $$M_X(t)=Ee^{tX} =e^{(e^t-1)},$$ {which is finite for, e.g., $t\in (-1/2, 1/2)$ (so the MGF is well-defined)}.\\
			\item MGF of $Y=\frac{1}{\sqrt{n}}\sum_{i=1}^n (X_i-1),$
			\begin{align*}
				M_{Y}(t)= \prod_{i=1}^{n}M_{(X_i-1)}(t/\sqrt{n}) =(e^{-t/\sqrt{n}} M_{X_1}(t/\sqrt{n}))^n = {\color{red}e^{n\left(e^{\frac{t}{\sqrt{n}}}-1-\frac{t}{\sqrt{n}}\right)}},
			\end{align*} {which is finite for $t\in \mathbb{R}$, so the MGF,  $M_Y$, is well defined.}
			\item {MGF $M_\xi(t)$ for $\xi \sim N(\mu,\sigma^2)$.}
			\begin{align*}
				{ M_\xi(t)}& =  {\color{red}e^{\mu t + \frac{1}{2}\sigma^2 t^2}},
			\end{align*}  {which is finite for $t\in \mathbb{R}$, so the MGF,  $M_\xi$, is well defined.}
		\end{enumerate}
		\item Note that $\lim\limits_{n\rightarrow \infty }{\color{red}e^{n\left(e^{\frac{t}{\sqrt{n}}}-1-\frac{t}{\sqrt{n}}\right)}}= \lim\limits_{n\rightarrow \infty}{e^{n\left(\sum_{i=0}^{\infty}\left(\frac{t}{\sqrt{n}} \right)^i/i!   -1-\frac{t}{\sqrt{n}}\right)}}=  {\color{red}e^{\frac{1}{2}t^2}}$.
		\item{The MGFs of $\frac{1}{\sqrt{n}}\sum_{i=1}^n (X_i-1)$ and $N(0,1)$ are the same in the limit}, which implies in the limit these two distributions coincide.
	\end{enumerate}~\\
	
\end{Solution}
\begin{Solution}{1.28}
~\\
		\begin{enumerate}		
			\item \begin{enumerate}
				\item $X_i\sim \text{i.i.d. Bern}(p)$, $$M_{X_i}(t)=p(e^t-1)+1, t\in \mathbb{R};$$
				\item $Y=\sum_{i=1}^{n}X_i\sim \text{Bin}(n,p),$
				$$M_{Y}(t)= \prod_{i=1}^{n}M_{X_i}(t) =(M_{X_1}(t))^n = {\color{red}\left(1+p(e^t-1)\right)^n}, t\in \mathbb{R}.$$
				\item[] $Z\sim\text{Pois($\lambda$)}$, $$M_{Z}(t)=\sum_{n=0}^\infty e^{tn}e^{-\lambda} \lambda^n/n!=e^{-\lambda} \sum_{n=0}^\infty  (\lambda e^t)^n/n! =e^{-\lambda}e^{\lambda e^t}    = {\color{red}e^{\lambda (e^t-1) }}, t\in \mathbb{R}.$$
			\end{enumerate}
			\item Note that $\lim\limits_{n\rightarrow \infty, p=\lambda/n}{\color{red}\left(1+p(e^t-1)\right)^n}= \lim\limits_{n\rightarrow \infty}\left(1+ \frac{\lambda(e^t-1)}{n}\right)^n=  {\color{red}e^{\lambda (e^t-1) }}$.
			\item{ The MGFs are the same in the limit}, which implies in the limit these two distributions coincide.
		\end{enumerate}~\\
	
\end{Solution}
\begin{Solution}{1.29}
  	\begin{enumerate}
  		\item 		 The probability of having strictly more than one winner is zero (\textbf{use words to motivate is also okay, here I only show solutions motivated by formulas} Methods are not unique, using indicator functions is also okay.):
  		\begin{align*}
  			\mathbb{P}\left(\left\{ \bigcup_{i=1}^n \left\{ \max_{l\neq i} \{X_l\} =\max_{1\leq j \leq n} \{X_j\} \right\}    \right\}^c\right) \leq \mathbb{P}\left( \bigcup_{i\neq j} \{ X_i =X_j \} \right) \leq \sum_{i\neq j} \mathbb{P}\left( X_i =X_j \right)=0
  		\end{align*}	
  		where the last equation is due to the fact that $X_i-X_j$ is one continuous r.v. and thus $\mathbb{P}\left( \{ X_i =X_j \} \right) = \mathbb{P}\left( \{ X_i-X_j =0\} \right)=0$ (the probability of one continuous r.v. equal to one fixed constant is zero).
  		\item			By symmetry of continuous random variables we know
  		\begin{align*}
  			\mathbb{P}\left(X_{a_1}<X_{a_2}<\cdots < X_{a_j} \right) =1/j!
  		\end{align*}	
  		for arbitrary permutation $(a_1,\cdots, a_j)$ of $(1,\cdots, j)$. Now among all permutations, there are $(j-1)!$ permutations of the format  $(a_1,\cdots, a_{j-1}, j)$: therefore,
  		\begin{align*}
  			\mathbb{P}\left(\bigcup_{(a_1,\cdots, a_{j-1})}\left\{X_{a_1}<X_{a_2}<\cdots < X_{a_{j-1}}<X_j\right\} \right) =1/j
  		\end{align*}
  	\item 			Note that  $X_1+X_2$ follows the same distribution as $1-X_{n-1}+1-X_n$ (they have the same PDF). Therefore, by the LOTUS,
  	\begin{align*}
  		&\mathbb{E} \left(\log\left(\frac{X_1+X_2}{2-(X_{n-1}+X_{n})} \right) \right)=\mathbb{E}\left(\log\left({X_1+X_2} \right) \right)- \mathbb{E}\left(\log\left( {2-(X_{n-1}+X_{n})} \right) \right) \\&= \int_{-\infty}^{+\infty} \log(x)g(x)dx- \int_{-\infty}^{+\infty} \log(x)g(x)dx =0
  	\end{align*}
  \item 			We first derive the MGF  of $X_1$ (for $t\neq 0$):
  \begin{align*}
  	\mathbb{E}[e^{tX_1}] =\int_{0}^{1} e^{tx} dx =  (e^t-1)/t
  \end{align*}	
  for $t=0$, we know $	\mathbb{E}[e^{tX_1}]=	\mathbb{E}[1]=1$	
  Therefore the MGF is well defined. Next, for $t\neq 0$
  \begin{align*}
  	M_{a+b X_1}(t) = e^{at}M_{X_1}(bt)= e^{at}\left(e^{bt}-1\right)/t
  \end{align*}	
  for $t=0$, $	M_{a+b X_1}(t) = 1$.
  \item 						The variance (by independence) is $Var(X^*)=\sum_{i=1}^nVar(X_i)/n = 1/12$.\\~\\
  For $t\neq 0$,
  \begin{align*}
  	&M_{X^*}(t) = M_{\left(\sum_{i=1}^n X_i\right) -\sqrt{n}/2  }(t/\sqrt{n}) =e^{-\sqrt{n}t/2 }\prod_{i=1}^nM_{ X_i}(t/\sqrt{n})\\
  	=& e^{-\sqrt{n}t/2 } \left(\sqrt{n}\left(e^{t/\sqrt{n}} -1 \right)/t\right)^n= \left(e^{-t/(2\sqrt{n}) }\sqrt{n}\left(e^{t/\sqrt{n}} -1 \right)/t\right)^n\\
  	=&  \left(\sqrt{n}\left(e^{t/(2\sqrt{n})} -e^{-t/(2\sqrt{n}) } \right)/t\right)^n =_{\textit{large n}} \left((t+t^3/24/n)/t\right)^n\\ \rightarrow_{n\rightarrow \infty}& e^{t^2/24}
  \end{align*}	
  When $t=0$, $M_{X^*}(t) =1$. 	
  The limiting distribution should be $N(0,1/12)$ by the format of our derived limiting MGF, since the normal distribution has the MGF $e^{\mu t + \sigma^2t^2/2 }$.	
  	\end{enumerate}
  ~\\~\\~\\
  Some simulation codes to help you with this exercise, also you may need some coding skills for the PD course.
  For continuouse random varaible, we can redraw many of its realised values, and then draw a histogram. Histogram can be regarded as a sample-version of the PDF (actually, it is indeed one estimator for the PDF). You can see the more data you use to draw the histogram, the histogram is smoother and closer to its PDF.

  We may use the following codes to draw $S=2,10,100,5000$ realised values from Unif(0,1) and then draw histogram:
  \begin{minted}[]{R}
set.seed(12)
n=1
sigma2=1/12
par(mfrow=c(2,2))
require(ggplot2)
loop.vector <- c(2,10,100,5000)
plot_list = list()
j=0
for(i in loop.vector){
	S=i
	j=j+1
	Unif_draws= runif(n*S, min = 0, max = 1)
	
	plot <- ggplot(data.frame(Unif_draws), aes(Unif_draws)) +
	geom_histogram(aes(y=..density..)) +
	ggtitle("n=",n)
	plot_list[[j]] = plot
}
#library(gridExtra)
require(gridExtra)
grid.arrange(plot_list[[1]], plot_list[[2]],plot_list[[3]],plot_list[[4]], nrow = 2)
  \end{minted}
 The output is the following figure~\\
 \begin{figure}[htbp!]
 	\includegraphics[width=0.7\textwidth]{0}
 \end{figure}  		   ~\\
 and indeed the larger the $S$ the closer the histogram to the PDF of Unif(0,1).\\~\\~\\

 \begin{minipage}{0.45\textwidth}
 	\includegraphics[width=0.9\textwidth]{1n2} \captionof{figure}{n=\textbf{2},S=1000}
 	\includegraphics[width=0.9\textwidth]{1n20} \captionof{figure}{n=\textbf{20},S=1000}
 \end{minipage}
 \begin{minipage}[b]{0.45\textwidth}
 	From the comparison between the histograms of $X^*_n$ (blue) and normal density (red), we can see it indeed gets closer and closer as $n$ increases. \\~\\
 	Here we fix a large $S$ to make sure the histograms are good approximations for the density of $X_n^*$.
 \end{minipage}
\newline
Each line of code can be executed using $\text{ctrl}+\text{Enter}$ in Rstudio.
~\\~\\
This draws 4 values from Unif(0,1)
\begin{minted}[]{R}
runif(4, min = 0, max = 1)
\end{minted}
Here are the codes that you may find how many are the largest number
\begin{minted}[]{R}
rand.a=runif(4, min = 0, max = 1)
sum(rand.a==max(rand.a))
\end{minted}
We may repeat this game many many times and check how many more than 1 winner in S repeated games via the following codes:
\begin{minted}[]{R}
n=4;
S=100;
rand.Srepeats=matrix(runif(n*S, min = 0, max = 1),ncol=S)
colMax<- apply(rand.Srepeats, 2, max)
# this gives you the a True and false matrix, and only true if the max in that column
result=colMax==t(matrix(rep((colMax),n),nrow=S))
# how many more than 1 winner in S repeated games
sum(colSums(result)>1)
\end{minted}

Here we comparing the histogram of X*
(we generate S values drawn from the same distribution as the one of X* and then draw histogram)
with the density function of a normal distribution.
\begin{minted}[]{R}
set.seed(12)
require(ggplot2)
n=4
S=100
sigma2=1/12
# we generate S realised values of X* (simu_Sbar) by drawing from Unif distribution
simu_Sbar= colSums(matrix(runif(n*S, min = 0, max = 1)-1/2,ncol=S))/sqrt(n)

ggplot(data.frame(simu_Sbar), aes(simu_Sbar)) +
geom_histogram(aes(y=..density..)) +
stat_function(fun=function(x)1/sqrt(2*pi*sigma2)*exp(-(x)^2/sigma2/2),
color=rgb(0.6, 0.2, 0.2, 0.35), size=2)
\end{minted}

Here are also Python codes
\begin{minted}{python}
import numpy as np
np.random.seed(10)

# now again this draws 4 values from Unif(0,1)
np.random.uniform(0,1,4)

n=4;
S=100;
rand_Srepeats= np.random.uniform(0,1,n*S).reshape(-1,S)
# how many more than 1 winner in S repeated games
np.sum(np.sum(rand_Srepeats==rand_Srepeats.max(axis=0), axis=0)>1)


# X* distribution analysis
n=4
S=100
sigma2=1/12

rand_Srepeats= np.random.uniform(0,1,n*S).reshape(-1,S) -1/2
simu_Sbar= np.sum(rand_Srepeats.reshape(-1,S),axis=0)/np.sqrt(n)
#  Histogram
import seaborn as sns
sns.distplot(simu_Sbar, hist=True, kde=True,
bins=int(180/5), color = 'darkblue',
hist_kws={'edgecolor':'black'},
kde_kws={'linewidth': 4})

# we add some normal density
import math
ax=sns.distplot(simu_Sbar, hist=True, kde=True,
bins=int(180/5), color = 'darkblue',
hist_kws={'edgecolor':'black'},
kde_kws={'linewidth': 4})
# calculate the pdf
sigma2=1/12
x0, x1 = ax.get_xlim()  # extract the endpoints for the x-axis
x_pdf = np.linspace(x0, x1, 100)

y_pdf = 1/(np.sqrt(2*math.pi*sigma2))*np.exp(-np.power(x_pdf,2)/(2*sigma2))

ax.plot(x_pdf, y_pdf, 'r', lw=2, label='pdf')
ax.legend()
\end{minted}


\end{Solution}

% }

\end{document}


%%% Local Variables:
%%% mode: latex
%%% TeX-master: "study-guide.tex"
%%% End:

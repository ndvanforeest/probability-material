% arara: pdflatex: { shell: yes }
% arara: pythontex: {verbose: yes, rerun: modified }
% arara: pdflatex: { shell: yes }

\documentclass[bh_problems_check]{subfiles}

\opt{check}{
\Opensolutionfile{hint}
\Opensolutionfile{ans}
}

\begin{document}

\chapter{Chapter 1: Exercises and remarks}

Try to link the theoretical concepts from the first chapter with our daily life stories. This makes memorizing these concepts easier.\\~\\
The first chapter formalizes our daily-life uncertainty measurement. For example, when describing how likely it is that it is going to rain tomorrow, you may say something like 50\%, a number between 0 (almost surely not to happen) and 1 (almost sure to happen). \\~\\
Similarly, \textbf{probability} is nothing but a function (with some special properties),
which \textbf{assigns values from 0 to 1 to random events to measure
	uncertainty}. To specify a function, we need to specify three elements:
(1) the domain (a collection of various events); (2) the codomain/image (real numbers between 0 and 1); and (3) how this function
map one element within the domain to one element within the image (naive probability is one way to map the events to numbers, but not the unique way).
The naive definition of probability is the starting point, from which we proceed to the general definition of probability.\\~\\
The non-naive definition of probability is also consistent with our daily experience: the probability of the outcomes being in the whole sample space is one, in the empty set is zero, and the probability of \textbf{disjoint} events should be the sum of the probability values of each of these events. The \textbf{disjoint} request is the key, and you will see that in order to calculate probability values of events, it is sometimes easier to decompose events into \textbf{disjoint} ones first. \\~~\\ 

\section{Sample spaces}
\label{sec:section-1.1}

\begin{exercise} \label{ex:chap01:01} 
	Consider two coin tosses. Denote by $H_i$ and $T_i$ the event that the $i$th coin lands heads and tails, respectively.
	\begin{enumerate}
		\item Determine which of the following sets can serve as one sample space (suppose that you do not necessarily care about both tosses):
		\begin{enumerate}
			\item $\{H_1H_2,H_1T_2,T_1H_2,T_1T_2\}$;
			\item $\{H_1, T_1\}$;
			\item $\{H_2, T_2\}$;
			\item $\{H_1H_2 ~~\textit{or}~~ T_1T_2,  H_1T_2 ~~\textit{or}~~ T_1H_2 \}$;
			\item $\{H_1H_2 ~~\textit{or}~~ T_1T_2,  H_1T_2 , T_1H_2 \}$.
		\end{enumerate}
		\item If one only cares about the outcomes of the second coin, which of the above sets can serve as the sample space? 
		\item If one only cares about whether two coins have the same outcome, which of the above sets can serve as the sample space? 
		\item Calculate the probability of $\{H_1\}, \{H_1H_2 ~~\textit{or}~~ T_1T_2\}, \{ H_1T_2 ~~\textit{or}~~ T_1H_2 \}, \{T_1H_2 \}$
		%		\item In which of the above sets is each outcome equally likely to happen?		
		\item Suppose you work with the sample space $S=\{H_1H_2,H_1T_2,T_1H_2,T_1T_2\}$, can you propose a partition of $S$? 
	\end{enumerate}
	
	\begin{hint} 
		From Wikipedia: \textit{A partition of a set means a grouping of its elements into non-empty subsets, in such a way that every element is included in exactly one subset. Every equivalence relation on a set defines a partition of this set, and every partition defines an equivalence relation.}
	\end{hint}
	
	\begin{solution} 
		\begin{enumerate}
			\item The sets $\{H_1H_2,H_1T_2,T_1H_2,T_1T_2\}$, $\{H_1H_2 ~~\textit{or}~~ T_1T_2, H_1T_2 ~~\textit{or}~~ T_1H_2 \}$ and \\$\{H_1H_2 ~~\textit{or}~~ T_1T_2, H_1T_2, T_1H_2 \}$ are valid sample spaces. The sets $\{H_1, T_1\}$ and $\{H_2, T_2\}$ are not valid sample spaces, as they do not contain all possible outcomes of the experiment.
			\item If one only cares about the outcomes of the second coin, the set $\{H_2, T_2\}$ can serve as the sample space.
			\item If one only cares about whether two coins have the same outcome, the set \\$\{H_1H_2 ~~\textit{or}~~ T_1T_2,  H_1T_2 ~~\textit{or}~~ T_1H_2 \}$ can serve as the sample space.
			\item $\P{H_1} = \frac{1}{2}$, $\P{H_1H_2 ~~\textit{or}~~ T_1T_2} = \frac{1}{2}$ and $\P{H_1T_2 ~~\textit{or}~~ T_1H_2} = \frac{1}{2}$ and $\P{T_1H_2} = \frac{1}{4}$.
			%			\item In the sets $\{H_1H_2,H_1T_2,T_1H_2,T_1T_2\}$, $\{H_1, T_1\}$, $\{H_2, T_2\}$ and $\{H_1H_2 ~~\textit{or}~~ T_1T_2,  H_1T_2 , T_1H_2 \}$, each outcome is equally likely to happen. In the set $\{H_1H_2 ~~\textit{or}~~ T_1T_2,  H_1T_2 ~~\textit{or}~~ T_1H_2 \}$, not every outcome is equally likely to happen.
			\item E.g.,  $\{H_1T_2, H_1H2\}$ and $\{T_1T_2, T_1H2\}$.
		\end{enumerate}
	\end{solution}
\end{exercise}
	
\begin{exercise} \label{ex:chap01:02}
	Provide one possible sample space resulting from two fair coin tosses.  
	\begin{solution}
		There are many possible answers. One correct answer is $\{H_1H_2,H_1T_2,T_1H_2,T_1T_2\}$.
	\end{solution}
\end{exercise}
	
\begin{remark}~
	\begin{enumerate}
		\item Ex \ref{ex:chap01:01} shows the flexibility one has when defining a sample space. Even within the same coin-flipping game, there are many possible specifications of the sample spaces you can choose. In practice, we usually either choose the \textbf{finest} set 1.(a) (\textit{the outcomes in other specifications can always be regarded as events under the specification of 1.(a), e.g., the outcome $H_1$ in 1.(b) sample space under the 1.(a) specification should be the event $\{H_1T_2, H_1H_2\}$})\footnote{It is okay to have different ways of modeling, or different notations, for the same probabilistic story. It is legitimate to regard events in one specification as outcomes (thus be elements of events of \textit{second order}, e.g., the 1.(b)'s outcomes $H_1$ under the specification 1.(a) should be the event $\{H_1T_2, H_1H_2\}$, and these two different ways of modeling, or two different notations both describe the same thing: first coin lands head)  in another specification. These events of second order may again be elements of events of a higher order in another specification. \textbf{As long as we maintain the consistency of the notations within the same specification, namely, events are always a set of outcomes within the same specification (e.g., $H_1$ is one outcome in 1.(b), and it is not a well-defined notation in 1.(a) unless otherwise specified; $\{H_1\}$ is one event in 1.(b), and to describe the same event in 1.(a), we should use the notation $\{H_1H_2, H_1T_2\}$ unless otherwise specified).} We shall avoid such terms as ``the sample space containing all events as elements,'' which leads to contradictions, as the sample space does not contain itself, which is also an event. When specifying the sample space, though quite flexible to model, we shall be careful to only include \textit{disjoint} outcomes to which the resulting sample space does not belong. You may find similar arguments in any textbook for basic set theory, and we skip too detailed discussion here.} or choose one based on the very question you want to explore as long as the specification can distinguish between the results you care about (e.g., it is also natural to choose 1.(d) sample space in subquestion 3, though it only contains two outcomes: both coins having or not having the same side, it distinguishes whether or not these two coins have the same results).
        \item Compare your answers in Ex \ref{ex:chap01:02} with the Ex \ref{ex:chap01:01}. Learn to model your own sample spaces. Sometimes choosing one easier/clearer sample space gives rise to much easier solutions.
	\end{enumerate}
\end{remark}
	 
\section{How to count}
\label{sec:section-1.2}	 
	 
	\begin{exercise}
		Consider a jar with two balls, one labelled $H$ and one marked $T$. We sample balls one at a time with replacement, meaning that each time a ball is chosen, it is returned to the jar.
		\begin{enumerate}
			\item How many ways are there to draw one ball from the jar?
			\item How many ways are there to draw two balls from the jar?
			\item How many ways are there to draw $n$ balls from the jar?
		\end{enumerate}
		\begin{solution}~
			\begin{enumerate}
				\item There are 2 ways to draw one ball from the jar ($H$ and $T$).
				\item By the multiplication rule, there are $2 \cdot 2 = 4$ ways to draw two balls from the jar.
				\item By the multiplication rule, there are $\underbrace{2 \cdot 2 \cdot 2}_{n \: times} = 2^n$ ways to draw $n$ balls from the jar.
			\end{enumerate}
		\end{solution}
	\end{exercise}
	
	
	\begin{exercise}
		Consider a jar with two balls, one labelled $H$ and one marked $T$. We sample balls one at a time without replacement, meaning that each time a ball is chosen, it is left out of the jar.
		\begin{enumerate}
			\item How many ways are there to draw one ball from the jar?
			\item How many ways are there to draw two balls from the jar?
			\item How many ways are there to draw three balls from the jar?
		\end{enumerate}
		\begin{solution}~
			\begin{enumerate}
				\item There are 2 ways to draw one ball from the jar ($H$ and $T$).
				\item There are 2 ways to draw the first ball from the jar. Then there is 1 way to draw the remaining ball from the jar. By the multiplication rule, there are $2 \cdot 1 = 2$ ways to draw two balls.
				\item There are 0 ways to draw three balls from a jar containing 2 balls.
			\end{enumerate}
		\end{solution}
	\end{exercise}
	
	\begin{exercise}
		Consider a jar with $N$ balls, $n_H$ labelled $H$ and the remainder marked $T$. We sample balls one at a time without replacement, meaning that each time a ball is chosen, it is left out of the jar. Assume $n_H \geq 3$.
		\begin{enumerate}
			\item How many ways are there to draw one ball from the jar?
			\item How many ways are there to draw two balls from the jar?
			\item How many ways are there to draw three balls from the jar?
		\end{enumerate}
		\begin{solution}~
			\begin{enumerate}
				\item If $n-n_H=0$ it means that there are no balls marked with $T$, therefore there is one way to draw one ball from the jar.
				
				If $n-n_H\geq 1$ there are two ways to draw one ball from the jar ($H$ and $T$).
				
				\item If $n-n_H=0$ there is one way to draw two balls from the jar.
				
				If $n-n_H\geq 2$ there are two ways to draw the first ball from the jar. Then there are two ways to draw the second ball. By the multiplication rule, there are $2 \cdot 2 = 4$ ways to draw two balls.
				
				If $n-n_H=1$ the same outcomes as when $n-n_H\geq 2$ are possible, with the exception of drawing two balls marked with $T$. Therefore in this case the number of different ways to draw two balls from the jar is three.
				
				\item If $n-n_H=0$ there is one way to draw two balls from the jar.
				
				If $n-n_H=1$ we can draw only balls marked with $H$ or we can draw one ball marked with $T$ and the remaining ones marked with $H$. The ball marked with $T$ can be drawn at three different moments: as the first, second or third ball. Therefore the different ways to draw three balls from the jar is in this case four. 

				If $n-n_H\geq 3$ there are two ways to draw the first ball from the jar. Then there are two ways to draw the second ball. Finally, there are two ways to draw the remaining ball. By the multiplication rule, there are $2 \cdot 2 \cdot 2 = 8$ ways to draw two balls.

				If $n-n_H=2$ the same outcomes as when $n-n_H\geq 3$ are possible with the exception of drawing three balls marked with $T$. Therefore in this case the number of different ways to draw three balls from the jar is seven.
			\end{enumerate}
		\end{solution}
	\end{exercise}
	 
	\begin{exercise}~
		\begin{enumerate}
			\item Suppose you have $n$ distinct cards. How many different ordered sequences of $n$ cards can you make?
			\item How many different ordered sequences of $k$ cards can you make, for $k \leq n$?
			\item How many different \text{un}ordered sequences of $k$ cards can you make, for $k \leq n$? What do you notice?
		\end{enumerate}
		\begin{solution}~
			\begin{enumerate}
				\item $n(n - 1)(n - 2 )\ldots 1 = n!$
				\item $n(n - 1)(n - 2 )\ldots (n - k) = \frac{n!}{(n - k)!}$
				\item Each ordered sequence can be ordered in $k!$ ways, so there are $\frac{n!}{(n - k)!k!}$ unordered sequences. Notice that $\frac{n!}{(n - k)!k!} = \binom{n}{k}$.
			\end{enumerate}	
		\end{solution}
	\end{exercise}

\section{Story proof}
\label{sec:section-1.3}	 
	
	\begin{exercise}\label{ex:chap01:11}
		Give a story proof that
		\begin{equation}
			n \binom{n - 1}{k - 1} = k \binom{n}{k}
		\end{equation}
		for any positive integers $n$ and $k$ with $k \leq n$.
		\begin{solution}
			Consider a group of $n$ people, from which a team of $k$ will be chosen, one of whom will be the team captain. To specify a possibility, we could first choose the team captain and then choose the remaining $k - 1$ team members; this gives the left-hand side. Equivalently, we could first choose the $k$ team members and then choose one of them to be captain; this gives the right-hand side.
		\end{solution}
	\end{exercise}
	
	\begin{exercise}\label{ex:chap01:12}
		Give a story proof that
		\begin{align*}
			\sum_{j = 0}^k \binom{m}{k - j} \binom{n}{j} = \binom{n + m}{k}
		\end{align*}
		for any positive integers $m, n, k$ with $k \leq n + m$.
		\begin{solution}
			Consider a student organization consisting of $m$ juniors and $n$ seniors, from which a committee of size $k$ will be chosen. There are $\binom{m+n}{k}$ possibilities. If there are $j$ juniors in the committee, then there must be $k - j$ seniors in the committee. The right-hand side of the identity sums up the cases for $j$.
		\end{solution}
	\end{exercise}

	\begin{exercise}
		Give a story proof that
		\begin{equation}
			(a + b)^n = \sum_{k=0}^n {n \choose k}a^k b^{n-k}
		\end{equation}
		for any $a,b \in \mathbb{R}$ and $n \in \mathbb{N}$. This result is called the \emph{binomial theorem}.
		\begin{solution}
			Expand out the product
			\begin{equation*}
				(x + y)^n = \underbrace{(x + y)(x + y) \ldots (x + y)}_{n \: factors}.
			\end{equation*}
			The terms of $(x + y)^n$ are obtained by picking either the $x$ or the $y$ from each factor. There are $\binom{n}{k}$ ways to choose exactly $k$ of the $x$'s, and each such choice yields the term $x^k y^{n - k}$. The binomial theorem follows.
		\end{solution}
	\end{exercise}

\section{Probability}
\label{sec:section-1.4}	 
	
	\begin{exercise}
		Prove that the naive probability satisfies the axioms of non-naive probability. 
		\begin{solution}
			Consider a finite sample space $S$. For any event $A$ in $S$, the \emph{naive} probability of $A$ is $\P{A} = \frac{|A|}{|S|}$. Notice that for any two disjoints sets $A$ and $B$, $|A \cup B| = |A| + |B|$. As such we have:
			\begin{enumerate}
				\item $\P{\emptyset} = \frac{|\emptyset|}{|S|} = 0$ and $\P{S} = \frac{|S|}{|S|} = 1$;
				\item If $A_1, A_2, \hdots$ are disjoint events in $S$, then $\P{\bigcup_{i = 1}^{\infty} A_i} = \frac{\left|\bigcup_{i = 1}^{\infty} A_i\right|}{|S|} = \frac{\sum_{i = 1}^{\infty} |A_i|}{|S|} = \sum_{i = 1}^{\infty} \P{A_i}$.
			\end{enumerate}
			Hence, the naive probability satisfies the axioms of non-naive probability. 
		\end{solution}
	\end{exercise}
	
	\begin{exercise}
		Flip a coin $n$ times. Denote $S_i$ the outcome of the $i$th flip, and let $S_i = 1$ if the $i$th flip lands heads and $S_i = 0$ otherwise. All flips are independent from each other and $\P{S_i = 1} = p$.
		\begin{enumerate}
			\item If the first two flips all land heads, what is the value of $S_1 \times S_2$?
			\item Calculate $\P{S_1 \times S_2 = (0,0)}$.
			\item Calculate $\P{\{S_1=1\}\cup \{S_1\times S_2 =(0,0) \}}$.
			\item Calculate $\P{\cup_{i=2}^3 \left\{\cap_{j=1}^{i-1}\{S_j=1 \} \cap \{S_i=0\} \right\}}$.
			\item Suppose we flip the coin an infinite number of times. Calculate 
			$\P{\cup_{i=2}^\infty \left\{\cap_{j=1}^{i-1}\{S_j=1 \} \cap \{S_i=0\} \right\}}$.
		\end{enumerate}
		Verify your results for the special case that $p = 1$. Does your answer make sense?
		\begin{hint}
			To calculate $\P{\cup_{i=1}^\infty \left\{\cap_{j=1}^{i-1}\{S_j=1 \} \cap \{S_i=0\} \right\} }$, you can start with a simpler case, such as $\P{\cup_{i=2}^3 \left\{\cap_{j=1}^{i-1}\{S_j=1 \} \cap \{S_i=0\} \right\} }$. Which events are disjoint?
		\end{hint}
		\begin{solution}~
			\begin{enumerate}
				\item If the first two flips all land heads, then $S_1 \times S_2 = (1,1)$.
				\item $\P{S_1 \times S_2 = (0,0)} = \P{(S_1=0)\P(S_2=0)}=(1-\P{S_1=0})(1-\P{S_2=0})=(1-p)^2$.
				\item $\P{\{S_1=1\}\cup \{S_1\times S_2 =(0,0) \}} = \P{\{S_1=1\} \cup \{S_1 = S_2 =0\}}$\\$= \P{\{S_1=1\}} + \P{\{S_1 = S_2 = 0\}} - \P{\{S_1=1\} \cap \{S_1\times S_2 =(0,0) \}} =\P{\{S_1=1\}} + \P{\{S_1 = S_2 = 0\}}= p + (1 - p)^2$
				\item $\P{\cup_{i=2}^3 \left\{\cap_{j=1}^{i-1}\{S_j=1 \} \cap \{S_i=0\} \right\} } = \P{\{\{S_1 = 1\} \cap \{S_2 = 0\}\} \cup \{\{S_1 = 1\} \cap \{S_2 = 1\} \cap \{S_3 = 0\}\}}$ \\ $= \P{\{S_1 = 1\} \cap \{S_2 = 0\}} + \P{\{S_1 = 1\} \cap \{S_2 = 1\} \cap \{S_3 = 0\}} = p(1 - p) + p^2 (1 - p)$
				\item $\P{\cup_{i=2}^\infty \left\{\cap_{j=1}^{i-1}\{S_j=1 \} \cap \{S_i=0\} \right\}} = \P{\cup_{i=2}^\infty \{S_i=0\} \cap \left\{\cap_{j=1}^{i-1}\{S_j=1 \} \right\}}$ \\ $ = \sum_{i = 2}^{\infty} \P{\{S_i=0\} \cap \left\{\cap_{j=1}^{i-1}\{S_j=1 \} \right\}} = \sum_{i = 2}^{\infty} (1 - p) p^{i - 1} $ \\ $= p^{-1} (1 - p) \sum_{i = 2}^{\infty} p^i = p^{-1} (1 - p) (\sum_{i = 0}^{\infty} p^i - p^0 - p^1) = p^{-1} - p^{-1}(1-p)(1 + p) = p$ for $0<p<1$.
			\end{enumerate}
		\end{solution}
	\end{exercise}
	
	\begin{exercise}
		Prove the following statements \emph{mathematically}:
		\begin{enumerate}
			\item For any events $A$ and $B$, $\P{A \cup B} = \P{A} + \P{B} - \P{A \cap B}$;
			\item For any events $A$ and $B$, $\P{A \cap B^c} = \P{A} - \P{A \cap B}$;
			\item For any events $A$ and $B$, $\P{A \cap B} \geq \P{A} + \P{B} - 1$.
		\end{enumerate}
		\begin{hint}
			For the first part of this exercise, make use of the identities $A \cup (A^c \cap B)$ and $B = (A \cap B) \cup (A^c \cap B)$.  For the third part of this exercise, write $A = (A \cap B^c) \cup (A \cap B)$. Check that these identities are true!
		\end{hint}
		\begin{solution}~
			\begin{enumerate}
				\item Using the hint, $\P{A \cup B} = \P{A \cup (A^c \cap B)} = \P{A} + \P{A^c \cap B} = \P{A} + \P{A^c \cap B} + \P{A \cap B} - \P{A \cap B} = \P{A} + \P{B} - \P{A \cap B}$.
				\item Notice that $A = (A \cap B^c) \cup (A \cap B)$. Hence, $\P{A} = \P{(A \cap B^c) \cup (A \cap B)} = \P{A \cap B^c} + \P{A \cap B}$ as $A \cap B^c$ and $(A \cap B)$ are disjoint. The result follows after rearranging terms.
				\item For any events $A$ and $B$, $\P{A \cup B} \leq 1$. Therefore, using the result of the first part of this exercise, $1 \geq \P{A \cup B} = \P{A} + \P{B} - \P{A \cap B}$. The result follows after rearranging terms.
			\end{enumerate}
		\end{solution}
	\end{exercise}
	
	\begin{exercise}
		The event that ``$A$ or $B$ but not both" will occur can be written as $(A \cap B^c) \cup (A^c \cap B)$. Express the probability of this event in terms of $\P{A}$, $\P{B}$ and $\P{A \cap B}$.
		\begin{hint}
			The use of a Venn diagram may help.
		\end{hint}
		\begin{solution}
			Using a Venn diagram or the previous exercise, it is easy to see that $\P{A \cap B^c} = \P{A} - \P{A \cap B}$ and $\P{A^c \cap B} = \P{B} - \P{A \cap B}$. As the events $(A \cap B^c) \cup (A^c \cap B)$ are disjoint, it follows that $\P{(A \cap B^c) \cup (A^c \cap B)} = \P{A \cap B^c} + \P{A^c \cap B} = \P{A} + \P{B} - 2 \P{A \cap B}$. See also the previous exercise.
		\end{solution}
	\end{exercise}
	
	\begin{exercise}
		A card player is dealt a 13-card hand from a well-shuffled, standard deck of cards. What is the probability that the hand is void in at least one suit?
		\begin{solution}
			Let $S$, $H$, $D$ and $C$ be the events of being void in spades, hearts, diamonds and clubs, respectively. We want to find $\P{S \cup D \cup H \cup C}$. By inclusion-exclusion and symmetry,
			\begin{equation*}
				\P{S \cup D \cup H \cup C} = 4 \P{S} - 6 \P{S \cap H} + 4 \P{S \cap H \cap D} - \P{S \cap H \cap D \cap C}.
			\end{equation*}
			The probability of being void in a specific suit is $\frac{\binom{39}{13}}{\binom{52}{13}}$. The probability of being void in two specific suits is $\frac{\binom{26}{13}}{\binom{52}{13}}$. The probability of being void in three suits is $\frac{1}{\binom{52}{13}}$. The probability of being void in four suits is 0. As such,
			\begin{equation*}
				\P{S \cup D \cup H \cup C} = 4 \cdot \frac{\binom{39}{13}}{\binom{52}{13}} - 6 \cdot \frac{\binom{26}{13}}{\binom{52}{13}} + \frac{4}{\binom{52}{13}} \approx 0.051.
			\end{equation*}
		\end{solution}
	\end{exercise}
	
	\begin{exercise}
		A company has 100 employees. Of them, 45 are proficient in German, 30 in French, 20 in Spanish, six in French and German, one in German and Spanish, five in French and Spanish, and just one employee is proficient in all three languages. What is the probability that a randomly chosen employee is proficient in at least one of these three languages?
		\begin{hint}
			Use the inclusion-exclusion principle.
		\end{hint}
		\begin{solution}
			Let $G$, $F$, $S$ denote the probability of an employee being proficient in German, French and Spanish, respectively. Using the naive definition of probability, we know $\P{G} = 0.45$, $\P{F} = 0.30$, $\P{S} = 0.20$, $\P{G \cap F} = 0.06$, $\P{G \cap S} = 0.01$, $\P{F \cap S} = 0.05$ and $\P{G \cap F \cap S} = 0.01$. Using the inclusion-exclusion principle, we obtain
			\begin{align*}
				P{G \cup F \cup S} & = \P{G} + \P{F} + \P{S} - \P{G \cap F} - \P{G \cap S} - \P{F \cap S} + \P{G \cap F \cap S} \\
				& = 0.45 + 0.30 + 0.20 - 0.06 - 0.01 - 0.05 + 0.01 = 0.84
			\end{align*}
		\end{solution}
	\end{exercise}

% \opt{check}{\Closesolutionfile{hint}
% \Closesolutionfile{ans}
% % \begin{Hint}{1.2}
			Note the difference between mean and median. This question sheds light on the link between our informal daily languages and formal mathematical concepts.
		
\end{Hint}
\begin{Hint}{1.5}
			For the median, use the fact that the Cauchy density function is symmetric about $0$.
		
\end{Hint}
\begin{Hint}{1.6}
			Given any random variable $X$ whose distribution is symmetric about some point $\mu$, you can construct a random variable $Y$ that is symmetric about 0. What can you say about $E(Y^3)$ and $E((-Y)^3)$?
		
\end{Hint}
\begin{Hint}{1.7}
			Check from the definition that a random variable $X$ has zero skewness if $E(X) = E(X^3) = 0$. Construct a random variable satisfying this property. The easiest option is to consider a discrete random variable with 3 values in its support.
		
\end{Hint}
\begin{Hint}{1.9}
			Note that variances (if exist) are always non-negative.
%			\textbf{Intuition:} ${\displaystyle \operatorname {E} (X)=\operatorname {P} (X<a)\cdot \operatorname {E} (X|X<a)+\operatorname {P} (X\geq a)\cdot \operatorname {E} (X|X\geq a)}$ where ${\displaystyle \operatorname {E} (X|X<a)}$  is larger than 0 as r.v. ${\displaystyle X}$ is non-negative and ${\displaystyle \operatorname {E} (X|X\geq a)}$  is larger than ${\displaystyle a}$ because the conditional expectation only takes into account of values larger than ${\displaystyle a}$ which r.v. ${\displaystyle X}$ can take. Hence intuitively ${\displaystyle \operatorname {E} (X)\geq \operatorname {P} (X\geq a)\cdot \operatorname {E} (X|X\geq a)\geq a\cdot \operatorname {P} (X\geq a)}$${\displaystyle \operatorname {E} (X)\geq \operatorname {P} (X\geq a)\cdot \operatorname {E} (X|X\geq a)\geq a\cdot \operatorname {P} (X\geq a)}$, which directly leads to ${\displaystyle \operatorname {P} (X\geq a)\leq {\frac {\operatorname {E} (X)}{a}}}$.
		
\end{Hint}
\begin{Hint}{1.10}
		\begin{enumerate}[i.]
			\item Use the fundamental bridge. Note that
			\begin{equation*}
				\begin{array}{cl}
						P(|X-\mu|\geq \epsilon) &= E(1_{\{\left( |X-\mu|\geq \epsilon\right) \}})= E(1_{\{  \frac{|X-\mu|}{\epsilon}\geq 1  \}})
				\end{array}
			\end{equation*}
		\item Show that $1_{\{ \frac{|X-\mu|}{\epsilon}\geq 1  \}}\leq \left( \frac{|X-\mu|}{\epsilon}\right)^2 $.
		\item The above two imply that $P(|X-\mu|\geq \epsilon)\leq E\left( \frac{|X-\mu|}{\epsilon}\right)^2$.
		\end{enumerate}
	
\end{Hint}
\begin{Hint}{1.12}
			Use the result from Ex \ref{ex:chap06:05}.
		
\end{Hint}
\begin{Hint}{1.13}
			First, derive the identity $\sum_{i = 1}^n (X_i - \mu)^2 = \sum_{i = 1}^n (X_i - \bar{X}_n)^2 + n (\bar{X}_n - \mu)^2$.
		
\end{Hint}
\begin{Hint}{1.14}
		Try to make use the fact that sample average of i.i.d. data goes to the expectation by decomposing $S^2$ as the sum of components with sample averages.  $$S_n^2 = \frac{n}{n - 1}\frac{1}{n} \sum_{i = 1}^n (X_i - \mu)^2 - \frac{n}{n - 1} (\bar{X}_n - \mu)^2.$$
	
\end{Hint}
\begin{Hint}{1.15}
			If you were to throw a fair coin a large number of times, what is the proportion of heads you would expect?
		
\end{Hint}
\begin{Hint}{1.16}
			FUse that if $X \sim N(\mu, \sigma^2)$, the MGF of $X$ is given by $M_X(t) = e^{\mu t} e^{\frac{1}{2} \sigma^2 t^2}$.
		
\end{Hint}
\begin{Hint}{1.18}
			Recall the formula for geometric series: for $|\rho| < 1$, $\sum_{k = 0}^{\infty} \rho^k = \frac{1}{1 - \rho}$.
		
\end{Hint}
\begin{Hint}{1.19}
			For $X \sim N(\mu, \sigma^2)$, the MGF of $X$ is given by $M_X(t) = e^{\mu t} e^{\frac{1}{2} \sigma^2 t^2}$. Now take derivatives.
		
\end{Hint}
\begin{Hint}{1.20}
			For $X \sim N(\mu, \sigma^2)$, the MGF of $X$ is given by $M_X(t) = e^{\mu t} e^{\frac{1}{2} \sigma^2 t^2}$. Now take derivatives.
		
\end{Hint}
\begin{Hint}{1.21}
		MGFs determines distributions. Show the MGF of the sum can not be written in the form of the Expo MGF.
	
\end{Hint}
\begin{Hint}{1.22}
		MGF!
	
\end{Hint}
\begin{Hint}{1.23}
		MGF!
	
\end{Hint}
\begin{Hint}{1.24}
		The sum of independent Gaussian is Gaussian, use the fact that $X_1+X_2\sim N(\mu_1+\mu_2, \sigma_1^2+\sigma_2^2)$ when $X_1,X_2$ are independent and $X_i\sim N(\mu_i, \sigma_i^2), i=1,2$.
	
\end{Hint}
\begin{Hint}{1.28}
		MGF!
	
\end{Hint}

% % \begin{Solution}{1.1}
			Let $X = 10^{100} B$, where $B \sim \text{Bern}(10^{-10})$. The mean $\mu$ of $X$ is $10^{100} \cdot 10^{-10} = 10^{90}$, which is very large. In contrast, the median is 0, which is closer to the value $X$ generally takes.
		
\end{Solution}
\begin{Solution}{1.2}
			The first sentence uses the ``Median'', and the ``average level'' refers to the ``Mean''.  The second sentence compares the ``median'' with the ``mean''.
		
\end{Solution}
\begin{Solution}{1.3}
~
			\begin{enumerate}
				\item Let $X$ be a random variable. We want to show that the value of $c$ that minimizes $E(X - c)^2$ is $c = \mu$, where $\mu$ denotes the mean of $X$. We have
				\begin{align*}
					E(X - c)^2 & = E((X - \mu) + (\mu - c))^2 \\
					& = E(X - \mu)^2 + 2 E((X - \mu)(\mu - c)) + E(\mu - c)^2 \\
					& = E(X - \mu)^2 + (\mu - c)^2
				\end{align*}
				It is easily seen that $E(X - c)^2$ is minimal for $c = \mu$.
				\item Let $X$ be a random variable. We want to show that the value of $a$ that minimizes $E|X - a|$ is $a = m$, where $m$ denotes the median of $X$. We want to evaluate $E|X - a|$ for $a \neq m$.
					
				Assume $m < a$. If $X \leq m$, then
				\begin{equation*}
					|X - a| - |X - m| = a - X - (m - X) = a - m.
				\end{equation*}
				If $X > m$, then
				\begin{equation*}
					|X - a| - |X - m| = X - a - (X - m) = m - a.
				\end{equation*}
				Now let $Y = |X - a| - |X - m|$ and let $I = 1$ if $X \leq m$ and $I = 0$ if $X > m$. Then
				\begin{align*}
					E(Y) & = E(YI) + E(Y(1 - I)) \\
					& \geq (a - m) E(I)	+ (m - a) E(1 - I) \\
					& = (a - m) \P{X \leq m} + (m - a) \P{X > m} \\
					& = (a - m) \P{X \leq m} - (a - m) (1 - \P{X \leq m}) \\
					& = (a - m) (2 \P{X \leq m} - 1).
				\end{align*}
				By the definition of a median, we have $2 \P{X \leq m} - 1 \geq 0$. Hence, $E(Y) \geq 0$, which implies $E(|X - m|) \leq E(|X - a|)$. Hence for all $E(|X - m|) \leq E(|X - a|)$ for all $m < a$. Repeat similar steps for $m > a$ and conclude $E|X - a|$ is minimal for $a = m$.
				\item Let $X \sim \text{Bern}(0.25)$. Then the mean of $X$ is $\mu = 0.25$, while the median of $X$ is $m = 0$. Note that $E(X - \mu)^2 = V(X) = 0.25(1 - 0.25) = 0.1875$ and $E(X - m)^2 = E(X)^2 = 0.25$; hence $E(X - \mu)^2 \leq E(X - m)^2$ as expected. Moreover, using LOTUS, $E|X - \mu| = |0 - 0.25|(1 - 0.25) + |1 - 0.25|0.25 = 0.375$ and $E|X - m| = E|X| = E(X) = 0.25$; thus, $E|X - m| \leq E(X - \mu)^2$, as expected.
			\end{enumerate}
		
\end{Solution}
\begin{Solution}{1.5}
			Let $X$ follow a standard Cauchy distribution. The PDF of $X$ is given by $f(x) = \frac{1}{\pi (1 + x^2)}$. Note that $f'(x) = -\frac{2x}{\pi (1 + x^2)}$; hence $f'(x) = 0 \iff x = 0$. It follows that $f$ has a maximum at $x = 0$. (Formally, you have to check $f''(0) < 0$, too.) Since this maximum is unique, the mode of $X$ is $0$. As $f(x) = f(-x) = 0$ for all $x \in \mathbb{R}$, the standard Cauchy distribution is symmetric about $0$. Therefore, $P(X \leq 0) = \int_{-\infty}^0 f_X(x) \mathrm{d}x = \int_{-\infty}^0 f_X(-x) \mathrm{d}x = \int_0^{\infty} f_X(y) \mathrm{d}y = \P{X \geq 0}$. As $\P{X \leq 0} + \P{X \geq 0} = 1$ it follows that $\P{X \leq 0} = \frac{1}{2}$. Hence, by definition, the median of the Cauchy distribution is $0$.
		
\end{Solution}
\begin{Solution}{1.6}
			Let $X$ be a random variable whose distribution is symmetric about its mean $\mu$. Then $Y = X - \mu$ is symmetric about 0. Due to symmetry, $Y$ and $-Y$ have the same distribution. That implies $E(Y^3) = E((-Y)^3)$. This in turn implies $E(Y^3) = 0$. It follows that $\text{Skew}(X) = E\left(\frac{X - \mu}{\sigma}\right)^3 = \frac{1}{\sigma^3} E(Y^3) = 0$.
		
\end{Solution}
\begin{Solution}{1.7}
			There are infinitely many possible asymmetric distributions with zero skewness. Zero skewness means that overall, the tails on both sides of the mean balance out. This occurs, for example, when one tail is ``long" but the other tail is ``fat". An easy example of an asymmetric distribution with zero skewness is obtained by considering a discrete random variable with 3 values in its support. Check from the definition that a random variable $X$ has zero skewness if $E(X) = E(X^3) = 0$. One random variable satisfying this property is the random variable $X$ with $\P{X = -3} = 0.1$, $\P{X = -1} = 0.5$ and $\P{X = 2} = 0.4$. The distribution of $X$ is asymmetric by construction. Verify yourself that $E(X) = E(X^3) = 0$.
		
\end{Solution}
\begin{Solution}{1.8}
			The $r$th central moment is given by
			\begin{align*}
				\mu_r & = \int_a^b \left[x - E(X)\right]^r \cdot f_X(x) \mathrm{d}x = \frac{1}{b - a} \int_a^b \left[x - \frac{b - a}{2}\right]^r \mathrm{d}x = \frac{1}{(b - a) 2^r} \int_a^b \left[2x - (a + b)\right]^r \mathrm{d}x \\
				&= \frac{1}{(b - a) 2^r} \left[\frac{(2x - (a + b))^{r + 1}}{2(r + 1)}\right]_a^b = \frac{1}{(b - a) 2^r} \cdot \frac{(b - a)^{r + 1} - (-1)^{r + 1} (b - a)^{r + 1}}{2(r + 1)}
			\end{align*}
			which is zero when $r$ is odd.
		
\end{Solution}
\begin{Solution}{1.9}
			Correct. Recall that the variance of a random variable $X$ is defined as $V(X) = E(X - E(X))^2$. Because $(X - E(X))^2$ is strictly non-negative, $V(X)$ can only be zero if $(X - E(X))^2$ is always zero (or with probability one). $(X - E(X))^2$ is always zero if and only if $X = E(X)$ with probability one. If $X = E(X)$, $X$ always has the same value, i.e. is constant with probability one. Hence, if a random variable is of zero variance, then it is a constant with probability one.
		
\end{Solution}
\begin{Solution}{1.10}
			\begin{enumerate}[i.]
			\item Use the fundamental bridge. Note that
			\begin{equation*}
				\begin{array}{cl}
					P(|X-\mu|\geq \epsilon) &= E(1_{\{\left( |X-\mu|\geq \epsilon\right) \}})= E(1_{\{  \frac{|X-\mu|}{\epsilon}\geq 1  \}})
				\end{array}
			\end{equation*}
			\item Show that $1_{\{ \frac{|X-\mu|}{\epsilon}\geq 1  \}}\leq \left( \frac{|X-\mu|}{\epsilon}\right)^2 $. For any $s\in S$, if $1_{\{ \frac{|X-\mu|}{\epsilon}\geq 1  \}}(s)=0$, then we know by the non-negativity of the square function $\left( \frac{|X-\mu|}{\epsilon}\right)^2(s)\geq 0=1_{\{ \frac{|X-\mu|}{\epsilon}\geq 1  \}}(s)$;  if $1_{\{ \frac{|X-\mu|}{\epsilon}\geq 1  \}}(s)=1$, then we know by the definition of the indicator function that $\frac{|X-\mu|}{\epsilon}(s)\geq 1 =1_{\{ \frac{|X-\mu|}{\epsilon}\geq 1  \}}(s)$. Therefore, for all outcomes $s\in S$, $1_{\{ \frac{|X-\mu|}{\epsilon}\geq 1  \}}(s)\leq \left( \frac{|X-\mu|}{\epsilon}\right)^2(s)$, and thus $$P\left\{1_{\{ \frac{|X-\mu|}{\epsilon}\geq 1  \}}\leq \left( \frac{|X-\mu|}{\epsilon}\right)^2\right\}=P(S)=1. $$
			\item The above two imply that $P(|X-\mu|\geq \epsilon)\leq E\left( \frac{|X-\mu|}{\epsilon}\right)^2$.
		\end{enumerate}
	
\end{Solution}
\begin{Solution}{1.11}
		We want to show that for some constant $c$ we have that for any $\varepsilon>0$ $\P{|\frac{1}{n}\sum_{i=1}^n(X_{i}-E(X_i))^2-c|>\varepsilon}\rightarrow 0$. Denote $E(X_i)=\mu$ and $Y_n=\frac{1}{n}\sum_{i=1}^n(X_{i}-\mu)^2$, then using the result from the previous exercise we obtain $\P{|Y_n-E(Y_n)|\geq\varepsilon}\leq \frac{Var(Y_n)}{\varepsilon^2}$. By independence of the $X_i$ $Var(Y_n)=Var(Y_n=\frac{1}{n}\sum_{i=1}^n(X_{i}-\mu)^2)=\frac{1}{n}Var((X_i-\mu)^2)\rightarrow 0$, because of $Var((X_i-\mu)^2)$ is finite by the finite fourth moment of $X_i$. We conclude that $\frac{1}{n}\sum_{i=1}^n(X_{i}-E(X_i))^2$ converges to $E(\frac{1}{n}\sum_{i=1}^n(X_{i}-E(X_i))^2)=Var(X_i)$.
	
\end{Solution}
\begin{Solution}{1.12}
			Let $X_1, \ldots, X_n$ be i.i.d. random variables with mean $\mu$ and variance $\sigma^2$. The sample mean is given by $\bar{X}_n = \frac{1}{n} \sum_{i = 1}^n X_i$. The variance of the sample mean is given by
			\begin{align*}
				V(X_n^2) & = V\left(\frac{1}{n} \sum_{i = 1}^n X_i\right) = \frac{1}{n^2} \sum_{i = 1}^n V(X_i) = \frac{1}{n^2} \cdot n \sigma^2 = \frac{\sigma}{n}.
			\end{align*}	
			It follows that $V(X_n^2) \to 0$ as $n \to \infty$. Now invoke the result of Ex \ref{ex:chap06:05} to conclude that the sample mean converges to a constant with probability one.
		
\end{Solution}
\begin{Solution}{1.13}
			Let $X_1, \ldots, X_n$ be i.i.d. random variables with mean $\mu$ and variance $\sigma^2$. The sample mean is given by $\bar{X}_n = \frac{1}{n} \sum_{i = 1}^n X_i$. The sample variance is given by $S_n^2 = \frac{1}{n - 1} \sum_{i = 1}^n (X_i - \bar{X}_n)^2$. First, we construct the identity
			\begin{align*}
				\sum_{i = 1}^n (X_i - \mu)^2 & = \sum_{i = 1}^n ((X_i - \bar{X}_n) + (\bar{X}_n - \mu))^2 \\
				& = \sum_{i = 1}^n (X_i - \bar{X}_n)^2 + 2 (\bar{X}_n - \mu) \sum_{i = 1}^n (X_i - \bar{X}_n) + \sum_{i = 1}^n (\bar{X}_n - \mu)^2 \\
				& = \sum_{i = 1}^n (X_i - \bar{X}_n)^2 + n (\bar{X}_n - \mu)^2
			\end{align*}
			(Here, we used that that $\sum_{i = 1}^n (X_i - \bar{X}_n) = \left(\sum_{i = 1}^n X_i\right) - n \bar{X}_n = n \bar{X}_n - n \bar{X}_n = 0$.) Rewriting this identity yields
			\begin{equation*}
				\sum_{i = 1}^n (X_i - \bar{X}_n)^2 = \sum_{i = 1}^n (X_i - \mu)^2 - n (\bar{X}_n - \mu)^2.
			\end{equation*}
			Note that $E(\sum_{i = 1}^n (X_i - \mu)^2) = n \sigma^2$ and $E(n (\bar{X}_n - \mu)^2) = n V(\bar{X}_n) = n \cdot \frac{\sigma}{n} = \sigma$. Hence,
			\begin{align*}
				E(S_n^2) & = E\left(\frac{1}{n - 1} \sum_{i = 1}^n (X_i - \bar{X}_n)^2\right) \\
				& = \frac{1}{n + 1} \left(E\left(\sum_{i = 1}^n (X_i - \mu)^2\right) - E\left(n\left(\bar{X}_n - \mu\right)^2\right)\right) = \frac{1}{n - 1} (n \sigma^2 - \sigma^2) = \sigma^2.
			\end{align*}
		
\end{Solution}
\begin{Solution}{1.14}
		By the result of Ex \ref{ex:chap06:04}, we have $\sum_{i = 1}^n (X_i - \bar{X}_n)^2 = \sum_{i = 1}^n (X_i - \mu)^2 - n (\bar{X}_n - \mu)^2$. As such, $S_n^2 = \frac{n}{n - 1}\frac{1}{n} \sum_{i = 1}^n (X_i - \mu)^2 - \frac{n}{n - 1} (\bar{X}_n - \mu)^2$. The latter term converges to $0$ as $\bar{X}_n$ converges to $\mu$. Hence, as $n \to \infty$, the term $\frac{1}{n} \sum_{i = 1}^n (X_i - \mu)^2$ which is the sample average of i.i.d. $Z_i=(X_i - \mu)^2$ converges to the expectation of $Z_i$, which in turn is the variance of $X_i$.
		
\end{Solution}
\begin{Solution}{1.15}
		If you were to throw a fair coin a large number of times, you would expect the proportion of heads to converge to 0.5 (see Ex \ref{ex:chap06:05}). So to verify whether the coin is fair, you could throw it a large number of time and assess whether the sample proportion of heads approximates 0.5. To illustrate, run the following code:
\begin{minted}{python}
import numpy as np
import matplotlib.pyplot as plt

nSeq = 5
nTrials = 10 ** 3
p = 0.5

for j in range(nSeq):
    x = np.zeros(nTrials + 1, float)
    Mean_list = []
    for i in range(nTrials):
        x[i] = np.random.binomial(1, p)
        xbar = np.mean(x[:i+1])
        Mean_list.append(xbar)

    plt.plot(range(nTrials), Mean_list, label='Sample_' + str(j + 1))
plt.ylabel('Estimated proportion of heads')
plt.xlabel('Trials')
plt.legend(loc=0, ncol=3, fontsize='small')
plt.show()
\end{minted}
		Indeed, after throwing a fair coin 1000 times, the sample proportion of heads is close to 0.5. (Check yourself what happens if $p \neq 0.5$!)
		
\end{Solution}
\begin{Solution}{1.16}
			For $X \sim N(\mu, \sigma^2)$, the MGF of $X$ is given by $M_X(t) = e^{\mu t} e^{\frac{1}{2} \sigma^2 t^2}$. To verify this, first write $X = \mu + \sigma Z$ for $Z \sim N(0,1)$, and calculate
			\begin{align*}
				M_Z(t) & = E(e^{tZ}) = \int_{-\infty}^{\infty} e^{tz} \cdot \frac{1}{\sqrt{2 \pi}} e^{-\frac{1}{2} z^2} \mathrm{d}z = \int_{-\infty}^{\infty} e^{tz} \cdot \frac{1}{\sqrt{2 \pi}} e^{-\frac{1}{2} z^2} \mathrm{d}z \\
				& = e^{\frac{1}{2} t^2} \int_{-\infty}^{\infty} \frac{1}{\sqrt{2 \pi}} e^{-\frac{1}{2} (z - t)^2} \mathrm{d}z = e^{\frac{1}{2} t^2}
			\end{align*}
			(The last step follows from recognizing $\frac{1}{\sqrt{2 \pi}} e^{-\frac{1}{2} (z - t)^2}$ as a PDF.) It then follows that
			\begin{align*}
				M_X(t) & = E\left(e^{tX}\right) = E\left(e^{t(\mu + \sigma Z)}\right) = e^{\mu t} E\left(e^{t \sigma Z}\right) = e^{\mu t} M_Z(\sigma t) = e^{\mu t} e^{\frac{1}{2} \sigma^2 t^2}.
			\end{align*}
			Now let $X_1\sim N(\mu_1, \sigma_1^2)$ and $X_2\sim N(\mu_2, \sigma_2^2)$. We have
			\begin{align*}
				M_{X_1 + X_2}(t) & = E\left(e^{t(X_1 + X_2)}\right) = E\left(e^{t X_1}\right) E\left(e^{t X_2}\right) = e^{\mu_1 t} e^{\frac{1}{2} \sigma_1^2 t^2} e^{\mu_2 t} e^{\frac{1}{2} \sigma_2^2 t^2} = e^{(\mu_1 + \mu_2) t} e^{\frac{1}{2} (\sigma_1^2 + \sigma_2^2) t^2}
			\end{align*}
			This is the MGF of the $N(\mu_1 + \mu_2, \sigma_1^2 + \sigma_2^2)$ distribution. It follows that $X_1 + X_2 \sim N(\mu_1 + \mu_2, \sigma_1^2 + \sigma_2^2)$.
		
\end{Solution}
\begin{Solution}{1.17}
			Let $X \sim \text{Expo}(\lambda)$ for some $\lambda > 0$. Let $Y = \lambda X$ for some $\lambda > 0$. The MGF of $X$ is given by
			\begin{align*}
				M_X(t) = E(e^{tX}) = \int_0^{\infty} e^{tx} \lambda e^{-\lambda x} \mathrm{d}x = \int_0^{\infty} \lambda e^{-x(\lambda - t)} \mathrm{d}x = \left[- \frac{\lambda}{\lambda - t} e^{-x(\lambda - t)}\right]_0^{\infty} = \frac{\lambda}{\lambda - t}, \quad t < \lambda.
			\end{align*}
			It follows that the MGF of $Y$ is given by
			\begin{align*}
				M_Y(t) = E(e^{tY}) = E(e^{\lambda t X}) = M_X(\lambda t) =  \frac{\lambda}{\lambda - \lambda t} = \frac{1}{1 - t}, \quad t < 1.
			\end{align*}
			This is the MGF of the $\text{Expo}(1)$ distribution. Hence, $Y \sim \text{Expo}(1)$.
		
\end{Solution}
\begin{Solution}{1.18}
			Let $X$ have the probability distribution $f(x) = e \left(\frac{1}{3}\right)^x$ for $x = 1, 2, \ldots$. Using LOTUS, the MGF is given by
		\begin{align*}
			M_X(t) = E(e^{tX}) = \sum_{x=1}^{\infty} e^{tx} f(x) = \sum_{x=1}^{\infty} e^{tx} \cdot 2 \left(\frac{1}{3}\right)^x = \sum_{x=0}^{\infty} e^{t(x+1)} \cdot 2 \left(\frac{1}{3}\right)^{x+1} = \frac{2 e^t}{3} \sum_{x=0}^{\infty} \left(\frac{e^t}{3}\right)^x = \frac{2 \left(\frac{e^t}{3}\right)}{1 - \left(\frac{e^t}{3}\right)} = \frac{2e^t}{3 - e^t},
		\end{align*}
		for $|t|<1$. Taking derivatives, we obtain
		\begin{align*}
			M_X'(t) & = \frac{(3 - e^t) 2e^t - 2e^t (-e^t)}{(3 - e^t)^2} = \frac{6e^t}{(3 - e^t)^2} \\
			M_X''(t) & = \frac{(3 - e^t) \cdot 6e^t - 6e^t \cdot 2(3 - e^t)(-e^t}{(3 - e^t)^4}.
		\end{align*}
		It follows that $E(X) = M_X'(0) = \frac{6}{4} = \frac{3}{2}$ and $E(X^2) = \frac{24 - 12 \cdot 2 \cdot -1}{16} = 3$. Therefore $V(X) = E(X^2) - E(X)^2 = 3 - \left(\frac{6}{4}\right)^2 = 3 - \frac{9}{4} = \frac{3}{4}$.
		
\end{Solution}
\begin{Solution}{1.19}
			Let $X \sim N(\mu, \sigma^2)$. The MGF of $X$ is given by $M_X(t) = e^{\mu t} e^{\frac{1}{2} \sigma^2 t^2}$. As such,
			\begin{align*}
				M_X'(t) & = (\mu + \sigma^2 t) e^{\mu t} e^{\frac{1}{2} \sigma^2 t^2} = (\mu + \sigma^2 t) M_X(t) \\
				M_X''(t) & = (\mu + \sigma^2 t)^2 M_X(t) + \sigma^2 M_X(t)
			\end{align*}
			We obtain $E(X) = M_X'(0) = \mu M_X(0) = \mu \cdot 1 = \mu$ and $E(X^2) = M_X''(0) = \mu^2 M_X(0) + \sigma^2 M_X(0) = \mu^2 + \sigma^2$. It follows that $V(X) = E(X^2) - E(X)^2 = (\mu^2 + \sigma^2) - \mu^2 = \sigma^2$.
		
\end{Solution}
\begin{Solution}{1.20}
			Let $X \sim N(\mu, \sigma^2)$. The MGF of $X$ is given by $M_X(t) = e^{\mu t} e^{\frac{1}{2} \sigma^2 t^2}$. As such,
			\begin{align*}
				M_X'(t) & = (\mu + \sigma^2 t) e^{\mu t} e^{\frac{1}{2} \sigma^2 t^2} = (\mu + \sigma^2 t) M_X(t) \\
				M_X''(t) & = (\mu + \sigma^2 t)^2 M_X(t) + \sigma^2 M_X(t) \\
				M_X^{(3)}(t) & = (\mu + \sigma^2 t)^3 M_X(t) + 3 \sigma^2 (\mu + \sigma^2 t) M_X(t) \\
				M_X^{(4)}(t) & = (\mu + \sigma^2 t)^4 M_X(t) + 3 \sigma^2 (\mu + \sigma^2 t)^2 M_X(t) + 3 \sigma^2 (\mu + \sigma^2 t)^2 M_X(t) + 3 \sigma^4 M_X(t)
			\end{align*}	
			We obtain $E(X) = M_X'(0) = \mu$, $E(X^2) = M_X''(0) = \mu^2 + \sigma^2$, $E(X^3) = 3 \mu \sigma^2 + \mu^3$ and $E(X^4) = \mu^4 + 6 \mu^2 \sigma^2 + 3 \sigma^4$.
			
			The skewness of $X$ is given by
			\begin{align*}
				\text{Skew}(X) = E\left(\frac{X - \mu}{\sigma}\right)^3 = \frac{E\left(X^3 - 3X^2 \mu + 3X \mu^2 - \mu^3\right)}{\sigma^3}.
			\end{align*}	
			Plugging in the obtained values for $E(X)$, $E(X^2)$ and $E(X^3)$ yields $\text{Skew}(X) = 0$.
			
			The kurtosis of $X$ is given by
			\begin{align*}
				\text{Kurt}(X) = E\left(\frac{X - \mu}{\sigma}\right)^4 = \frac{E\left(X^4 - 4 \mu X^3 + 6 \mu^2 X^2 - 3 \mu^4\right)}{\sigma^4}.
			\end{align*}	
			Plugging in the obtained values for $E(X)$, $E(X^2)$ and $E(X^3)$ yields $\text{Kurt}(X) = 3$.
		
\end{Solution}
\begin{Solution}{1.21}
		\begin{enumerate}
			\item   MGF of a Expo($\lambda$)-distributed $X$: $$M_X(t)=Ee^{tX}=\int_{0}^{\infty} e^{tx}e^{-\lambda x} dx=  \frac{1}{\lambda -t},$$ {which is finite for, e.g., $t\in (-\lambda/2, \lambda/2)$ (so the MGF is well-defined)}.
			\item {MGF of $Y_1+Y_2$} with $Y_i\sim i.i.d.\text{Expo}(1)$:
			\begin{align*}
				M_{Y_1+Y_2}(t) = M_{Y_1}(t)M_{Y_2}(t)=\frac{1}{(1 -t)^2}.
			\end{align*}
			{which is finite for, e.g., $t\in (-1/2, 1/2)$ (so the MGF is well-defined).}\\~\\
			\item $\frac{1}{(1 -t)^2}$ can not be written in the form of $\frac{1}{\lambda -t}$ for any $\lambda$'s. We prove by contradiction, suppose they are equal for some $\lambda$ then $\left.\frac{1}{(1 -t)^2}\right|_{t=0}=\left.\frac{1}{\lambda -t}\right|_{t=0}$ and thus $\lambda=1$, however, $\left.\frac{1}{(1 -t)^2}\right|_{t=a}\neq \left.\frac{1}{1 -t}\right|_{t=a}$ for any $a\neq 0$.
			\item {The MGF of  $Y_1+Y_2$ is not the MGF of a exponential distribution}, then we know  a sum of two i.i.d. Expo($1$)-distributed r.v.'s is not exponentially distributed as MGF determines distributions (different MGF forms, different distributions).
		\end{enumerate}
	
\end{Solution}
\begin{Solution}{1.22}
		\begin{enumerate}
				\item {MGF of a Pois($\lambda$)-distributed $X$}: $$M_X(t)=Ee^{tX} =e^{\lambda (e^t-1)},$$ {which is finite for, e.g., $t\in (-\lambda/2, \lambda/2)$ (so the MGF is well-defined)}.
			\item {MGF of $\sum_{i=1}^{n} Y_i$} with $Y_i\sim \textit{independent  Pois}(\lambda_i)$:
			\begin{align*}
				M_{\sum_{i=1}^{n}Y_i}(t) =\prod_{i=1}^{n} e^{\lambda_i (e^t-1) }=e^{\sum_{i=1}^{n} \lambda_i (e^t-1) }.
			\end{align*}
			{which is finite for, e.g., $t\in (-1/2, 1/2)$ (so the MGF is well-defined) and is the MGF of a Pois($\sum_{i=1}^{n} \lambda_i$)-distributed $X$.} \\~\\
			\item The MGF of   $\sum_{i=1}^{n} Y_i$ is  the MGF of Pois($\sum_{i=1}^{n} \lambda_i$), then we know a  sum of $n$ independent Pois($\lambda_i$), $i\leq n$ is still Poisson.
		\end{enumerate}
	
\end{Solution}
\begin{Solution}{1.23}
		\begin{enumerate}
					\item MGF of $X \sim N(\mu,\sigma^2)$.
			\begin{align*}
				{ M_X(t)}& =  {\color{red}e^{\mu t + \frac{1}{2}\sigma^2 t^2}},
			\end{align*}  {which is finite for $t\in \mathbb{R}$, so the MGF,  $M_X$, is well defined.}
			\item MGF of $\sum_{i=1}^{n} X_i$ with $X_i\sim \textit{independent }N(\mu_i,\sigma^2_i)$, $i\leq n$:
			\begin{align*}
				M_{\sum_{i=1}^{n}X_i}(t) =\prod_{i=1}^{n} e^{\mu_i t + \frac{1}{2}\sigma_i^2 t^2}=e^{\left(\sum_{i=1}^{n} \mu_i\right) t + \frac{1}{2}\left(\sum_{i=1}^{n}\sigma_i^2\right) t^2 }.
			\end{align*}
			which is finite for, e.g., $t\in (-1, 1)$ (so the MGF is well-defined) and is the MGF of a $N\left(\sum_{i=1}^{n}\mu_i, \sum_{i=1}^{n}\sigma_i^2 \right)$-distributed r.v. \\~\\
			\item The MGF of   $\sum_{i=1}^{n} X_i$ is  the MGF of $N\left(\sum_{i=1}^{n}\mu_i, \frac{1}{2}\left(\sum_{i=1}^{n}\sigma_i^2\right)\right)$, then we know a  sum of n independent $N(\mu_i,\sigma^2_i)$, $i\leq n$ is still normal.
		\end{enumerate}
	
\end{Solution}
\begin{Solution}{1.24}
		\begin{enumerate}
			\item $N(\mu, \frac{1}{n})$, when $n\rightarrow \infty$, $\bar{X}_n$ goes closer and closer to a constant value $\mu$.
			\item $N(0,1)$.
			\item Use the fact that  $Y=\sqrt{n}(\bar{X}_n-\mu)\sim N(0,1)$ for a given $\mu$. For example, in the case $n=10000$, $\mu=2$, we would need to calculate $P(Y\geq \sqrt{n}(2.01-\mu))=P(Y\geq 100)=\Phi(100)$, which is very small.
		\end{enumerate}
	
\end{Solution}
\begin{Solution}{1.26}
		\begin{enumerate}
			\item
			\begin{enumerate}
			\item {MGF of a Pois(1)-distributed $X$}: $$M_X(t)=Ee^{tX} =e^{(e^t-1)},$$ {which is finite for, e.g., $t\in (-1/2, 1/2)$ (so the MGF is well-defined)}.\\
			\item MGF of $Y=\frac{1}{\sqrt{n}}\sum_{i=1}^n (X_i-1),$
			\begin{align*}
				M_{Y}(t)= \prod_{i=1}^{n}M_{(X_i-1)}(t/\sqrt{n}) =(e^{-t/\sqrt{n}} M_{X_1}(t/\sqrt{n}))^n = {\color{red}e^{n\left(e^{\frac{t}{\sqrt{n}}}-1-\frac{t}{\sqrt{n}}\right)}},
			\end{align*} {which is finite for $t\in \mathbb{R}$, so the MGF,  $M_Y$, is well defined.}
			\item {MGF $M_\xi(t)$ for $\xi \sim N(\mu,\sigma^2)$.}
			\begin{align*}
				{ M_\xi(t)}& =  {\color{red}e^{\mu t + \frac{1}{2}\sigma^2 t^2}},
			\end{align*}  {which is finite for $t\in \mathbb{R}$, so the MGF,  $M_\xi$, is well defined.}
		\end{enumerate}
		\item Note that $\lim\limits_{n\rightarrow \infty }{\color{red}e^{n\left(e^{\frac{t}{\sqrt{n}}}-1-\frac{t}{\sqrt{n}}\right)}}= \lim\limits_{n\rightarrow \infty}{e^{n\left(\sum_{i=0}^{\infty}\left(\frac{t}{\sqrt{n}} \right)^i/i!   -1-\frac{t}{\sqrt{n}}\right)}}=  {\color{red}e^{\frac{1}{2}t^2}}$.
		\item{The MGFs of $\frac{1}{\sqrt{n}}\sum_{i=1}^n (X_i-1)$ and $N(0,1)$ are the same in the limit}, which implies in the limit these two distributions coincide.
	\end{enumerate}~\\
	
\end{Solution}
\begin{Solution}{1.28}
~\\
		\begin{enumerate}		
			\item \begin{enumerate}
				\item $X_i\sim \text{i.i.d. Bern}(p)$, $$M_{X_i}(t)=p(e^t-1)+1, t\in \mathbb{R};$$
				\item $Y=\sum_{i=1}^{n}X_i\sim \text{Bin}(n,p),$
				$$M_{Y}(t)= \prod_{i=1}^{n}M_{X_i}(t) =(M_{X_1}(t))^n = {\color{red}\left(1+p(e^t-1)\right)^n}, t\in \mathbb{R}.$$
				\item[] $Z\sim\text{Pois($\lambda$)}$, $$M_{Z}(t)=\sum_{n=0}^\infty e^{tn}e^{-\lambda} \lambda^n/n!=e^{-\lambda} \sum_{n=0}^\infty  (\lambda e^t)^n/n! =e^{-\lambda}e^{\lambda e^t}    = {\color{red}e^{\lambda (e^t-1) }}, t\in \mathbb{R}.$$
			\end{enumerate}
			\item Note that $\lim\limits_{n\rightarrow \infty, p=\lambda/n}{\color{red}\left(1+p(e^t-1)\right)^n}= \lim\limits_{n\rightarrow \infty}\left(1+ \frac{\lambda(e^t-1)}{n}\right)^n=  {\color{red}e^{\lambda (e^t-1) }}$.
			\item{ The MGFs are the same in the limit}, which implies in the limit these two distributions coincide.
		\end{enumerate}~\\
	
\end{Solution}
\begin{Solution}{1.29}
  	\begin{enumerate}
  		\item 		 The probability of having strictly more than one winner is zero (\textbf{use words to motivate is also okay, here I only show solutions motivated by formulas} Methods are not unique, using indicator functions is also okay.):
  		\begin{align*}
  			\mathbb{P}\left(\left\{ \bigcup_{i=1}^n \left\{ \max_{l\neq i} \{X_l\} =\max_{1\leq j \leq n} \{X_j\} \right\}    \right\}^c\right) \leq \mathbb{P}\left( \bigcup_{i\neq j} \{ X_i =X_j \} \right) \leq \sum_{i\neq j} \mathbb{P}\left( X_i =X_j \right)=0
  		\end{align*}	
  		where the last equation is due to the fact that $X_i-X_j$ is one continuous r.v. and thus $\mathbb{P}\left( \{ X_i =X_j \} \right) = \mathbb{P}\left( \{ X_i-X_j =0\} \right)=0$ (the probability of one continuous r.v. equal to one fixed constant is zero).
  		\item			By symmetry of continuous random variables we know
  		\begin{align*}
  			\mathbb{P}\left(X_{a_1}<X_{a_2}<\cdots < X_{a_j} \right) =1/j!
  		\end{align*}	
  		for arbitrary permutation $(a_1,\cdots, a_j)$ of $(1,\cdots, j)$. Now among all permutations, there are $(j-1)!$ permutations of the format  $(a_1,\cdots, a_{j-1}, j)$: therefore,
  		\begin{align*}
  			\mathbb{P}\left(\bigcup_{(a_1,\cdots, a_{j-1})}\left\{X_{a_1}<X_{a_2}<\cdots < X_{a_{j-1}}<X_j\right\} \right) =1/j
  		\end{align*}
  	\item 			Note that  $X_1+X_2$ follows the same distribution as $1-X_{n-1}+1-X_n$ (they have the same PDF). Therefore, by the LOTUS,
  	\begin{align*}
  		&\mathbb{E} \left(\log\left(\frac{X_1+X_2}{2-(X_{n-1}+X_{n})} \right) \right)=\mathbb{E}\left(\log\left({X_1+X_2} \right) \right)- \mathbb{E}\left(\log\left( {2-(X_{n-1}+X_{n})} \right) \right) \\&= \int_{-\infty}^{+\infty} \log(x)g(x)dx- \int_{-\infty}^{+\infty} \log(x)g(x)dx =0
  	\end{align*}
  \item 			We first derive the MGF  of $X_1$ (for $t\neq 0$):
  \begin{align*}
  	\mathbb{E}[e^{tX_1}] =\int_{0}^{1} e^{tx} dx =  (e^t-1)/t
  \end{align*}	
  for $t=0$, we know $	\mathbb{E}[e^{tX_1}]=	\mathbb{E}[1]=1$	
  Therefore the MGF is well defined. Next, for $t\neq 0$
  \begin{align*}
  	M_{a+b X_1}(t) = e^{at}M_{X_1}(bt)= e^{at}\left(e^{bt}-1\right)/t
  \end{align*}	
  for $t=0$, $	M_{a+b X_1}(t) = 1$.
  \item 						The variance (by independence) is $Var(X^*)=\sum_{i=1}^nVar(X_i)/n = 1/12$.\\~\\
  For $t\neq 0$,
  \begin{align*}
  	&M_{X^*}(t) = M_{\left(\sum_{i=1}^n X_i\right) -\sqrt{n}/2  }(t/\sqrt{n}) =e^{-\sqrt{n}t/2 }\prod_{i=1}^nM_{ X_i}(t/\sqrt{n})\\
  	=& e^{-\sqrt{n}t/2 } \left(\sqrt{n}\left(e^{t/\sqrt{n}} -1 \right)/t\right)^n= \left(e^{-t/(2\sqrt{n}) }\sqrt{n}\left(e^{t/\sqrt{n}} -1 \right)/t\right)^n\\
  	=&  \left(\sqrt{n}\left(e^{t/(2\sqrt{n})} -e^{-t/(2\sqrt{n}) } \right)/t\right)^n =_{\textit{large n}} \left((t+t^3/24/n)/t\right)^n\\ \rightarrow_{n\rightarrow \infty}& e^{t^2/24}
  \end{align*}	
  When $t=0$, $M_{X^*}(t) =1$. 	
  The limiting distribution should be $N(0,1/12)$ by the format of our derived limiting MGF, since the normal distribution has the MGF $e^{\mu t + \sigma^2t^2/2 }$.	
  	\end{enumerate}
  ~\\~\\~\\
  Some simulation codes to help you with this exercise, also you may need some coding skills for the PD course.
  For continuouse random varaible, we can redraw many of its realised values, and then draw a histogram. Histogram can be regarded as a sample-version of the PDF (actually, it is indeed one estimator for the PDF). You can see the more data you use to draw the histogram, the histogram is smoother and closer to its PDF.

  We may use the following codes to draw $S=2,10,100,5000$ realised values from Unif(0,1) and then draw histogram:
  \begin{minted}[]{R}
set.seed(12)
n=1
sigma2=1/12
par(mfrow=c(2,2))
require(ggplot2)
loop.vector <- c(2,10,100,5000)
plot_list = list()
j=0
for(i in loop.vector){
	S=i
	j=j+1
	Unif_draws= runif(n*S, min = 0, max = 1)
	
	plot <- ggplot(data.frame(Unif_draws), aes(Unif_draws)) +
	geom_histogram(aes(y=..density..)) +
	ggtitle("n=",n)
	plot_list[[j]] = plot
}
#library(gridExtra)
require(gridExtra)
grid.arrange(plot_list[[1]], plot_list[[2]],plot_list[[3]],plot_list[[4]], nrow = 2)
  \end{minted}
 The output is the following figure~\\
 \begin{figure}[htbp!]
 	\includegraphics[width=0.7\textwidth]{0}
 \end{figure}  		   ~\\
 and indeed the larger the $S$ the closer the histogram to the PDF of Unif(0,1).\\~\\~\\

 \begin{minipage}{0.45\textwidth}
 	\includegraphics[width=0.9\textwidth]{1n2} \captionof{figure}{n=\textbf{2},S=1000}
 	\includegraphics[width=0.9\textwidth]{1n20} \captionof{figure}{n=\textbf{20},S=1000}
 \end{minipage}
 \begin{minipage}[b]{0.45\textwidth}
 	From the comparison between the histograms of $X^*_n$ (blue) and normal density (red), we can see it indeed gets closer and closer as $n$ increases. \\~\\
 	Here we fix a large $S$ to make sure the histograms are good approximations for the density of $X_n^*$.
 \end{minipage}
\newline
Each line of code can be executed using $\text{ctrl}+\text{Enter}$ in Rstudio.
~\\~\\
This draws 4 values from Unif(0,1)
\begin{minted}[]{R}
runif(4, min = 0, max = 1)
\end{minted}
Here are the codes that you may find how many are the largest number
\begin{minted}[]{R}
rand.a=runif(4, min = 0, max = 1)
sum(rand.a==max(rand.a))
\end{minted}
We may repeat this game many many times and check how many more than 1 winner in S repeated games via the following codes:
\begin{minted}[]{R}
n=4;
S=100;
rand.Srepeats=matrix(runif(n*S, min = 0, max = 1),ncol=S)
colMax<- apply(rand.Srepeats, 2, max)
# this gives you the a True and false matrix, and only true if the max in that column
result=colMax==t(matrix(rep((colMax),n),nrow=S))
# how many more than 1 winner in S repeated games
sum(colSums(result)>1)
\end{minted}

Here we comparing the histogram of X*
(we generate S values drawn from the same distribution as the one of X* and then draw histogram)
with the density function of a normal distribution.
\begin{minted}[]{R}
set.seed(12)
require(ggplot2)
n=4
S=100
sigma2=1/12
# we generate S realised values of X* (simu_Sbar) by drawing from Unif distribution
simu_Sbar= colSums(matrix(runif(n*S, min = 0, max = 1)-1/2,ncol=S))/sqrt(n)

ggplot(data.frame(simu_Sbar), aes(simu_Sbar)) +
geom_histogram(aes(y=..density..)) +
stat_function(fun=function(x)1/sqrt(2*pi*sigma2)*exp(-(x)^2/sigma2/2),
color=rgb(0.6, 0.2, 0.2, 0.35), size=2)
\end{minted}

Here are also Python codes
\begin{minted}{python}
import numpy as np
np.random.seed(10)

# now again this draws 4 values from Unif(0,1)
np.random.uniform(0,1,4)

n=4;
S=100;
rand_Srepeats= np.random.uniform(0,1,n*S).reshape(-1,S)
# how many more than 1 winner in S repeated games
np.sum(np.sum(rand_Srepeats==rand_Srepeats.max(axis=0), axis=0)>1)


# X* distribution analysis
n=4
S=100
sigma2=1/12

rand_Srepeats= np.random.uniform(0,1,n*S).reshape(-1,S) -1/2
simu_Sbar= np.sum(rand_Srepeats.reshape(-1,S),axis=0)/np.sqrt(n)
#  Histogram
import seaborn as sns
sns.distplot(simu_Sbar, hist=True, kde=True,
bins=int(180/5), color = 'darkblue',
hist_kws={'edgecolor':'black'},
kde_kws={'linewidth': 4})

# we add some normal density
import math
ax=sns.distplot(simu_Sbar, hist=True, kde=True,
bins=int(180/5), color = 'darkblue',
hist_kws={'edgecolor':'black'},
kde_kws={'linewidth': 4})
# calculate the pdf
sigma2=1/12
x0, x1 = ax.get_xlim()  # extract the endpoints for the x-axis
x_pdf = np.linspace(x0, x1, 100)

y_pdf = 1/(np.sqrt(2*math.pi*sigma2))*np.exp(-np.power(x_pdf,2)/(2*sigma2))

ax.plot(x_pdf, y_pdf, 'r', lw=2, label='pdf')
ax.legend()
\end{minted}


\end{Solution}

% }

\end{document}


%%% Local Variables:
%%% mode: latex
%%% TeX-master: "study-guide.tex"
%%% End:

% arara: pdflatex: { shell: yes }
% arara: pythontex: {verbose: yes, rerun: always }
% arara: pdflatex: { shell: yes }

\documentclass[bh_problems_check]{subfiles}

\opt{check}{
\Opensolutionfile{hint}
\Opensolutionfile{ans}
}

\begin{document}


\title{General Feedback}

\begin{document}
\maketitle

\section*{Assignment 1}
Below is some general feedback on the first assignment. If you have
any specific questions, feel free to ask them at the next tutorial.

\begin{itemize}
\item If a mathematical step is not generally true, add reasoning.
\item Make note of the support of a PDF, especially when constructing one.
\end{itemize}

Example:
$$\P{X=x, Y=y, N=n} = \P{X=x, Y=y}$$

Why does $N=n$ fall away: add reasoning.

Without mentioning ``if $x+y=n$'' or adding an indicator, the
statement is false.

\begin{itemize}
\item The exercises in the assignment explore the results from the BH
exercise. Keep this in mind while making the exercises and relate
back to the exercise in your answers when appropriate.
\end{itemize}

Example:

Many students did not realize that the PMF's in the final exercise
corresponded to $X\mid N=n$ from the BH exercise, where it was shown
that it has a uniform distribution.

If the code is difficult to interpret, keeping the BH exercise in
mind could make it easier.

\section*{Assignment 3}

\begin{itemize}
\item Part I
\end{itemize}

A short summary on what (parts of) the function does, helps the reader
with the rest of the explanation. A line by line description (of the
syntax) often does not convey the inner workings of an algorithm as
well as a few well-constructed sentences on a slightly higher level of
abstraction.

\begin{itemize}
\item Part II
\end{itemize}

Before you can explain what you see, you must understand what you see.
Why are the relations as they are, why does it make sense, what are
the implications? Show that you've thought about it.

Be precise in your descriptions. Not all convex/concave increasing
functions are logarithmic/exponential.

\end{document}

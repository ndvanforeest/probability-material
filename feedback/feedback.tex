% arara: pdflatex: { shell: yes }
% arara: pythontex: {verbose: yes, rerun: always }
% arara: pdflatex: { shell: yes }

\input{header}

\title{General Feedback}

\begin{document}
\maketitle
Below is some general feedback on the assignments. If you have any
specific questions, feel free to ask them at the next tutorial.

\section*{Assignment 1}

\begin{itemize}
\item If a mathematical step is not generally true, add reasoning.
\item Make note of the support of a PDF, especially when constructing one.
\end{itemize}

Example:
$$\P{X=x, Y=y, N=n} = \P{X=x, Y=y}$$

Why does $N=n$ fall away: add reasoning.

Without mentioning ``if $x+y=n$'' or adding an indicator, the
statement is false.

\begin{itemize}
\item The exercises in the assignment explore the results from the BH
exercise. Keep this in mind while making the exercises and relate
back to the exercise in your answers when appropriate.
\end{itemize}

Example:

Many students did not realize that the PMF's in the final exercise
corresponded to $X\mid N=n$ from the BH exercise, where it was shown
that it has a uniform distribution.

If the code is difficult to interpret, keeping the BH exercise in
mind could make it easier.

\section*{Assignment 2}
\begin{itemize}
\item Integral sign order
\item Indirectly assuming independence
\item Exponential PDF
\end{itemize}
Examples:

Many people had a notation that was equivalent to:
$$\int_{supp(A)}\int_{supp(B)} a \cdot b \cdot f_{A,B}(a, b)\d a \d b$$
This is not correct, as this can be interpreted as:
$$\int_{supp(A)} \left ( \int_{supp(B)} a \cdot b \cdot f_{A,B}(a, b)\d a \right )\d b$$

Hopefully it becomes apparent now why this is not correct.
The correct notation would be:
$$\int_{supp(A)}\int_{supp(B)} a \cdot b \cdot f_{A,B}(a, b)\d b \d a$$

Students also often indirectly assume independence, by using either of the following results:
$$ \E{XY} = \E{X}\E{Y}$$
$$ \P{A \cup B} = \P{A} \P{B}$$ Whenever using these results, make
sure that independence is satisfied. Otherwise, these results will
generally not hold.

In exercise 5.13, the area variable was calculated in the following
way:

$$\mathrm{value} = \int_0^5 \int_0^1 \1{b - 0.25 < a} \1{a < b + 0.25} e^{-\lambda b}\d a\d b$$
$$\mathrm{area} = \lambda \cdot value$$

Many students saw $e^{-\lambda d}$ inside the integral as the PDF of
$B$. However, we must note that the PDF of an exponentially
distributed rv is given by $f(b) = \lambda e^{-\lambda d}$. In our
case, $\lambda$ was taken out of the integral and multiplied in the
next line.


\section*{Assignment 3}
There is no need to include pictures of the handwritten solutions in
the assignment, hand them in on paper.

\begin{itemize}
\item Part I
\end{itemize}

Add a short summary on what the function does, it helps with the rest
of the explanation. A line by line description often does not convey
the inner workings of an algorithm as well as a few well-constructed
sentences on a slightly higher level of abstraction.

\begin{itemize}
\item Part II
\end{itemize}

We asked to explain the graph, not just to describe it. Before you can
explain what you see, you must understand what you see. Why are the
relations as they are, why does it make sense, what are the
implications? Show that you've thought about it.

Be precise in your descriptions. Not all convex/concave increasing
functions are logarithmic/exponential.

\section*{Assignment 4}

\begin{itemize}
\item Relate the code to the exercise
\end{itemize}

Example:

In exercise 5.2, students are asked to explain the code given in the
exercise. Many students literally explain what the code does. However,
linking the code to the exercise we are trying to solve is important.
What is the meaning of N? It is the number of ping pong balls that fit
into a Beluga, given the respective draws of their volume from the
Normal distribution.

\begin{itemize}
\item Note on plotting the PDF 
\end{itemize}

In exercise 5.4, you were asked to provide code to make a histogram of
the \emph{PDF} of N. Plotting the density (rather than the frequency)
on the y-axis is more appropriate.

Furthermore, although the figure was not explicitly requested, for
future assignments, please include them even when not explicitly asked
for. No points were deducted for not including the figure this week.

\section*{Assignment 5}

\begin{itemize}
\item The choice of $\mu = 1/2$ and $\sigma^2 = 1/12$
\end{itemize}

The $\Unif{0,1}$ distribution is the same as the $\Beta{1,1}$
distribution, a well-known fact about the Beta distribution. So by
taking $\mu = 1/2$ and $\sigma^2 = 1/12$, we know $a=b=1$. Working the
other way around works too: it's easy to calculate $\mu$ and
$\sigma^2$ for $a=b=1$, making it a good test case.

\begin{itemize}
\item Variance of an rv in $[0,1]$
\end{itemize}

From a conceptual point of view it is easy to see that the most
variance that can be obtained for an rv $X\in[0,1]$, is when $X$ is
Bernoulli distributed with probability $0.5$, giving a variance of
$0.25$. A proof for any rv $X\in [0, 1]$ could start with $X^2 < X$.
What kind of rv has the least variance?

\begin{itemize}
\item The commented check
\end{itemize}

Both within textbooks and software libraries there are different
conventions for parameterizing distributions. Especially when dealing
with parameters that can be interpreted as a rate you have to be
careful.

\begin{itemize}
\item What are these variables?
\end{itemize}

Total demand would refer to the sum of individual demands, not demand
conditional on the number of customers. 

\begin{itemize}
\item \verb|EXn[n]| $=  \E{X \given N=n}$
\item \verb|EXN| $=  \E{X \given N}$ 
\item \verb|VXn[n]| $= \V{X \given N=n}$
\item \verb|VXN| $=  \V{X \given N}$
\item \verb|VE| $=  \V{\E{X \given N}}$
\item \verb|EXN.mean()| $=  \E{\E{X \given N}} = \E{X}$
\end{itemize}


\end{document}

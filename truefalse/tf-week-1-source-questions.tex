\documentclass[tf-tutorial-all.tex]{subfiles}
\begin{document}

\setcounter{section}{0}
\section{TF questions PD Week 1}


\begin{truefalse}
$X$ is a rv, $X  \in \R$. Claim: $\supp{X} = (c, \infty) \implies \P{X\leq c} = 0$.
\begin{solution}
Yes.
\end{solution}
\end{truefalse}

\begin{truefalse}
Suppose $X$ a real-valued rv with $\supp X = [0, c]$. Claim:
\begin{equation}
\label{eq:3}
\V X = \E{X^2} - (\E{X})^2 \leq c \E X - (\E X)^2 = (c-\E X)\E X.
\end{equation}
\begin{solution}
Yes.
\end{solution}
\end{truefalse}


\begin{truefalse}
Let $X$ and $Y$ be independent rvs. Claim: $F_{X+Y}(x, y) = F_X(x) + F_{Y}(y)$.
\begin{solution}
No. We should write $F_{X,Y}$ rather than $F_{X+Y}$ , and since the rvs are independent, consider the product of the CDFs, not the sum.
\end{solution}
\end{truefalse}

\begin{truefalse}
Let $M_{X}(s)$ be the moment generating function of some rv $X$.  Claim:
\begin{equation}
\label{eq:2}
M_{X}(0) = 0.
\end{equation}
\begin{solution}
It's false.
\end{solution}
\end{truefalse}


\begin{truefalse}
Let $M_{X}(s)$ be the moment generating function of some rv $X$.  Claim:
\begin{equation}
\label{eq:2}
\left(\frac{\d}{\d s}\right)^{2 }M_{X}(s)|_{s=0} =  \V X + (\E{X})^{2}.
\end{equation}
\begin{solution}
It's true.
\end{solution}
\end{truefalse}


\begin{truefalse}
We have two positive rvs $X$ and $Y$. Claim: $\V{X+Y} = \V X + \V Y$.
\begin{solution}
It's false, it's not given that $X$  and $Y$ are independent.
\end{solution}
\end{truefalse}



\begin{truefalse}
We have two independent positive rvs $X$ and $Y$. Claim: $M_{2X+Y}(s)  = (M_{X}(s))^{2} M_{Y}(s)$.
\begin{solution}
It's false in general, because $X$ is not independent of itself.
\end{solution}
\end{truefalse}




\begin{truefalse}
People enter a shop such that the time $X$ between  any two consecutive customers is $X\sim \Exp{\lambda}$ with $\lambda=10$ per h. Claim: $\P{X > x} = e^{-\lambda x}$, for $x\geq 0$.
\begin{solution}
It's correct, see BH.5.45.
\end{solution}
\end{truefalse}

\begin{truefalse}
People enter a shop such that the time $X$ between any two consecutive customers is $X\sim \Exp{\lambda}$ with $\lambda=10$ per h.
Assume that the interarrival times between customers are iid.
Let $N(t)$ be the number of people that enter during a time interval of length $t$.
Claim $N(t) \sim \Pois{\lambda}$.
\begin{solution}
It's false, see BH.5.45. It's $\sim \Pois{\lambda t}$.
\end{solution}
\end{truefalse}

\begin{truefalse}
People enter a shop such that the time $X$ between any two consecutive customers is $X\sim \Exp{\lambda}$ with $\lambda=10$ per h.
Assume that the interarrival times between customers are iid.
Let $N(t)$ be the number of people that enter during a time interval of length $t$.
Suppose that $T_{3}$ is the time the third person enters.
Claim: $\P{N(t) <3} = \P{T_{3}>t}$.
\begin{solution}
It's true.
\end{solution}
\end{truefalse}





\begin{truefalse}
Write $m$ for the median $\E X$ of the rv $X$. Claim, the following definition correct:
\begin{equation*}
\V X := \E{X^2} - m^{2}.
\end{equation*}
\begin{solution}
False, $\E X$ need not be equal to the median, moreover the \emph{definition} of the variance involves the mean, not the median.
\end{solution}
\end{truefalse}


\begin{truefalse}
    For two events $A$ and $B$. Claim: $\E{\1{A}\1{B}} = P(A) + P(B) - P(A \cup B)$
\begin{solution}
It's True
\end{solution}
\end{truefalse}

\begin{truefalse}
For an unfair 4-sided die that throws 4 half of the time and 1 to 3 with equal probability.
If the rv
$X$ denotes the thrown value of the dice.
Claim: $\E{X^2} = \frac{38}{3}$
\begin{solution}
It's False. Calculating using LOTUS gives $\E{X^2} = \frac{1}{2} 4^{2} + \frac{1}{2\cdot3}(1 + 4 + 9) = \frac{31}{3}$
\end{solution}
\end{truefalse}

\begin{truefalse}
For a degenerate rv $X$ and $c$ an arbitrary, non-zero constant. Claim: $\V{cX} > 0$
\begin{solution}
It's False. the variance of a degenerate rv is always 0.
\end{solution}
\end{truefalse}

\begin{truefalse}
Assume that $\V{X} = \sigma^2$ and $\E{X^2} = a^2$ both exist and are finite. Claim:$ \E{X} = \sqrt{a^2-\sigma^2}$
\begin{solution}
It's True.
\end{solution}
\end{truefalse}

\begin{truefalse}
Given $\V{X+Y} = \V{X} + \V{Y}$. Claim: $X$ and $Y$ are independent,
\begin{solution}
It's False. Independence is sufficient but not necessary for the equality to hold.
\end{solution}
\end{truefalse}

\begin{truefalse}
For two rvs $X$ and $Y$, where $Y$ is always equal to $X$, given $\V{X}>0$. Claim: $\V{X+Y} = \V{X} + \V{Y}$
\begin{solution}
It's False, since if $Y$ is always equal to $X$ they are definitely not independent. (see BH. p. 172) In fact, as $Y=X$, $\V{X+Y} = \V{2X} = 4 \V X}$. Isn't it a bit counter intuitive that when $X$ and $Y$ are  dependent like this, the variance is larger than if they would be independent?
\end{solution}
\end{truefalse}




\end{document}

\documentclass[tf-tutorial-all.tex]{subfiles}
\begin{document}



\begin{truefalse}
Claim: $\E{Y|A} = \sum_{y=0}^{\infty} y \P{Y=y|A}$  if $Y$ is discrete.

\begin{solution}
No, what if $Y$ can take negative values?
\end{solution}
\end{truefalse}

\begin{truefalse}
Claim: $\E{Y|A} = \sum_{y=-\infty}^{\infty} y \P{Y=y|A} = 0$  if $\P{A}=0$ and $Y$ is discrete.

\begin{solution}
It is false, the definition in BH requires that $\P{A}>0$.
\end{solution}
\end{truefalse}


\begin{truefalse}
Let $X$ be a continuous rv with PDF $f_{X}(x) > 0$ on $x\in[a, b]$ and $Y$ discrete.
Claim: when $x\in [a, b]$, $\E{Y|X=x} = \sum_{y=-\infty}^{\infty} y \frac{f_{X,Y}(x,y)}{f_{X}(x)}$.

\begin{solution}
Yes.
\end{solution}
\end{truefalse}

\begin{truefalse}
Let $X_{(i)}, i = 1, \ldots, n$ be the order statistic of $\{X_{i}, i=1, \ldots, n\}$.
Claim: $\P{X_{(j)}\leq x} = \sum_{k=0}^{n}{n \choose k} F(x)^k (1-F(x))^{n-k}$.

\begin{solution}
False, check theorem 8.6.3.
It's easy to check.
The LHS depends on $j$, the RHS not, hence it must be false, unless the probability would not depend on $j$, but that cannot be true.
\end{solution}
\end{truefalse}


\begin{truefalse}
Let $X_{(i)}, i = 1, \ldots, n$ be the order statistic of the continuous rvs $\{X_{i}, i=1, \ldots, n\}$.
Claim: $f_{(j)}(x) \d x = n f(x) \d x {n \choose j} F(x)^j (1-F(x))^{n-j}$.
\begin{solution}
It's false, theorem 8.6.4.
\end{solution}
\end{truefalse}



\begin{truefalse}
Claim: It is a good idea to conceptualize the order statistic as a set rather than as a list.
\begin{solution}
No. Elements in a set are not ordered, elements in a sequence or list are.
\end{solution}
\end{truefalse}


% \begin{truefalse}
% Claim: The order statistic is a sorted set of numbers.

% \vspace{0.5cm}\noindent  Which of the following options applies?
% \begin{enumerate}
% \item The claim is correct.
% \item The claim is false.
% \end{enumerate}
% % \begin{solution}
% % No. Elements in a set are not ordered, elements in a sequence or list are.
% % \end{solution}
% \end{truefalse}


\begin{truefalse}
Let $X_{(i)}, i = 1, \ldots, n$ be the order statistic of the continuous iid rvs $\{X_{i}, i=1, \ldots, n\}$.
Claim: $\E{X_{(i)} | X_{(j)}=x} \leq x$ for any $i\leq j$.
\begin{solution}
It's true.
\end{solution}
\end{truefalse}

\begin{truefalse}
Given two positive iid rvs $X$ and $Y$. Let $L=\min\{X,  Y\}, M = \max\{X, Y\}$. Claim: $\E{L|M=3}=3/2$.

\begin{solution}
It's false. For instance, take $X, Y \sim \Exp{10}$.
\end{solution}
\end{truefalse}

\begin{truefalse}
The continuous rvs $\{X_{i}\}$ with support on $(0, \infty)$ are idd, and $S_n=\sum_{i=1}^n X_{i}$.
Claim: for some $x$ and $n$,
\begin{equation}
\label{eq:6}
\E{X_{n}| S_{n-1} = x} = S_n/n.
\end{equation}
\begin{solution}
It's false because the condition is on $x$ being given. On the LHS we have an $x$, but on the RHS there is no $x$.
\end{solution}
\end{truefalse}

\begin{truefalse}
Take $g(x) = \E{Y|X=x}$.
Define the conditional expectation of the rv $Y$ given $X$ as $g(X)$, and write it as $\E{Y|X}$.
Claim: this is one of the most important definitions in probability.

\begin{solution}
True. This is just a bonus to stress the importance of the definition of conditional expectation.
\end{solution}
\end{truefalse}


\end{document}

\documentclass[lectures]{subfiles}
\begin{document}

\section{Lecture 9}
\label{sec:lecture-9}




\begin{exercise}
The lifetime $X$ of a machine is $\Exp{\lambda}$.
Compute $\E{X \given X\leq \tau}$ where $\tau$ is some positive constant.

Define $Y=\1{X\leq \tau}$.  Use  Adam's law and LOTP to show that $\E{X} = \E{\E{X\given Y}} = 1/\lambda$. Observe that this is just a check on our results.
\begin{solution}
Since $X\sim \Exp{\lambda}$,
\begin{equation*}
\P{X\leq t | X\leq \tau} = \frac{\P{X\leq \min\{t, \tau\}}}{\P{X\leq \tau}} = \frac{1-e^{-\lambda\min\{t, \tau\}}}{1-e^{-\lambda \tau}}.
\end{equation*}
 Defining
\begin{equation*}
  f(t) = \partial_{t} \P{X\leq t | X\leq \tau} = \lambda \frac{e^{-\lambda t}}{1-e^{-\lambda \tau}} \1{0\leq t \leq \tau},
\end{equation*}
we get
\begin{align}
  \label{eq:9925}
\E{X|X\leq \tau}
&= \int t f(t) \d t \\
&= \frac{1}{1-e^{-\lambda \tau}}\int_{0}^{\tau} \lambda t e^{-\lambda t} \d t.
\end{align}
Simplifying,
\begin{align}
\label{eq:926}
\int_0^{\tau} \lambda t e^{-\lambda t} \d t
&= \left.- t e^{-\lambda t}\right|_{0}^{\tau} + \int_0^{\tau} e^{-\lambda t} \d t \\
&= -\tau e^{-\lambda \tau} + \frac{e^{-\lambda \tau} -1}{-\lambda} \\
&= \frac{ 1- e^{-\lambda \tau}- \tau \lambda  e^{-\lambda \tau}}{\lambda},
\end{align}
\begin{align}
  \label{eq:925}
\E{X|X\leq \tau}
&= \frac{ 1- e^{-\lambda \tau}- \tau \lambda  e^{-\lambda \tau}}{\lambda(1-e^{-\lambda \tau})}.
\end{align}

For part 2: by memorylessness, $\E{X\given X>\tau} = \tau + 1/\lambda$. Hence,
\begin{align}
  \label{eq:94}
\E X
&= \E{\E{X\given Y}}, \quad \text{Adam's law} \\
&= \E{X\given Y=1}\P{Y=1} + \E{X\given Y=0}\P{Y=0}, \quad \text{LOTE}\\
&= \E{X\given X\leq \tau}\P{X\leq \tau} + \E{X\given X>\tau} \P{X>\tau} \\
&= \frac{ 1- e^{-\lambda \tau}- \tau \lambda  e^{-\lambda \tau}}{\lambda(1-e^{-\lambda \tau})} (1-e^{-\lambda \tau}) + (\tau+ 1/\lambda) e^{-\lambda \tau} \\
&= 1/\lambda - e^{-\lambda \tau}/\lambda - \tau e^{-\lambda \tau} + \tau e^{-\lambda T} + e^{-\lambda \tau}/\lambda \\
&=1/\lambda.
\end{align}

Here is a final point for you to think hard about: ${X\leq \tau}$ is an event, $Y=\1{X \leq \tau}$ is a r.v. Therefore the following is sloppy notation: $\E{\E{X\given X\leq \tau}}$, and when using, it is easy to get confused, because where is the random variable we should have according to Adam's law?
\end{solution}
\end{exercise}


\begin{exercise}
We have a station with two machines, one is working, the other is off.
If the first fails, the other machine takes over.
The repair time of the first machine is a constant $\tau$.
If the second machine fails before the first is repaired, the station stops working, i.e., is `down'.
Use a conditioning argument to find the expected time $\E T$ until the station is down when the lifetimes of both machines is iid $\sim\Exp{\lambda}$.
\begin{solution}
Let $X_i$ be the lifetime of machine $i$. Suppose that  the second machine does not break down before $\tau$, then
\begin{equation}
\E{T| X_{2}>\tau} = \E{X_{1} + \tau + C |X_2>\tau} = 1/\lambda + \tau + \E T,
\end{equation}
because the expected time until machine 1 fails is $1/\lambda$, then we wait until it is repaired, and then, by memoryless, a new cycle $C$ starts, which has an expected duration of $\E T$.

Next condition on the second machine breaking down before $\tau$. Then, since $X_{1}$ and $X_2$ are independent and by the previous exercise,
\begin{equation}
\E{T| X_{2}\leq \tau} = \E{X_{1} + X_{2}  |X_2\leq \tau} = \frac{1}{\lambda} + \frac{ 1- e^{-\lambda \tau}- \tau \lambda  e^{-\lambda \tau}}{\lambda(1-e^{-\lambda \tau})}.
\end{equation}

By LOTE and using $\P{X_{2}\geq \tau} = e^{-\lambda \tau}$,
\begin{align}
  \label{eq:924}
\E T
&=
\E{T| X_{2}\leq \tau} \P{X_{2}\leq \tau} + \E{T| X_{2}\geq \tau} \P{X_{2}\geq \tau} \\
&=(1-e^{-\lambda \tau})\left( \frac 1\lambda+
\frac{ 1- e^{-\lambda \tau}- \tau \lambda  e^{-\lambda \tau}}{\lambda(1-e^{-\lambda \tau})} \right)
+e^{-\lambda \tau}\left( \frac{1}{\lambda} + \tau + \E T \right) \\
&=\frac{1}{\lambda} + \frac{ 1- e^{-\lambda \tau}- \tau \lambda  e^{-\lambda \tau}}{\lambda}
+e^{-\lambda \tau} \tau + e^{-\lambda \tau}\E T \\
&\implies \\
\left( 1-e^{-\lambda \tau} \right) \E T
&=\frac{2-e^{-\lambda \tau}}{\lambda} \\
&\implies \\
\E T
&=\frac{2-e^{-\lambda \tau}}{\lambda(1-e^{-\lambda \tau})}.
\end{align}

For you to do: Check the limits $\tau=0$ and $\tau=\infty$. Is the result in accordance with your intuition?

Two challenges: Does the problem become much harder when $X_{1}$ and $X_{2}$ are still exponential, but have different failure rates $\lambda_1$ and $\lambda_{2}$? Can you extend the analysis to 3 machines, or $n$ machines?

\end{solution}
\end{exercise}

\begin{exercise}
We draw, with replacement, balls, numbered 1 to $N$, from an urn.
Find a recursion to compute the expected number $\E T$ of draws necessary to see all balls.
\begin{solution}
Write $T_{n}$ for expected time to finish given that we have seen $n$ different balls. Thus, $\E T = T_{0}$.
Then for $n< N$ and using conditioning and LOTE
\begin{align}
T_n
&= 1
+T_{n+1} \P{\text{draw a new ball}}
+T_{n} \P{\text{draw an old ball}}  \\
&= 1
+T_{n+1} \frac{N-n}{N}
+T_{n} \frac{n}{N} \\
&\implies\\
T_n \frac{N-n}{N} &= 1 +T_{n+1} \frac{N-n}{N} \\
&\implies \\
T_n &= \frac{N}{N-n}  +T_{n+1}.
\end{align}
How about the boundary conditions, i.e., what is $T_{n}$ for $n=N$?
That's obvious: $T_{N} = 0$. Hence,
\begin{equation}
T_{N-1} = \frac{N}{N-(N-1)} + T_{N} = N,
\end{equation}
and so on.

If you are interested,  push the result a bit further: approximate $T_{0}$ for $N\gg 0$.
\end{solution}
\end{exercise}


\begin{exercise}
Write code to compute $\E T$ for $N=45$.
\begin{solution}
Here is the python code.

\begin{pyblock}
N = 45


def T(n):
    if n >= N:
        return 0
    return N / (N - n) + T(n + 1)

# print(T(44))
# print(T(0))
\end{pyblock}
\begin{minted}{R}
N = 45

bigT = function(n) {
  if(n >= N) {
    return(0)
  }
  return(N / (N - n) + bigT(n + 1))
}

#print(bigT(44))
#print(bigT(0))
\end{minted}
Here are the results $T(44) = \py{T(44)}$ (check!) and $T(0) = \py{T(0)}$.
\end{solution}
\end{exercise}

\begin{exercise}
We draw, with replacement, balls, numbered 1 to $N$, from an urn, but 6 at a time (not just one as in the previous exercise).
Find a recursion to compute the expected number $\E T$ of draws necessary to see all balls.
\begin{solution}
  Write $T_{n}$ for expected time to finish given that we have seen $n$ different balls.
  Take 6 balls.
  If we would know the number $k$ of new balls drawn, then $T_{n} = 1 + T_{n+k}$.
  What is the probability to draw $k$ new balls  out of the 6 we pick?
  This must be
\begin{equation}
\label{eq:930}
{n \choose 6-k}{N-n \choose k}\big/{N \choose 6}.
\end{equation}
Therefore, by LOTE,
\begin{align}
T_{n}
\approx
1 + \sum_{k=0}^{6} \frac{{n \choose 6-k}{N-n \choose k}}{{N \choose 6}} T_{n+k}.
\end{align}
But there is a slight problem, if there are just 2 new balls,  we cannot pick $k=6$ new balls. In general, we can pick $k=\min\{6, N-n\}$ new balls. Hence,
\begin{align}
T_{n}
&=
1 + \sum_{k=0}^{\min\{6, N-n\}} \frac{{n \choose 6-k}{N-n \choose k}}{{N \choose 6}} T_{n+k} \\
&\implies \\
T_{n} - \frac{{n \choose 6}{N-n \choose 0}}{{N \choose 6}} T_{n}
&=1 + \sum_{k=1}^{\min\{6, N-n\}} \frac{{n \choose 6-k}{N-n \choose k}}{{N \choose 6}} T_{n+k} \\
&\implies \\
T_{n}\left( {N\choose 6}  -{n \choose 6} \right)
&={N\choose 6} + \sum_{k=1}^{\min\{6, N-n\}}{n \choose 6-k}{N-n \choose k} T_{n+k} \\
\end{align}
And this is the final result.

\end{solution}
\end{exercise}

\begin{exercise}
For the previous exercise, compute $\E T$ for $N=45$.
\begin{solution}
  When we draw $m=6$ balls at a time, we should take care how we compute the recursion.

  To see this, consider for ease the case in which we draw two balls at at time.  Now I want to know how often each function gets called in the computation. For this, I write a simple helper function $S(n, N)$ that returns $1$ for being called plus $S(n+1, N) + S(n+2, N)$. Like this, I can find out how often the functions are called in the recursion.

\begin{pyblock}
def S(n, N):
    if n >= N:
        return 0
    return 1 + S(n + 1, N) + S(n + 2, N)


for N in range(1, 25):
    print(S(0, N))
\end{pyblock}
\begin{minted}{R}
S = function(n, N) {
  if(n >= N) {
    return(0)
  }
  return(1 + S(n + 1, N) + S(n + 2, N))
}

for (N in 1:24) {
  print(S(0, N))
}
\end{minted}
The result is this: \printpythontex.
Apparently, the number of functions called grows very rapidly.
When we want to compute for 6 balls and $N=45$ the situation must be much, much worse.

The way out is to \emph{store} intermediate results rather than computing them time and again.
For this we use the concept \emph{memoization}.
You should memorize this concept when dealing with the computation of recursions.
In short, this means that we check whether we computed the value of a function earlier.
If so, get the value from memory, otherwise, do the computation, and store it for later purposes.

Let's check what happens if we print out how often the function gets called when using memoization.
\begin{pyblock}
from functools import lru_cache

@lru_cache   # This realizes the memoization.
def S(n, N):
    if n >= N:
        return 0
    print(f"Called for n = {n}\n")
    return 1 + S(n + 1, N) + S(n + 2, N)


S(0, 5)
\end{pyblock}
\begin{minted}{R}
library("memoise")

S = function(n, N) {
  if(n >= N) {
    return(0)
  }
  cat("Called for n =", n, "\n")
  return(1 + S(n + 1, N) + S(n + 2, N))
}
S = memoise(S)

S(0, 5)
\end{minted}
Now the result is this:
\begin{center}
\printpythontex
\end{center}
This is much better, for each $n$ the function in the recursion gets called just once.

To convince you that memoization really works, let's remove the memoization.
\begin{pyblock}
def S(n, N):
    if n >= N:
        return 0
    print(f"Called for n = {n}\n")
    return 1 + S(n + 1, N) + S(n + 2, N)


S(0, 5)
\end{pyblock}
\begin{minted}{R}
S = function(n, N) {
  if(n >= N) {
    return(0)
  }
  cat("Called for n =", n, "\n")
  return(1 + S(n + 1, N) + S(n + 2, N))
}

S(0, 5)
\end{minted}
This is the result:
\begin{center}
\printpythontex
\end{center}
You see how often the function is called with $n=4$?



For the course you don't have to look up on the web how memoization is implemented, but I advice you to read about.
Understanding such ideas makes you much better at programming, which is important in view of the fact that many of you will have to use computers a lot in your profession careers.
Besides this, memoization lies at the heart of how spreadsheet programs such as excel work.

Now we know that we have to use memoization, we can turn to the balls and urn problem.
I build a general function $T(n,m)$ for the situation in which we have seen $n$ balls and we pick $m$ balls at a time.
Like this, I can check with the earlier case in which we picked just one ball by computing $T(n, 1)$.
(Hopefully you learn from this that you should also test your code.)


%\begin{pyblock}[][numbers=left]
\begin{pyblock}
from math import comb
from functools import lru_cache

N = 45


@lru_cache
def T(n, m):
    if n >= N:
        return 0
    res = comb(N, m)
    for k in range(1, min(m, N - n) + 1):
        P = comb(n, m - k)
        P *= comb(N - n, k)
        res += P * T(n + k, m)
    return res / (comb(N, m) - comb(n, m))

\end{pyblock}
\begin{minted}{R}
library("memoise")

N = 45

bigT = function(n, m) {
  if(n >= N) {
    return(0)
  }
  res = choose(N, m)
  for (k in 1:min(m, N - n)) {
    P = choose(n, m - k)
    P = P * choose(N - n, k)
    res = res + P * bigT(n + k, m)
  }
  return(res / (choose(N, m) - choose(n, m)))
}
bigT = memoise(bigT)

bigT(0, 1)
bigT(0, 6)
\end{minted}
Here is the check: $T(0, 1) = \py{T(0, 1)}$, which is the same as our earlier computation.

Next, $T(0, 6) = \py{T(0,6)}$.
This is more than $6$ as fast, i.e., $6\times T(0, 6) = \py{6*T(0, 6)} < T(0,1)$.
Why is that?
Well, the 6 balls we pick from the urn are guaranteed to be drawn without replacement.


Can you solve the following extension? Suppose for each time you draw a set of balls, you get the value (in Euro's say) of the numbers on the balls, and then you put the balls back in the urn.
The game stops when you have seen all balls at least once. What is your expected gain?

\end{solution}
\end{exercise}
\end{document}

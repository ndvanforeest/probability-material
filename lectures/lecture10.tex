%\documentclass[lectures-questions]{subfiles}
\documentclass[lectures-demo]{subfiles}
\begin{document}

\section{Lecture 10}

We have a wooden stick of length 100 cm that we break twice.
First, we break the stick at a random point that is uniformly distributed over the entire stick.
We keep the left end of the stick.
Then, we break the remaining stick again at a random point, uniformly distributed again, and we keep the left end again.

\begin{exercise}
What is the expected length of the stick we end up with?
\begin{solution}
Let $X \sim \Unif{0,100}$ denote the point where the first stick is broken (in cm). Let $S$ denote the point where the second stick is broken (i.e., the length of the remaining stick). Then, $S \sim \Unif{0,X}$. We want to compute $\E{Y}$. Using Adam's law we obtain
\begin{align}
    \E{S} &= \E{\E{S|X}} \\
    &= \E{X/2} \\
    &= \E{X}/2 \\
    &= 100/4 = 25.
\end{align}
Hence, we end up with a stick of expected length 25 cm.
\end{solution}
\end{exercise}

\begin{exercise}
Now we change the story slightly. Every time we break a stick, we keep the \textit{longest} part. What is the expected length of the remaining stick?
\begin{solution}
By symmetry we can assume that the point of the first break is $Y\sim\Unif{1/2, 1}$, in meters. By the same reasoning, the point of the second break, conditional on $Y=y$, is $Z|Y=y \sim \Unif{y/2, y}$. Then, $\E{Z|Y=y} = 3y/4$. Replacing $y$ by the rv $Y$, we have that $\E{Z|Y} = 3 Y/4$. With this, and Adam's law,
\begin{equation}
\E Z = \E{\E{Z|Y}} = \E{3Y/4} = 3/4\cdot \E{Y}= 3/4\cdot (1/2+1)/2 = (3/4)^{2},
\end{equation}
in meters.

\end{solution}
\end{exercise}



(Same story as last week; BH.8.4.5) Fred is waiting at a bus stop. He knows that buses arrive according to a Poisson process with rate $\lambda$ buses per hour. Fred does not know the value of $\lambda$, though. Fred quantifies his uncertainty of $\lambda$ by the \textit{prior distribution} $\lambda \sim \text{Gamma}(r_0, b_0)$, where $r_0$ and $b_0$ are known, positive constants with $r_0$ an integer.

Fred has waited for $t$ hours at the bus stop. Let $Y$ denote the number of buses that arrive during this time interval. Suppose that Fred has observed that $Y=y$.

\begin{exercise}
How many buses does Fred expect to observe? I.e., compute $\E{Y}$
\begin{solution}
Last week, we solved this question by first deriving the marginal distribution of $Y$ (which was negative binomial) and then computing the mean of this distribution. We obtained $\E{Y} = \frac{r_0}{b_0}t$.\\
Now, let's use conditional expectations instead. By Adam's law,
\begin{align}
    \E{Y} &= \E{\E{Y|\lambda}}.
\end{align}
Since $Y | \lambda \sim \text{Pois}(\lambda t)$, we have
\begin{align}
    \E{\E{Y|\lambda}} &= \lambda t.
\end{align}
Hence,
\begin{align}
    \E{Y} &= \E{\E{Y|\lambda}} \\
    &= \E{\lambda t} \\
    &= \E{\lambda} t \\
    &= \frac{r_0}{b_0} t.
\end{align}
\end{solution}
\end{exercise}

(BH.9.3.10).
An extremely widely used method for data analysis in statistics is \textit{linear regression}.
In its most basic form, the linear regression model uses a single explanatory variable $X$ to predict a response variable $Y$.
For instance, let $X$ be the number of hours studied for an exam and let $Y$ be the grade on the exam.
The linear regression model assumes that the conditional expectation of $Y$ is \textit{linear} in $X$:
\begin{align}
    \E{Y|X} = a + b X.
\end{align}

\begin{exercise}
Show that an equivalent way to express this is to write
\begin{align}
    Y = a + bX + \varepsilon,
\end{align}
where $\varepsilon$ is a random variable (called the \textit{error}) with $\E{\varepsilon|X} = 0$.
\begin{solution}
See the book.
\end{solution}
\end{exercise}

\begin{exercise}
Solve for the constants $a$ and $b$ in terms of $\E{X}$, $\E{Y}$, $\cov{X,Y}$, and $\V{Y}$.
\begin{solution}
See the book.
\end{solution}
\end{exercise}
\end{document}
